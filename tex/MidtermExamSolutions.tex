\documentclass[11pt]{paper}

%%%%%%%%%%%%%%%%%%%%%%%%%%%%%%%%%%%%%%%%
% Basic packages
%%%%%%%%%%%%%%%%%%%%%%%%%%%%%%%%%%%%%%%%
\usepackage{amsmath,amsthm,amssymb}
\usepackage[mathscr]{euscript}
\usepackage{mathtools}
\usepackage{etoolbox}
\usepackage{fancyhdr}
\usepackage{xcolor}
\usepackage{hyperref}
\usepackage{xspace}
\usepackage{comment}
%\usepackage{ulem}
\usepackage{url} % for url in bib entries
%\usepackage{mathrsfs}


\theoremstyle{remark}
\newtheorem{theorem}{Theorem}
\newtheorem*{prop}{Proposition}
\newtheorem{problem}{Problem}
\newtheorem*{prob}{Problem}
\newtheorem*{solution}{{\bf Solution}}
\newtheorem*{hint}{{\it Hint}}

%%%%%%%%%%%%%%%%%%%%%%%%%%%%%%%%%%%%%%%%%%%%%%%%%%
%% Surround the problem and solution with 
%% \begin{ProbBox}  and   \end{ProbBox}
%% to prevent pagebreaks.
\newenvironment{ProbBox}{\noindent\begin{minipage}{\linewidth}}{\end{minipage}}

\usepackage{enumerate}

\usepackage[
top    = 3cm,
bottom = 3cm,
left   = 3.00cm,
right  = 3.00cm]{geometry}

%%%%%%%%%%%%%%%%%%%%%%%%%%%%%%%%%%%%%%%%
% Fancy page style
%%%%%%%%%%%%%%%%%%%%%%%%%%%%%%%%%%%%%%%%
\pagestyle{fancy}
\newcommand{\metadata}[2]{
  \rhead{Midterm Exam}
  \chead{}
  \lhead{Math 700: Linear Algebra}
%  \lfoot{#1}\cfoot{#2}
  % \lfoot{}\cfoot{}
  % \rfoot{\thepage}
}
\renewcommand{\headrulewidth}{0.4pt}
\renewcommand{\footrulewidth}{0.4pt}

% \newrobustcmd*{\vocab}[1]{\emph{#1}}
% \newrobustcmd*{\latin}[1]{\textit{#1}}

%%%%%%%%%%%%%%%%%%%%%%%%%%%%%%%%%%%%%%%%
% Customize list enviroonments
%%%%%%%%%%%%%%%%%%%%%%%%%%%%%%%%%%%%%%%%
% package to customize three basic list environments: enumerate, itemize and description.
% \usepackage{enumitem}
% \setitemize{noitemsep, topsep=0pt, leftmargin=*}
% \setenumerate{noitemsep, topsep=0pt, leftmargin=*}
% \setdescription{noitemsep, topsep=0pt, leftmargin=*}

%%%%%%%%%%%%%%%%%%%%%%%%%%%%%%%%%%%%%%%%
%% Space between problems
\newrobustcmd*{\probskip}{\vskip5mm}

\usepackage{tikz}
\usetikzlibrary{matrix,arrows}
%% 
         \newcommand\alg[1]{\ensuremath{\mathbf{#1}}}
         \newcommand\0{\ensuremath{\mathbf{0}}}
         \newcommand{\<}{\ensuremath{\langle}}
         \renewcommand{\>}{\ensuremath{\rangle}}
         \renewcommand{\emptyset}{\ensuremath{\varnothing}}
         \newcommand\fld[1]{\ensuremath{\mathbb{#1}}}
         \newcommand\N{\ensuremath{\fld{N}}}
         \newcommand\R{\ensuremath{\fld{R}}}
         \newcommand\dom{\ensuremath{\operatorname{dom}}}
         \newcommand\cod{\ensuremath{\operatorname{cod}}}
         \newcommand\GF{\ensuremath{\operatorname{GF}}}
         \newcommand\End{\ensuremath{\operatorname{End}}}
         \newcommand\Hom{\ensuremath{\operatorname{Hom}}}
         \newcommand{\Aff}{\ensuremath{\operatorname{Aff}}}
         \newcommand{\ann}[1]{\ensuremath{\operatorname{ann}(#1)}}
         \newcommand{\id}{\ensuremath{\operatorname{id}}}
         \newcommand{\nulity}[1]{\ensuremath{\operatorname{null}(#1)}}
         \renewcommand{\ker}[1]{\ensuremath{\operatorname{ker}(#1)}}
         \renewcommand{\dim}[1]{\ensuremath{\operatorname{dim}(#1)}}
         \newcommand\im[1]{\ensuremath{\operatorname{im}(#1)}}
         \newcommand{\rank}[1]{\ensuremath{\operatorname{rank}(#1)}}
         \newcommand{\trace}[1]{\ensuremath{\operatorname{trace}(#1)}}
         \renewcommand{\phi}{\ensuremath{\varphi}}
         \newcommand{\Sub}{\ensuremath{\operatorname{Sub}}}
         \renewcommand{\leq}{\ensuremath{\leqslant}}
         \renewcommand{\nleq}{\ensuremath{\nleqslant}}
         \renewcommand{\geq}{\ensuremath{\geqslant}}
         \renewcommand{\lneq}{\ensuremath{\lneqslant}}
         \renewcommand{\gneq}{\ensuremath{\gneqslant}}
         \renewcommand{\ngeq}{\ensuremath{\ngeqslant}}
         \newcommand{\sD}{\ensuremath{\mathscr{D}}}
         \newcommand{\sS}{\ensuremath{\mathscr{S}}}
         \newcommand{\sC}{\ensuremath{\mathscr{C}}}
         \newcommand{\sA}{\ensuremath{\mathscr{A}}}


         \metadata{Math 700}{Midterm Exam -- 2014/03/07}
         \author{}
%%
%%    8. Update the title and date as appropriate.
         \title{Midterm Exam: Solutions}
         \subtitle{Math 700: Spring 2014}
         \date{6 March 2014}



\begin{document}

\maketitle
\vskip-1cm

\noindent {\bf INSTRUCTIONS:} 
\begin{itemize}
\item 
Solve the problems below. Write up your solutions (neatly!),
giving complete justifications for all arguments, and turn in a hard copy of your
solutions in class on the\\[4pt]
Due Date: {\bf Wednesday, March 19}

\medskip

\item The questions are meant to test your understanding of elementary concepts, and
you should write down definitions of any technical terms you use, even if these
terms are mentioned in the statement of the problem. Of course, you must use
your best judgment about which definitions to state.  (You probably don't want
to define the integers or real numbers, for example.)

\medskip

\item It will help me (and probably your grade) if you do the following:
\begin{enumerate}
\item 
State what you are trying to prove.
\item 
Mention informally how you plan to prove it before giving the details.
\item If you believe your proof is complete, use an end-of-proof symbol (like
  QED or \qedsymbol); on the other hand, if you believe your proof is
  incomplete, say so.
\end{enumerate}
\end{itemize}

\noindent {\bf HONOR CODE:} You are expected to solve the exam problems on your own
  with no outside help.  You may consult the lecture notes and textbook for this
  course only.  No other books or internet usage is allowed.
  If you get stuck, please ask \emph{me} for help, and I may post hints on our wiki page. 


\medskip

\noindent When you finish the exam, please sign the following pledge:\\
\\
``On my honor as a student I,
\underline{\phantom{XXXXXXXXXXXXXXXX}}, have neither
given nor received unauthorized aid on this exam.''
\hbox{} \hskip .5in {\small (Print Name)}\\[4pt]
\begin{flushright} Signature: \underline{\phantom{XXXXXXXXXXXXXXXXXXXX}}
  Date: \underline{\phantom{XXXXXXXXXX}}
\end{flushright}

\medskip

\noindent {\bf NOTATION:}
For the most part, we follow the notation used in the textbook.
Recall that if $V$ is a vector space over the field $F$ and if
$c\in F$, then $\sigma_c v = cv$ for all $v \in V$.
In particular, $\sigma_0$ and $\sigma_1$ denote the zero and identity maps,
respectively. However, when $V$ and $W$ are vector spaces over the same field,
it is clearer to denote their identity maps by $\id_V$ and $\id_W$, resp.
We use $W\leq V$ to denote that $W$ is a subspace of $V$, 
whereas $W\subseteq V$, means that $W$ is a subset of $V$ 
(which may or may not be a subspace).  By an ``$F$-vector space'' we mean a
vector space over the field $F$. If $\phi : V \rightarrow W$, then
$\im{\phi} := \phi(V)$, $\ker{\phi} := \{v \in V : \phi(v) = 0_W\}$,
$\rank{\alpha} := \dim{\im{\alpha}}$, and $\nulity{\alpha} := \dim{\ker{\alpha}}$.
\probskip

\newcommand{\btrue}{\boxed{\mathrm{true}}}
\newcommand{\bfalse}{\boxed{\mathrm{false}}}
% \newcommand{\strue}{\sout{true}}
% \newcommand{\sfalse}{\sout{false}}
\newcommand{\strue}{true}
\newcommand{\sfalse}{false}

%%%%%%%%%%%%%%%%%%%%%%%%%%%%%%%%%%%%%%%%%%%%%%%%%%%%%%%%%%%%%%%%%%%%%%%%%%%%
\begin{problem} % Problem 1
Let $V$ and $W$ be finite dimensional vector spaces over the field $F$, and
suppose $\alpha \in \Hom(V,W)$.
Circle true or false, where true means ``always true'' (no proof required):
\begin{enumerate}[(a)]
\item If $\alpha(v) = 0_W$ only when $v=0_V$, then $\dim{V} = \dim{W}$.
 \hfill \strue \hskip1cm \bfalse
\item If $\im{\alpha} = \{0_W\}$, then $\alpha = \sigma_0$.
 \hfill \btrue \hskip1cm \sfalse
\item If $\dim{V} = \rank{\alpha}$, then $\ker{\alpha} = \{0_V\}$.
 \hfill \btrue \hskip1cm \sfalse
\item $\ker{\alpha}\leq \ker{\alpha^2}$ (assuming $W=V$)
 \hfill \btrue \hskip1cm \sfalse
\item $\im{\alpha}\geq \im{\alpha^2}$ (assuming $W=V$)
 \hfill \btrue \hskip1cm \sfalse
\item $\nulity{\alpha} \leq \rank{\alpha}$
 \hfill \strue \hskip1cm \bfalse
\item $\nulity{\alpha} \leq \dim{V}$
 \hfill \btrue \hskip1cm \sfalse
\item $\alpha$ is a one-to-one if and only if $\ker{\alpha} = \{0_V\}$.
 \hfill \btrue \hskip1cm \sfalse
\item $\alpha$ is a one-to-one if and only if $\dim{V} \leq \dim{W}$.
 \hfill \strue \hskip1cm \bfalse
\item $\alpha$ is a one-to-one if and only if $\nulity{\alpha} = 0$.
 \hfill \btrue \hskip1cm \sfalse
\item $\alpha$ is a onto if and only if $\dim{V} \geq \dim{W}$.
 \hfill \strue \hskip1cm \bfalse
\item $\alpha$ is a onto if and only if $\rank{\alpha} =\dim{W}$.
 \hfill \btrue \hskip1cm \sfalse
\end{enumerate}
\end{problem}

\probskip

%%%%%%%%%%%%%%%%%%%%%%%%%%%%%%%%%%%%%%%%%%%%%%%%%%%%%%%%%%%%%%%%%%%%%%%%%%%%
\begin{problem} % Problem 2
Prove that the lattice of subspaces of a vector space
is modular but not necessarily distributive, as follows:
Let $U$, $Y$, and $W$ be subspaces of a vector space $V$. 
\begin{enumerate}[(a)]
\item Show that $U\cap (Y + (U \cap W)) = (U \cap Y) + (U \cap W)$.
\item Show that $U \cap (Y + W) = (U \cap Y) + (U \cap W)$ is not always valid.
\end{enumerate}
\end{problem}
\begin{solution}
To prove (a), we show the right hand side is contained in the left and
vice-versa.
First note that
$U \cap Y \subseteq U\cap (Y + (U \cap W))$ and 
$U \cap W \subseteq U\cap (Y + (U \cap W))$.
Therefore, since $(U \cap Y) + (U \cap W)$ is the smallest subspace containing both 
$U \cap Y$ and $U \cap W$, we see that the right hand side is contained in the left.
On the other hand, suppose $x \in U\cap (Y + (U \cap W))$.
Then $x = u = y + z$ with $u \in U$, $y\in Y$, and $z\in U \cap W$.
Whence, $y = u-z \in U \cap Y$, so $x = y+z$ belongs to the right hand side.
The equation in part (a) is called the \emph{modular law}, and we have
just shown that, for any vector space $V$, the lattice 
$\<\Sub(V), \cap, +\>$ of subspaces of
$V$ is a \emph{modular lattice}.

To prove (b), consider the following subspaces of $\R^2$:
\[
U = \left\{
\begin{pmatrix} a\\ 0 \end{pmatrix} : a\in \R \right\}, \quad 
Y = \left\{
\begin{pmatrix} b\\ b \end{pmatrix} : b\in \R \right\}, \quad 
W = \left\{
\begin{pmatrix} 0\\ c \end{pmatrix} : c\in \R \right\}.
\]
Let us denote the zero vector of $\R^2$ by 
$\0 = (0, 0)$.
Note that any pair of (distinct) subspaces chosen from the set
$\{U, Y, W\}$ intersects at $\{\0\}$ and sums to $\R^2$.
Therefore, the left hand side of the equation in (b) is
$U \cap (Y + W) = U \cap \R^2 = U$, while the right hand side is 
\[
(U\cap Y) + (U\cap W) = \{\0\} + \{\0\}
= \{\0\}.
\]
The equation in part (b) is called the \emph{distributive law} and we have just shown
that the lattice $\<\Sub(V), \cap, +\>$ of subspaces of a vector space is not
always a \emph{distributive lattice}.

\end{solution}

\probskip

%%%%%%%%%%%%%%%%%%%%%%%%%%%%%%%%%%%%%%%%%%%%%%%%%%%%%%%%%%%%%%%%%%%%%%%%%%%%
\begin{problem}  % Problem 3
\label{prob:zorn}
Prove that every nontrivial vector space $V$ has a basis.\\[4pt]
[{\it Hint:} First prove that
every linearly independent subset of $V$ is contained in a basis.
As we did in class, let $S$ be a linearly independent subset and
  let $\sS$ be the set of all linearly independent subsets that contain $S$. 
  Partially order $\sS$ by inclusion $\subseteq$ and apply
  Zorn's Lemma.\footnote{{\bf Zorn's Lemma:} 
    If a partially ordered set $(\sS, \subseteq)$ has the property that every
    chain $S_1 \subseteq S_2 \subseteq \cdots$ has an upper bound in $\sS$, 
    then $\sS$ contains a maximal element.}
  Finally, say why if follows from this that every vector space has a basis.]
\end{problem}

\begin{solution}
Let $S$ be a linearly independent subset of $V$ and
  let $\sS$ be the set of all linearly independent subsets that contain $S$. 
 Partially order $\sS$ by set inclusion to obtain a poset 
 $(\sS, \subseteq)$.  To apply
  Zorn's Lemma, we must check that 
every chain in $\sS$ has an upper bound in $\sS$.
Fix a chain $\sC = \{S_\alpha : \alpha \in \sA \}$ in 
$\sS$.  Let $B = \bigcup \sC$. Since every $S_\alpha \in \sC$ is contained in  
$B$, it's clear that $B$ is an upper bound for $\sC$, with respect to
$\subseteq$.
We must show that $B$ belongs to $\sS$. Since $S \subseteq B$ is obvious, we
just need to check that $B$ is linearly independent.  Let 
$b_1, \dots, b_n$ be elements of 
$B$. Then there exist $\alpha(1), \dots, \alpha(n)$ such that 
$b_i \in S_{\alpha(i)}$.  Let $N = \max\{\alpha(i) : 1\leq i \leq n\}$. 
Since $\sC$ is a chain, all the elements $b_1, \dots, b_n$ belong to $S_N$.
Since $S_N$ is linearly independent, the $b_1, \dots, b_n$ are linearly
independent. Now $b_1, \dots, b_n$ was an arbitrary finite collection of vectors
from $B$, so $B$ is linearly independent.
Thus, the hypotheses of Zorn's Lemma are satisfied, and we see that $\sS$
has a maximal element; that is, a maximal linearly independent set that contains
$S$. 
Finally, note that $\emptyset$ is a linearly
idependent subset of every vector space.  So, by the foregoing argument with $S$
replaced by $\emptyset$, we see that every vector space contains a maximal linearly
independent set (i.e., a basis).
\end{solution}

\probskip

%%%%%%%%%%%%%%%%%%%%%%%%%%%%%%%%%%%%%%%%%%%%%%%%%%%%%%%%%%%%%%%%%%%%%%%%%%%%
\begin{problem}  % Problem 4
Let $V$ be a vector space over the field $F$, and suppose the subset 
$S \subseteq V$ satisfies $FS = V$. Prove that $S$ contains a basis.
\end{problem}

\probskip

%%%%%%%%%%%%%%%%%%%%%%%%%%%%%%%%%%%%%%%%%%%%%%%%%%%%%%%%%%%%%%%%%%%%%%%%%%%%
\begin{problem}  % Problem 5
Let $V$ be a vector space over the field $F$ and let $\Omega$ be a (possibly
uncountable) set.  Describe $V^\Omega$.  Can you make 
$V^\Omega$ into an  $F$-vector space?  An $F$-algebra? Explain.
\end{problem}
\begin{solution}
The set $V^\Omega$ is the set of all functions from $\Omega$ into $V$. 
That is, $V^\Omega = \{f : \Omega \rightarrow V\}$. We can make $V^\Omega$ into a
vector space by defining
%  a binary addition operation, 
% $+ : V^\Omega  \times  V^\Omega  \rightarrow V^\Omega$, and, for each $r \in F$, 
% a unary scalar multiplication, $r : V^\Omega \rightarrow V^\Omega$, as follows:
\begin{itemize}
\item {\it addition:} 
  $\forall \,f, g \in V^\Omega$, define 
  $f + g \in V^\Omega$ to be the map $(f+g):\omega \mapsto f(\omega) +^V
  g(\omega)$.
(Here $f(\omega) +^V g(\omega)$ means addition in $V$, which is available by the initial assumption that $V$ is a vector space.)
\item {\it additive identity:}
  $\sigma_0 \in V^\Omega$ is the map $\sigma_0:\omega \mapsto 0_V$.
\item   {\it additive inverse:}
  $\forall \, f\in V^\Omega$, define $-f \in V^\Omega$ by
  $(-f)(\omega) = - f(\omega)$. (Again,
  interpret $-f(\omega)$ as additive inverse in the vector space $V$.)
\item {\it scalar multiplication:} $\forall \, r\in F, \; \forall \,f\in V^\Omega$ 
  define $rf \in V^\Omega$ to be the map  $(rf): \omega \mapsto r
  f(\omega)$ (where $r f(\omega)$ is scalar multiplication in
  $V$).
\end{itemize}

It is easy to see that $\<V^\Omega, +, -, \sigma_0\>$ is an abelian group,
since $+$ is a commutative and associative by virtue $+^V$ having these
properties, and since every $f \in V^\Omega$ has an inverse, namely $-f$, 
since $f + (-f) = \sigma_0$. % Explicitly, for every $\omega\in \Omega$
 % $(f + (-f))(\omega) = f(\omega) - f(\omega) = 0_v$.
Therefore, $V^\Omega$ is given the following vector space structure:
$\<V^\Omega, +, -, \sigma_0, \{r : r \in F\}\>$.  (Technically, we should also
check that scalar multiplication distributes over addition, and the other
vector space properties, but this is also straight forward.)

There's no obvious or ``natural'' way to make  $V^\Omega$ into an
$F$-algebra, unless we have more assumptions about $V$.   For example, we might
want to define a binary multiplication as  the ``pointwise'' product, 
$(f\circ g)(\omega) = f(\omega) \cdot g(\omega)$. However, it's unclear what
we mean by $f(\omega) \cdot g(\omega)$?  We need some 
notion of product $\cdot$ in the underlying vector space, $V$.  This would be
available if, 
for instance, we assumed $V$ itself were an $F$-algebra.  Alternatively, we
might consider defining multiply by function composition: $(f\circ g)(\omega) =
f(g(\omega))$. But for this to make sense, we need the domain of $f$ to contain
the range of $g$. If we assume $\Omega = V$ (e.g., 
$\Omega = V = F$), then we can
make $V^V$ into an $F$-algebra by defining product to be funciton composition.

\end{solution}
\probskip

%%%%%%%%%%%%%%%%%%%%%%%%%%%%%%%%%%%%%%%%%%%%%%%%%%%%%%%%%%%%%%%%%%%%%%%%%%%%
\begin{problem}  % Problem 6
Let $V$ and $W$ be vector spaces over the field $F$, and let 
$\phi : V \rightarrow W$ be a linear transformation. Give details and proof of
the following: $\phi$ is injective (respectively, surjective) if and only if 
there is a linear transformation $\psi: W \rightarrow V$ such that the
composition of $\phi$ with $\psi$ is the identity map.\\[4pt]
[{\it Hint:} There are two claims to prove, (a) ``$\phi$
  is injective iff...'', and (b) ``$\phi$ is surjective
  iff...'' There are two ways to form the composition, 
$\phi \psi = \id_W$ and $\psi \phi = \id_V$. Figure out which composition
you need to prove each claim.]
\end{problem}
\begin{solution} We prove the following claims 
  \begin{enumerate}[(a)]
\item
    $\phi\in \Hom(V, W)$ is injective iff 
    there exists $\psi \in \Hom(W, V)$ such that $\psi \phi = \id_V$.
  \item 
    $\phi\in \Hom(V, W)$ is surjective iff 
    there exists $\psi \in \Hom(W, V)$ such that $\phi \psi = \id_W$.
  \end{enumerate}
  \begin{enumerate}[(a)]
  \item ($\Leftarrow$) Assume there exists $\psi: W\rightarrow V$ such that
    $\psi \phi = \id_V$, and fix $v_1, v_2 \in V$. 
    If $\phi(v_1) = \phi(v_2)$, then $v_1 =  \id_V(v_1) = \psi \phi(v_1) =\psi \phi(v_2) =
    \id_V(v_2) =  v_2$. This proves that $\phi$ is one-to-one.

($\Rightarrow$) Assume $\phi$ is one-to-one.  Let $A$ be a basis for
    $V$.  Then, since $V$ is a homomorphism, $\phi(A) = B$ is a basis for the
    subspace $\im \phi \leq W$.  Extend $B$ to a basis $B\cup C$ for $W$.
    Define $\tilde{\psi}: B\cup C \rightarrow V$ as follows: for $w\in B\cup C$,
\[
\tilde\psi (w) =
\begin{cases}
  \phi^{-1}(w), & \text{ if $w \in B$,}\\
0_V, & \text{ if $w \in C$.}
\end{cases}
\]
Define $\psi: W \rightarrow V$ by linear extension of $\tilde \psi$.  That is,
given an arbitrary $w \in W$, which has a basis decomposition 
$w = \sum_B w(b) b +\sum_C w(c) c$, define $\psi \in V^W$ by 
\[
\psi(w) = \sum_{b\in B} w(b) \tilde{\psi}(b) +\sum_{c\in C} w(c)\tilde{\psi}(c)
 = \sum_{b\in B} w(b) \phi^{-1}(b) + 0_V.
\]
We claim $\psi \phi = \id_V$. Indeed, fix an arbitrary $v \in V$, which has
a unique basis decomposition as $v = \sum_A v(a) a$.  Then, $\phi(v)
= \sum_A v(a) \phi(a)$, so 
\[
\psi\phi(v) = 
\sum_{a\in A} v(a)\psi \phi(a) = 
\sum_{a\in A} v(a)\tilde{\psi} \phi(a) =
\sum_{a\in A} v(a)\phi^{-1}\phi(a)=
\sum_{a\in A} v(a)a = v.
\]
Therefore, $\psi \phi = \id_V$.

%Notice that, for this (or any other) $w\in W$, we have $\psi

  \item ($\Leftarrow$) Assume there exists $\psi: W\rightarrow V$ such that
    $\phi \psi = \id_W$.  Fix $w\in W$. We want to show there is a $v\in V$ such
    that $\phi(v) = w$.  Indeed, take $v = \psi(w)$. Then, 
    $\phi(v) = \phi(\psi(w)) = \id_W(w) = w$.

($\Rightarrow$) Assume $\phi$ is surjective.  Let $D$ be a basis for
    $W$.  Define $\tilde{\psi}: D\rightarrow V$ by $\tilde{\psi}(d) = v \in
    \phi^{-1}(d)$.  That is, for each $d \in D$, we simply pick our favorite
    element $v\in \phi^{-1}(d)$.\footnote{We
    must appeal to the Axiom of Choice here, since $\{\phi^{-1}(d) : d \in D\}$
    may be an infinite collection of inifinite sets.} Note that 
    $\phi\tilde{\psi}(d) = d$.  Extend $\tilde{\psi}$ to all of $W$ in the usual
    way: each $w\in W$ has a basis decomposition $w = \sum_D w(d) d$ and we
    define $\psi(w) = \sum_D w(d) \tilde{\psi}(d)$.  Then,
\[
\phi\psi(w) = \phi \sum_{d\in D} w(d) \tilde{\psi}(d)
=\sum_{d\in D} w(d)  \phi \tilde{\psi}(d)
=\sum_{d\in D} w(d)  d = w.
\]
Therefore, $ \phi \psi = \id_W(w)$.

  \end{enumerate}

\end{solution}

\probskip

%%%%%%%%%%%%%%%%%%%%%%%%%%%%%%%%%%%%%%%%%%%%%%%%%%%%%%%%%%%%%%%%%%%%%%%%%%%%
\begin{problem}  % Problem 7
Let $V$ and $W$ be vector spaces over the field $F$.
Recall that $W^V$ denotes the set of maps $\{f : V \rightarrow W\}$.
For fixed $\alpha \in \Hom(V,W)$ and $w \in W$, define the 
\emph{affine transformation} $\zeta_{\alpha, w}: V \rightarrow W$ to be the map 
$v \mapsto \alpha(v) + w$.
Denote by $\Aff (V, W)$ the set of all such affine transformations.  That is,
$\Aff (V, W) := \{\zeta_{\alpha, w} : \alpha \in \Hom(V,W) \text{ and } w \in W\}$.
\begin{enumerate}[(a)]
\item Can you make $W^V$ into an $F$-vector space? An $F$-algebra? Explain.
\item Prove or disprove: $\Hom(V,W) \leq \Aff (V, W) \leq W^{V}$. 
(Interpret $\leq$ here as you see fit.)
\item Note that $\zeta$ (without subscripts) may be viewed as a map from 
$\Hom(V,W)$ to $\Aff(V,W)^W$. Is $\zeta$ a vector space
  homomorphism?  An $F$-algebra homomorphism? Prove.
\end{enumerate}

\end{problem}

\probskip

%%%%%%%%%%%%%%%%%%%%%%%%%%%%%%%%%%%%%%%%%%%%%%%%%%%%%%%%%%%%%%%%%%%%%%%%%%%%
\begin{problem}  % Problem 8
\item Let $V$ be a finite dimensional vector space over the field $F$ and 
suppose $T \in \End(V)$.\\
Prove the following:
  \begin{enumerate}[(a)]
  \item
$\{0_V\} \leq \ker{T} \leq \ker{T^2} \leq \cdots$
\item
$V \geq \im{T} \geq \im{T^2} \geq \cdots$
\item $\dim{V} = \rank{T^k} + \nulity{T^k}$, for each $k= 0, 1, \dots$
\item The sets
    $V_1 := \bigcap\limits_{k=1}^\infty \im{T^k}$ and
    $V_2 := \bigcup\limits_{k=1}^\infty \ker{T^k}$ are $T$-invariant subspaces.
  \item $V = V_1 \oplus V_2$.
  \item If $T_i$ is the restriction of $T$ to $V_i$, then 
    $T_1$ is an isomorphism and $T_2$ is
    \emph{nilpotent}.\footnote{$\alpha\in \End(V)$ is called \emph{nilpotent} if
      there is a positive integer $k$ such that $\alpha^k = \sigma_0$.} 
  \end{enumerate}
\end{problem}
\begin{solution}
  \begin{enumerate}[(a)]
  \item If $T \in \End(V)$, then $T(0_V) = 0_V$, so $0_v\in \ker{T}$, so 
    $\{0_v\}\leq \ker{T}$.  Fix $n \in \N$.  We want to show 
    $\ker{T^n} \leq \ker{T^{n+1}}$. If $v\in \ker{T^n}$, then $T^nv = 0_V$, so 
    $T^{n+1}v = T(T^nv) = T(0_V) = 0_V$, so $v\in \ker{T^{n+1}}$.
\item  Fix $n \in \N$.  We must show $\im{T^n} \geq \im{T^{n+1}}$. 
If $w \in \im{T^{n+1}}$, then $w = T^{n+1}v$ for some $v\in V$. Let $v' =  Tv$.
Then $w =  T^n (Tv) = T^nv'$, so  $w \in \im{T^n}$.

\item 
There are two ways we could proceed.
First, note that for each $k= 0, 1, \dots$ we have $T^k\in \End(V)$, so 
$\dim{V} = \rank{T^k} + \nulity{T^k}$ follows from Proposition 6.10.

Alternatively, we can verify it directly (essentially reproving
Proposition 6.10), as follows: Fix an arbitrary $k \in \{0, 1, \dots\}$, and let
$N = \ker{T^k}$  Let $W$ be a complement of $N$ in $V$.  Then, by Grassman's
Theorem,
\[
\dim{V} = \dim{N} + \dim{W}-\dim{K\cap W} = \nulity{T^k} + \dim{W} - 0.
\]
It remains to show $\dim{W} = \rank{T^k}$, which would follow from
$W \cong \im{T^k}$, so we prove the latter.  Consider $T^k|_W$, the restriction
of $T^k$ to $W$. This is a homomorphism from $W$ into $\im{T^k}$.  We will show it is an
isomorphism by proving that it is one-to-one and onto.  It's one-to-one because 
$T^k|_W(v) = 0_V$ implies $v \in W \cap \ker{T^k} =  W \cap N = \{0_V\}$;
that is, $T^k|_W(v) = 0_V$ implies $v = 0_V$. To see that $T^k|_W$ is onto, fix
$u \in \im{T^k}$. Then there exists $v\in V$ with $T^k(v) = u$.  Let $v = z+y$,
where $z \in N$ and $y\in W$. (Recall, $N$ and $W$ are complementary
subspaces). Then $u = T^k(v) = T^k(z+y) =  T^k(z)+ T^k(y) = T^k(y)$, since $z
\in N = \ker{T^k}$. Finally, since  $y\in W$, we have $u = T^k(y) = T^k|_W(y)$.
This proves that $T^k|_W$ is a bijective homomorphism from $W$ onto $\im{T^k}$,
as desired.
\item (it's easy)
\item In light of parts (a) and (b), and the finite dimensionality of $V$, there
  exists $n\in \N$ such that $\ker{T^{n+k}} = \ker{T^{n}}$ for all $k =0, 1, \dots$.
Similarly, there exists $m\in \N$ such that 
$\im{T^{m+k}} = \im{T^{m}}$ for all $k =0, 1, \dots$. Let $N = \max\{n, m\}$. Then,
\[
V_1 := \bigcap\limits_{k=1}^\infty \im{T^k}= \im{T^N} \quad \text{ and } \quad
V_2 := \bigcup\limits_{k=1}^\infty \ker{T^k}= \ker{T^N}.
\]
So, the goal is to prove 
$V = \im{T^N} \oplus \ker{T^N}$. First we show that 
$\im{T^N} \cap \ker{T^N} = \{0_V\}$. 
Suppose $w \in \im{T^N} \cap \ker{T^N}$.  Then $w = T^Nv$ for some $v\in V$,
and 
\[
0_V = T^Nw = T^N (T^Nv) = T^{2N}v,
\]
so $v \in \ker{T^{2N}} =
\ker{T^{N}}$.
Therefore, $w = T^N(v) = 0_V$. 

It remains to show $V = \im{T^N} +
\ker{T^N}$. This can be seen in at least two ways.
One way is to notice that, by (c), we have 
\begin{equation}
  \label{eq:1}
\dim{V} = \dim{\im{T^N}} +\dim{\ker{T^N}} = \dim{\im{T^N} + \ker{T^N}}.
\end{equation}
If $A$ is a basis for $\im{T^N}$ and $B$ is a basis for $\ker{T^N}$, then since
these subspaces intersect at $0_V$, the set $A\cup B$ is a basis for
$V$. Otherwise there would be a vector $u\notin F(A\cup B)$ so that
$A\cup B \cup\{u\}$ is linearly independent; but then
\[
\dim{V} \geq \dim{F(A\cup B \cup\{u\})} > \dim{F(A\cup B)} =
 \dim{\im{T^N}} +\dim{\ker{T^N}},
\]
which would violate~(\ref{eq:1}).

As an alternative way to prove $V = \im{T^N} + \ker{T^N}$, we fix an arbitrary
$v \in V$ and show that $v = x + y$ for some $x \in \ker{T^N}$ and 
$y\in \im{T^N}$. Indeed, since $\im{T^N} = \im{T^{2N}}$, we have 
$T^N(v) =T^{2N}(w)$, for some $w \in V$.  Therefore, 
$T^N(v - T^N(w)) = 0$, that is,
$v - T^N(w) \in \ker{T^N}$.  Thus,
\[
v = (v - T^N(w)) + T^N(w) =  x + y,
\]
where $x \in \ker{T^N}$ and $y\in \im{T^N}$.
  \end{enumerate}
\end{solution}
\probskip



\end{document}
