\documentclass[11pt]{paper}
\usepackage[
letterpaper,
top    = 3cm,
bottom = 3cm,
left   = 3.00cm,
right  = 3.00cm]{geometry}


\usepackage{tikz-cd}
\usepackage{amsthm}


%%%%%%%%%%%%%%%%%%%%%%%%%%%%%%%%%%%%%%%%
% Basic packages
%%%%%%%%%%%%%%%%%%%%%%%%%%%%%%%%%%%%%%%%
\usepackage{amsmath,amsthm,amssymb}
\usepackage{mathtools}
\usepackage{etoolbox}
\usepackage{fancyhdr}
 \usepackage{xcolor}
\usepackage[colorlinks=true,urlcolor=blue,linkcolor=blue,citecolor=blue]{hyperref}
\usepackage{xspace}
\usepackage{comment}
\usepackage{url} % for url in bib entries
\usepackage{mathrsfs}


\theoremstyle{remark}
\newtheorem{theorem}{Theorem}
\newtheorem*{prop}{Proposition}
\newtheorem{problem}{Problem}
\newtheorem*{prob}{Problem}
\newtheorem*{solution}{{\bf Solution}}
\newtheorem*{hint}{{\it Hint}}
\newtheorem*{ex}{Exercise}


%%%%%%%%%%%%%%%%%%%%%%%%%%%%%%%%%%%%%%%%%%%%%%%%%%
%% Surround the problem and solution with 
%% \begin{ProbBox}  and   \end{ProbBox}
%% to prevent pagebreaks.
\newenvironment{ProbBox}{\noindent\begin{minipage}{\linewidth}}{\end{minipage}}

%%%%%%%%%%%%%%%%
% Acronyms     %
%%%%%%%%%%%%%%%%
\usepackage[acronym, shortcuts]{glossaries}

%% HERE IS HOW YOU DEFINE ACRONYMS:
\newacronym{FTA}{FTA}{Fundamental Theorem of Algebra}
\newacronym{CRT}{CRT}{Chinese Remainder Theorem}

% Make \ac robust.
\robustify{\ac}

%%%%%%%%%%%%%%%%%%%%%%%%
% Fancy page style     %
%%%%%%%%%%%%%%%%%%%%%%%%
\pagestyle{fancy}
\newcommand{\metadata}[2]{
  \lhead{}
  \chead{}
  \rhead{\bfseries Math 700: Linear Algebra}
  \lfoot{#1}
  \cfoot{#2}
  \rfoot{\thepage}
}
\renewcommand{\headrulewidth}{0.4pt}
\renewcommand{\footrulewidth}{0.4pt}


\newrobustcmd*{\vocab}[1]{\emph{#1}}
\newrobustcmd*{\latin}[1]{\textit{#1}}

%%%%%%%%%%%%%%%%%%%%%%%%%%%%%%%%%%
% Customize list enviroonments   %
%%%%%%%%%%%%%%%%%%%%%%%%%%%%%%%%%%
% package to customize three basic list environments: enumerate, itemize and description.
\usepackage{enumitem}
\setitemize{noitemsep, topsep=0pt, leftmargin=*}
\setenumerate{noitemsep, topsep=0pt, leftmargin=*}
\setdescription{noitemsep, topsep=0pt, leftmargin=*}

%%%%%%%%%%%%%%%%%%%%%%%%%%%%
%% Space between problems  %
%%%%%%%%%%%%%%%%%%%%%%%%%%%%
\newrobustcmd*{\probskip}{\vskip1cm}


%%%%%%%%%%%%%%%%%%%%%%%%%%
%%    Math shortcuts     %
%%%%%%%%%%%%%%%%%%%%%%%%%%
\newcommand\join{\ensuremath{\vee}}
\newcommand\meet{\ensuremath{\wedge}}
\newcommand\R{\fld{R}}
\newcommand\End{\ensuremath{\operatorname{End}}}
\newcommand\Aut{\ensuremath{\operatorname{Aut}}}
\newcommand\Hom{\ensuremath{\operatorname{Hom}}}
\newcommand{\Aff}{\ensuremath{\operatorname{Aff}}}
\newcommand{\ann}[1]{\ensuremath{\operatorname{ann}(#1)}}
\newcommand{\id}{\ensuremath{\operatorname{id}}}
\newcommand{\nulity}[1]{\ensuremath{\operatorname{null}(#1)}}
\renewcommand{\ker}[1]{\ensuremath{\operatorname{ker}(#1)}}
\renewcommand{\dim}[1]{\ensuremath{\operatorname{dim}(#1)}}
\newcommand\im[1]{\ensuremath{\operatorname{im}(#1)}}
\newcommand{\rank}[1]{\ensuremath{\operatorname{rank}(#1)}}
\newcommand{\trace}[1]{\ensuremath{\operatorname{trace}(#1)}}
\renewcommand{\phi}{\ensuremath{\varphi}}


%%test This is the Homework LaTeX template.  Use this file to fill in your solutions. 
%%
%% Notes: 
%%    1. Write your answers inside a \begin{solution}...\end{solution} environment.
%%
%%    2. If you will use references, insert bibtex reference entries in the file
%%       Math700.bib.  (Create that file if it doesn't yet exist.)
%%
%%    3. If you will use acronyms, please define them in the macros.tex file.
%%
%%    4. Please try to check that your file compiles:
%%
%%       Mac OS X users: you might try MacTeX. 
%%       Windows users: you might try proTeXt. 
%%       Linux users: most come with TeX; otherwise do a full install of TeXLive.
%%
%%       There is a Makefile in this directory, so on Linux you could just 
%%       enter `make` to compile all the Homework*.tex files at once.
%%
%%    5. Please don't hesitate to inform the prof if you have trouble, or open
%%       a ``New issue'' or create a new ``Wiki page'' on GitHub.  Otherwise,
%%       send an email to williamdemeo@gmail.com.
%%
%%    6. It will probably be hard to keep everyone's notation consistent.
%%       For the most basic symbols, we should have some conventions and use
%%       LaTeX macros to keep the conventions consistent and easy to remember.
%%       For example, to denote an algebra,
         \newcommand\alg[1]{\ensuremath{\mathbf{#1}}}
         \newcommand{\<}{\ensuremath{\langle}}
         \renewcommand{\>}{\ensuremath{\rangle}}
%%       So, an algebra in LaTeX is typed as $\alg{A} = \<A, F\>$.
%%       Similarly, for a field, let's use:
         \newcommand\fld[1]{\ensuremath{\mathbb{#1}}}
%%       So, a field in LaTeX is typed as $\fld{F}$.
%%       Example: We denote the integers, and the real, complex, and rational 
%%                numbers by \fld{Z}, \fld{R}, \fld{C}, \fld{Q}. 
%%                These come up often enough that it's useful to define
         \newcommand\Z{\fld{Z}}
         \newcommand\R{\fld{R}}
         \newcommand\C{\fld{C}}
         \newcommand\Q{\fld{Q}}
         % The natural numbers do not comprise a field, but we still use:
         \newcommand\N{\fld{N}}
%%
%%       For the Galois field, we use:
         \newcommand\GF{\ensuremath{\operatorname{GF}}}
%%
%%    7. Replace these names with yours:
         \metadata{Matt and John}{Homework 3 -- 2014/03/05}
         \author{Matt Swartz and John Tippie}
%%
%%    8. Update the title and date as appropriate.
         \title{Homework 3}
         \date{due date: 2014/03/05}

\begin{document}

\maketitle

%%%%%%%%%%%%%%%%%%%%%%%%%%%%%%%%%%%%%%%%%%%%%%%%%%%%%%%%%%%%%%%%%%%%%%%%%%%%
\begin{ProbBox} % Prevents breaking problem across pages (remove if you want)
\begin{problem}[Golan 124]
Let $F$ be a field and let $(K, \bullet)$ be an associative unital
$F$-algebra.  If $A$ and $B$ are subsets of $K$, we let $A\bullet B$ be the set
of all elements of $K$ of the form $a \bullet b$, with $a \in A$ and $b \in B$
(in particular, $\emptyset \bullet B = A \bullet \emptyset = \emptyset$).
We know that the set $V$ of all subsets of $K$ is a vector space over $\GF(2)$.
Is $(V, \bullet)$ a $\GF(2)$-algebra?  If so, is it associative?  Is it unital?
\end{problem}
\smallskip
\begin{solution}
(type your solution here)
\end{solution}
\end{ProbBox}
\probskip

%%%%%%%%%%%%%%%%%%%%%%%%%%%%%%%%%%%%%%%%%%%%%%%%%%%%%%%%%%%%%%%%%%%%%%%%%%%%
\begin{ProbBox} % Prevents breaking problem across pages (remove if you want)
\begin{problem}[Golan 132]
Let $F$ be a field and let $L$ be the set of all polynomials $f(X) \in F[X]$
satisfying the condition that $f(-a) = -f(a)$ for all $a\in F$.  Is $L$ a
subspace of $F[X]$?
\end{problem}
\smallskip
\begin{solution}
$L$ is the set of odd polynomials (i.e. only odd powers of $x$). To show that $L$ is a subspace of $F[X]$, we need to show that $L$ is a vector space in its own right with respect to the addition and scalar multiplication defined on $F$. Let $f,g \in L$, then $f(X)+g(X)=\sum\limits_{i=0}^\infty a_{2i+1}X^{2i+1} + \sum\limits_{i=0}^\infty b_{2i+1}X^{2i+1} = \sum\limits_{i=0}^\infty (a_{2i+1} + b_{2i+1})X^{2i+1} = \sum\limits_{i=0}^\infty c_{2i+1}X^{2i+1} \in L$. Let $c \in F$, then $cf(X) = c\sum\limits_{i=0}^\infty a_{2i+1}X^{2i+1} = \sum\limits_{i=0}^\infty ca_{2i+1}X^{2i+1} = \sum\limits_{i=0}^\infty b_{2i+1}X^{2i+1} \in L$. Thus, $L$ is a subspace of $F[X]$.
\end{solution}
\end{ProbBox}
\probskip

%%%%%%%%%%%%%%%%%%%%%%%%%%%%%%%%%%%%%%%%%%%%%%%%%%%%%%%%%%%%%%%%%%%%%%%%%%%%
\begin{ProbBox} % Prevents breaking problem across pages (remove if you want)
\begin{problem}[Golan 133]
Let $F$ be a field and let $L$ be the set of all polynomials $f(X) \in F[X]$
satisfying the condition that $\deg(f)$ is even.  Is $L$ a subspace of $F[X]$?
\end{problem}
\smallskip
\begin{solution}
$L$ is the set of polynomials with the highest power of even order. To show that $L$ is a subspace of $F[X]$, we need to show that $L$ is a vector space in its own right with respect to the addition and scalar multiplication defined on $F$. Let $f,g \in L$, then $f(X)+g(X)=\sum\limits_{i=0}^\infty a_{2i}X^{2i} + \sum\limits_{i=0}^\infty b_{2i}X^{2i} = \sum\limits_{i=0}^\infty (a_{2i} + b_{2i})X^{2i} = \sum\limits_{i=0}^\infty c_{2i}X^{2i} \in L$. Let $c \in F$, then $cf(X) = c\sum\limits_{i=0}^\infty a_{2i}X^{2i} = \sum\limits_{i=0}^\infty ca_{2i}X^{2i} = \sum\limits_{i=0}^\infty b_{2i}X^{2i} \in L$. Thus, $L$ is a subspace of $F[X]$.
\end{solution}
\end{ProbBox}
\probskip

%%%%%%%%%%%%%%%%%%%%%%%%%%%%%%%%%%%%%%%%%%%%%%%%%%%%%%%%%%%%%%%%%%%%%%%%%%%%
\begin{ProbBox} % Prevents breaking problem across pages (remove if you want)
\begin{problem}[Golan 142]
For a field $F$, compare the subsets $F[X^2]$ and $F[X^2+1]$ of $F[X]$.
\end{problem}
\smallskip
\begin{solution}
$f(X) \in F[X]$ is defined as $\sum\limits_{i=0}^\infty a_{i}X^{i}$, $f(X) \in F[X^2]$ is defined as $\sum\limits_{i=0}^\infty a_{2i}X^{2i}$, and $f(X) \in F[X^2+1]$ is defined as $\sum\limits_{i=0}^\infty a_{2i}(X^{2}+1)^i$. So, $F[X^2]=\{1,X^2,X^4,X^6,...\}$ and $F[X^2+1]=\{1,X^2+1,(X^2+1)^2,(X^2+1)^3,...\} = \{1,X^2+1,X^4+2X^2+1,X^6+3X^4+3X^2+1,...\}$. $F[X^2] \subseteq F[X^2+1]$ since every element of $F[X^2+1]$ is of the form $\sum\limits_{i=0}^\infty a_{2i}X^{2i}$ (e.g. $X^4+2X^2+1 = n_3+2n_2+n_1$ for $n_i \in F[X^2]$). Similarly, $F[X^2+1] \subseteq F[X^2]$ (e.g. $X^6=m_4-3m_3+3m_2-m_1$ for $m_i \in F[X^2+1]$). Thus, $F[X^2]=F[X^2+1]$.
\end{solution}
\end{ProbBox}
\probskip

%%%%%%%%%%%%%%%%%%%%%%%%%%%%%%%%%%%%%%%%%%%%%%%%%%%%%%%%%%%%%%%%%%%%%%%%%%%%
\begin{ProbBox} % Prevents breaking problem across pages (remove if you want)
\begin{problem}[Golan 154]
Let $F$ be a field and let $K = F^\N$.  Define operations $+$ and $\bullet$ on
$K$ by setting $f+g : i \mapsto f(i) + g(i)$ and 
$f \bullet g: i \mapsto \sum_{j+k=i}f(j) g(k)$.  
Show that $K$ is a an associative and commutative unital $F$-algebra.  Is it
entire? 
\end{problem}
\smallskip
\begin{solution}
To show that $K$ is an associative commutative unital F-algebra, we want to show the following conditions hold: let $u,v,w \in K, a \in F$\\
\begin{enumerate}
  \item $u \bullet (v+w) = u \bullet x = \sum\limits_{j+k=i} u(j)x(k) = \sum\limits_{j+k=i} u(j)[v(k)+w(k)] = \sum\limits_{j+k=i} u(j)v(k) + \sum\limits_{j+k=i} u(j)w(k) = u \bullet v + u \bullet w$,
  \item $(u+v) \bullet w = x \bullet w = \sum\limits_{j+k=i} x(j)w(k) = \sum\limits_{j+k=i} [u(j)+v(j)]w(k) = \sum\limits_{j+k=i} u(j)w(k) + \sum\limits_{j+k=i} v(j)w(k) = u \bullet w + v \bullet w$,
  \item $a(v \bullet w) = a\sum\limits_{j+k=i} v(j)w(k) = \sum\limits_{j+k=i} [av(j)]w(k) = (av) \bullet w = \sum\limits_{j+k=i} v(j)[aw(k)] = v \bullet (aw)$,
  \item $v \bullet (w \bullet y) = $
  \item $v \bullet e = $
  \item $v \bullet w = \sum\limits_{j+k=i} v(j)w(k) = \sum\limits_{k+j=i} w(k)v(j) = w \bullet v$,
\end{enumerate}
 If $v,w \neq 0$, then $v \bullet w = \sum\limits_{j+k=i} v(j)w(k)$ could be equal to $0$ if the vectors $v$ and $w$ are orthogonal. Therefore, $K$ is not entire.
\end{solution}
\end{ProbBox}
\probskip

%%%%%%%%%%%%%%%%%%%%%%%%%%%%%%%%%%%%%%%%%%%%%%%%%%%%%%%%%%%%%%%%%%%%%%%%%%%%
\begin{ProbBox} % Prevents breaking problem across pages (remove if you want)
\begin{problem}[Golan 157]
A \emph{trigonometric polynomial} in $\R^\R$ is a function of the form 
$t \mapsto a_0 + \sum_{h=1}^k [a_h \cos(ht) + b_h \sin(ht)]$, where
$a_0, \dots, a_k, b_1, \dots, b_k \in \R$.  Show that the subset, $K$, of $\R^\R$
consisting of all trigonometric polynomials is an entire $\R$-algebra.
\end{problem}
\smallskip
\begin{solution}
Let $u,v,w \in K$ and $a \in \R^\R$. To show that $K$ is an entire $\R$-algebra, we want to show the following:
\begin{enumerate}
  \item $u \bullet (v+w) = u \bullet x = \sum\limits_{j+k=i} u(j)x(k) = \sum\limits_{j+k=i} u(j)[v(k)+w(k)] = \sum\limits_{j+k=i} u(j)v(k) + \sum\limits_{j+k=i} u(j)w(k) = u \bullet v + u \bullet w$,
  \item $(u+v) \bullet w = x \bullet w = \sum\limits_{j+k=i} x(j)w(k) = \sum\limits_{j+k=i} [u(j)+v(j)]w(k) = \sum\limits_{j+k=i} u(j)w(k) + \sum\limits_{j+k=i} v(j)w(k) = u \bullet w + v \bullet w$,
  \item $a(v \bullet w) = a\sum\limits_{j+k=i} v(j)w(k) = \sum\limits_{j+k=i} [av(j)]w(k) = (av) \bullet w = \sum\limits_{j+k=i} v(j)[aw(k)] = v \bullet (aw)$.
\end{enumerate}
Let $f_i = a_0 + \sum_{h=1}^k [a_h \cos(ht) + b_h \sin(ht)] = a_0+b_0$ and $g_i = c_0 + \sum_{h=1}^k [c_h \cos(ht) + d_h \sin(ht)] = a_0+d_0$. If $f_i,g_i \neq 0$, then $f_i \bullet g_i = (a_0+b_0)(c_0+d_0) = a_0c_0 + a_0d_0 + b_0c_0 + b_0d_0$. Here, we have 4 cases:
\begin{enumerate}
  \item $a_0=0, b_0 \neq 0 c_0=0, d_0 \neq 0$
  \item $a_0=0, b_0 \neq 0 c_0 \neq 0, d_0=0$
  \item $a_0 \neq 0, b_0=0 c_0=0, d_0 \neq 0$
  \item $a_0 \neq 0, b_0=0 c_0 \neq 0, d_0=0$
\end{enumerate}
\end{solution}
\end{ProbBox}
\probskip

%%%%%%%%%%%%%%%%%%%%%%%%%%%%%%%%%%%%%%%%%%%%%%%%%%%%%%%%%%%%%%%%%%%%%%%%%%%%
%% I decided to remove Problem 160 from the assignment.  
%% Of course, you are welcomed to do it if you want. 
%% \begin{ProbBox} % Prevents breaking problem across pages (remove if you want)
%% \begin{problem}[Golan 160]
%% Let $F$ be a field and let $V$ be the subspace of $F[X]$ consisting of all
%% polynomials of degree at most $4$.  Let $p_1(X), \dots, p_5(X)$ be distinct
%% polynomials in $V$ satisfying the condition that $p_i(0) = 1$ for each 
%% $1\leq i \leq 5$. Is the set $\{p_1(X), \dots, p_5(X)\}$ necessarily linearly
%% independent? 
%% \end{problem}
%% \smallskip
%% \begin{solution}
%% (type your solution here)
%% \end{solution}
%% \end{ProbBox}
%% \probskip

%%%%%%%%%%%%%%%%%%%%%%%%%%%%%%%%%%%%%%%%%%%%%%%%%%%%%%%%%%%%%%%%%%%%%%%%%%%%
\begin{ProbBox} % Prevents breaking problem across pages (remove if you want)
\begin{problem}[Golan 163]
Let $F = \Q$.  Is the subset
\[
\left\{ 
\begin{bmatrix}
  4\\[0.3em] 2\\[0.3em] 1
\end{bmatrix},
\begin{bmatrix}
  1\\[0.3em] 0\\[0.3em] 0
\end{bmatrix},
\begin{bmatrix}
  1\\[0.3em] 3\\[0.3em] 4
\end{bmatrix}
\right\}
\]
of $F^3$ linearly independent?  What happens if $F = \GF(5)$?
\end{problem}
\smallskip
\begin{solution}
To see if the subset 
\{ 
$\begin{bmatrix}
  4\\[0.3em] 2\\[0.3em] 1
\end{bmatrix},
\begin{bmatrix}
  1\\[0.3em] 0\\[0.3em] 0
\end{bmatrix},
\begin{bmatrix}
  1\\[0.3em] 3\\[0.3em] 4
\end{bmatrix}$
\}
of $F^3$ is linearly independent, we can perform row operations to reach a RREF form of the matrix

\[ \left[ \begin{array}{ccc}
4 & 1 & 1 \\
2 & 0 & 3 \\
1 & 0 & 4 
\end{array} \right]
\].

We proceed as follows:

\[ \left[ \begin{array}{ccc}
4 & 1 & 1 \\
2 & 0 & 3 \\
1 & 0 & 4 
\end{array} \right]
%
\rightarrow
%
\left[ \begin{array}{ccc}
1 & 0 & 4 \\
4 & 1 & 1 \\
2 & 0 & 3 
\end{array} \right]
%
\rightarrow
%
\left[ \begin{array}{ccc}
1 & 0 & 4 \\
0 & 1 & -15 \\
0 & 0 & 5
\end{array} \right]
%
\rightarrow
%
\left[ \begin{array}{ccc}
1 & 0 & 4 \\
0 & 1 & -15 \\
0 & 0 & 1
\end{array} \right]
%
\rightarrow
%
\left[ \begin{array}{ccc}
1 & 0 & 4 \\
0 & 1 & 0 \\
0 & 0 & 1
\end{array} \right]
%
\rightarrow
%
\left[ \begin{array}{ccc}
1 & 0 & 0 \\
0 & 1 & 0 \\
0 & 0 & 1
\end{array} \right]
\]

Thus, the subset is linearly independent.

\end{solution}
\end{ProbBox}
\probskip

%%%%%%%%%%%%%%%%%%%%%%%%%%%%%%%%%%%%%%%%%%%%%%%%%%%%%%%%%%%%%%%%%%%%%%%%%%%%
%% I decided to remove Problem 160 from the assignment.  
%% Of course, you are welcomed to do it if you want. 
%% \begin{ProbBox} % Prevents breaking problem across pages (remove if you want)
%% \begin{problem}[Golan 171*]
%% Let $t \leq n$ be positive integers and, for all $1\leq i \leq t$, let 
%% $v_i = 
%% \begin{bmatrix}
%%   a_{i1}\\ \vdots \\a_{in}
%% \end{bmatrix}
%% $
%% be a vector in $\R$ chosen so that $2|a_{jj}| > \sum_{i=1}^t|a_{ij}|$ for all 
%% $1\leq j \leq n$.  Show that $\{v_1, \dots, v_t\}$ is linearly independent.

%% \end{problem}
%% \smallskip
%% \begin{solution}
%% (type your solution here)
%% \end{solution}
%% \end{ProbBox}
%% \probskip

%%%%%%%%%%%%%%%%%%%%%%%%%%%%%%%%%%%%%%%%%%%%%%%%%%%%%%%%%%%%%%%%%%%%%%%%%%%%
\begin{ProbBox} % Prevents breaking problem across pages (remove if you want)
\begin{problem}[Golan 177]
Show that the subset
\[
\left\{ 
\begin{bmatrix}
  1\\[0.3em] 2\\[0.3em] 0
\end{bmatrix},
\begin{bmatrix}
  0\\[0.3em] 1\\[0.3em] 2
\end{bmatrix},
\begin{bmatrix}
  2\\[0.3em] 0\\[0.3em] 1
\end{bmatrix}
\right\}
\]
is a linearly independent subset of $\GF(p)^3$ if and only if $p\neq 3$.
\end{problem}
\smallskip
\begin{solution}
Let $p=3$. Perform row reductions as follows:

\[ \left[ \begin{array}{ccc}
1 & 0 & 2 \\
2 & 1 & 0 \\
0 & 2 & 1
\end{array} \right]
%
\rightarrow
%
\left[ \begin{array}{ccc}
1 & 0 & 2 \\
0 & 1 & 2 \\
0 & 2 & 1
\end{array} \right]
%
\rightarrow
%
\left[ \begin{array}{ccc}
1 & 0 & 2 \\
0 & 1 & 2 \\
0 & 0 & 0
\end{array} \right]
\]

Thus, The set is linearly dependent if $p=3$. \\ \\

Let $p \neq 3$. Then if we set 

$a\begin{bmatrix}
  1\\[0.3em] 2\\[0.3em] 0
\end{bmatrix}
+
b\begin{bmatrix}
  0\\[0.3em] 1\\[0.3em] 2
\end{bmatrix}
=
c\begin{bmatrix}
  2\\[0.3em] 0\\[0.3em] 1
\end{bmatrix}$, we get the system of equations $a=2c, 2a+b=0, 2b=c$. Substituting, we then get $2(2c)+b=0 \Rightarrow 4c+b=0 \Rightarrow 4(2b)+b=0 \Rightarrow 9b=0$, which only happens if $p=3$. Thus, if $p \neq 3$, then the set is linearly independent. 

\end{solution}
\end{ProbBox}
\probskip

%%%%%%%%%%%%%%%%%%%%%%%%%%%%%%%%%%%%%%%%%%%%%%%%%%%%%%%%%%%%%%%%%%%%%%%%%%%%
\begin{ProbBox} % Prevents breaking problem across pages (remove if you want)
\noindent {\bf Clarification/amplification of Props 5.12 and 5.13.} 
Let $V$ be a vector space, and let
$\{W_\omega : \omega \in \Omega\}$ be a collection of subspaces of $V$.
Recall that $\sum_{\omega \in \Omega}W_\omega$ denotes the subspace of all
(finite) linear combinations of vectors in $\{W_\omega : \omega \in \Omega\}$,
which is equivalent to the subspace of vectors $w$ of the form
$w = \sum_{\lambda \in \Lambda}w_\lambda$, where $w_\lambda \in W_\lambda$ and
$\Lambda$ is a finite subset of $\Omega$.

We call the set $\{W_\omega : \omega \in \Omega\}$ \emph{independent} if and
only if it satisfies the following condition:
If $\Lambda$ is a finite subset of $\Omega$, if $w_\lambda \in W_\lambda$ for
each $\lambda \in \Lambda$, and if $\sum_{\lambda\in \Lambda} w_\lambda = 0_V$,
then $w_\lambda = 0_V$ for all $\lambda \in \Lambda$.


\begin{problem}
Let 
\[W_1 = \left\{\begin{bmatrix}  a \\ 0\end{bmatrix} : a\in \R \right\},  \;
W_2 = \left\{\begin{bmatrix}  b \\ b\end{bmatrix} : b\in \R\right\}, \; 
W_3 = \left\{\begin{bmatrix}  0 \\ c\end{bmatrix} : a\in \R\right\}.
\]  

\begin{enumerate}
\item Would our textbook author describe $\{W_1, W_2, W_3\}$ as ``pairwise disjoint?'' (explain)
%% That is, do we have $W_i \cap W_j = \left\{\begin{bmatrix}  0 \\ 0\end{bmatrix} \right\}$ for each pair $i\neq j$ in $\{1,2,3\}$.  
\item 
Describe the space $W_1 + W_2 + W_3$.
\item Describe the space $W_i + W_j$ for each pair $i\neq j$ in $\{1,2,3\}$.  
\item Is the set $\{W_1, W_2, W_3\}$ independent?
\end{enumerate}
\end{problem}
\end{ProbBox}
\smallskip
\begin{solution}
(type your solution here)
\end{solution}

\probskip

\begin{problem}
Below is an alleged theorem that attempts to combine Propositions 5.12 and 5.13
of the textbook.  Prove it, or disprove it by providing a counterexample.  If
you find a counterexample, what additional hypothesis would fix the theorem?
%%%%%%%%%%%%%%%%%%%%%%%%%%%%%%%%%%%%%%%%%%%%%%%%%%%%%%%%%%%%%%%%%%%%%%%%%%%%
\begin{prop}
Suppose $\{W_\omega : \omega \in \Omega\}$ is a collection of subspaces of a vector
space, and suppose, for each $\omega \in \Omega$, the set $B_\omega$ is a
basis for $W_\omega$.  Then the following are equivalent: \\[-0.5em]
\begin{enumerate}
\item The set $\{W_\omega : \omega \in \Omega\}$ is independent.\\[-0.5em]
\item Every $w \in \sum_{\omega \in \Omega}W_\omega$ can be written as 
$w  = \sum_{\lambda \in \Lambda}w_\lambda$ in exactly one way.\\[-0.5em]
% (Here $\Lambda$ is a finite subset of $\Omega$.)
\item For every $\lambda \in \Omega$, 
$W_\lambda \cap \sum_{\omega \neq\lambda}W_\omega = \{0_V\}$.\\[-0.5em]
\item The set $B = \bigcup_{\omega \in \Omega} B_\omega$ is a basis for 
$\sum_{\omega \in \Omega}W_\omega$.
\end{enumerate}
\end{prop}
\smallskip

(type your proof or counterexample here)
\end{problem}

\probskip

\noindent {\it Final remark.} We use 
% $\bigoplus \{W_\omega : \omega \in \Omega\}$ or
$\bigoplus_{\omega \in \Omega} W_\omega$ to denote 
$\sum_{\omega \in \Omega} W_\omega$ \emph{only when} the set
$\{W_\omega : \omega \in \Omega\}$ is independent.  
When I mentioned this in class, instead of \emph{only when}, I used
the phrase \emph{when and only when}. This is incorrect since the expression
$\sum_{\omega \in \Omega} W_\omega$ does not necessarily imply that the set
$\{W_\omega : \omega \in \Omega\}$ is dependent. 


%% If you will use references, add your refs to the Math700.bib file.
%% and then uncomment the following lines.
%% \bibliographystyle{plain}
%% \bibliography{Math700}

\end{document}
