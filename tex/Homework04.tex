\documentclass[11pt]{paper}
\usepackage[
letterpaper,
top    = 3cm,
bottom = 3cm,
left   = 3.00cm,
right  = 3.00cm]{geometry}


\usepackage{tikz-cd}
\usepackage{amsthm}


%%%%%%%%%%%%%%%%%%%%%%%%%%%%%%%%%%%%%%%%
% Basic packages
%%%%%%%%%%%%%%%%%%%%%%%%%%%%%%%%%%%%%%%%
\usepackage{amsmath,amsthm,amssymb}
\usepackage{mathtools}
\usepackage{etoolbox}
\usepackage{fancyhdr}
 \usepackage{xcolor}
\usepackage[colorlinks=true,urlcolor=blue,linkcolor=blue,citecolor=blue]{hyperref}
\usepackage{xspace}
\usepackage{comment}
\usepackage{url} % for url in bib entries
\usepackage{mathrsfs}


\theoremstyle{remark}
\newtheorem{theorem}{Theorem}
\newtheorem*{prop}{Proposition}
\newtheorem{problem}{Problem}
\newtheorem*{prob}{Problem}
\newtheorem*{solution}{{\bf Solution}}
\newtheorem*{hint}{{\it Hint}}
\newtheorem*{ex}{Exercise}


%%%%%%%%%%%%%%%%%%%%%%%%%%%%%%%%%%%%%%%%%%%%%%%%%%
%% Surround the problem and solution with 
%% \begin{ProbBox}  and   \end{ProbBox}
%% to prevent pagebreaks.
\newenvironment{ProbBox}{\noindent\begin{minipage}{\linewidth}}{\end{minipage}}

%%%%%%%%%%%%%%%%
% Acronyms     %
%%%%%%%%%%%%%%%%
\usepackage[acronym, shortcuts]{glossaries}

%% HERE IS HOW YOU DEFINE ACRONYMS:
\newacronym{FTA}{FTA}{Fundamental Theorem of Algebra}
\newacronym{CRT}{CRT}{Chinese Remainder Theorem}

% Make \ac robust.
\robustify{\ac}

%%%%%%%%%%%%%%%%%%%%%%%%
% Fancy page style     %
%%%%%%%%%%%%%%%%%%%%%%%%
\pagestyle{fancy}
\newcommand{\metadata}[2]{
  \lhead{}
  \chead{}
  \rhead{\bfseries Math 700: Linear Algebra}
  \lfoot{#1}
  \cfoot{#2}
  \rfoot{\thepage}
}
\renewcommand{\headrulewidth}{0.4pt}
\renewcommand{\footrulewidth}{0.4pt}


\newrobustcmd*{\vocab}[1]{\emph{#1}}
\newrobustcmd*{\latin}[1]{\textit{#1}}

%%%%%%%%%%%%%%%%%%%%%%%%%%%%%%%%%%
% Customize list enviroonments   %
%%%%%%%%%%%%%%%%%%%%%%%%%%%%%%%%%%
% package to customize three basic list environments: enumerate, itemize and description.
\usepackage{enumitem}
\setitemize{noitemsep, topsep=0pt, leftmargin=*}
\setenumerate{noitemsep, topsep=0pt, leftmargin=*}
\setdescription{noitemsep, topsep=0pt, leftmargin=*}

%%%%%%%%%%%%%%%%%%%%%%%%%%%%
%% Space between problems  %
%%%%%%%%%%%%%%%%%%%%%%%%%%%%
\newrobustcmd*{\probskip}{\vskip1cm}


%%%%%%%%%%%%%%%%%%%%%%%%%%
%%    Math shortcuts     %
%%%%%%%%%%%%%%%%%%%%%%%%%%
\newcommand\join{\ensuremath{\vee}}
\newcommand\meet{\ensuremath{\wedge}}
\newcommand\R{\fld{R}}
\newcommand\End{\ensuremath{\operatorname{End}}}
\newcommand\Aut{\ensuremath{\operatorname{Aut}}}
\newcommand\Hom{\ensuremath{\operatorname{Hom}}}
\newcommand{\Aff}{\ensuremath{\operatorname{Aff}}}
\newcommand{\ann}[1]{\ensuremath{\operatorname{ann}(#1)}}
\newcommand{\id}{\ensuremath{\operatorname{id}}}
\newcommand{\nulity}[1]{\ensuremath{\operatorname{null}(#1)}}
\renewcommand{\ker}[1]{\ensuremath{\operatorname{ker}(#1)}}
\renewcommand{\dim}[1]{\ensuremath{\operatorname{dim}(#1)}}
\newcommand\im[1]{\ensuremath{\operatorname{im}(#1)}}
\newcommand{\rank}[1]{\ensuremath{\operatorname{rank}(#1)}}
\newcommand{\trace}[1]{\ensuremath{\operatorname{trace}(#1)}}
\renewcommand{\phi}{\ensuremath{\varphi}}


%%test This is the Homework LaTeX template.  Use this file to fill in your solutions. 
%%
%% Notes: 
%%    1. Write your answers inside a \begin{solution}...\end{solution} environment.
%%
%%    2. If you will use references, insert bibtex reference entries in the file
%%       Math700.bib.  (Create that file if it doesn't yet exist.)
%%
%%    3. If you will use acronyms, please define them in the macros.tex file.
%%
%%    4. Please try to check that your file compiles:
%%
%%       Mac OS X users: you might try MacTeX. 
%%       Windows users: you might try proTeXt. 
%%       Linux users: most come with TeX; otherwise do a full install of TeXLive.
%%
%%       There is a Makefile in this directory, so on Linux you could just 
%%       enter `make` to compile all the Homework*.tex files at once.
%%
%%    5. Please don't hesitate to inform the prof if you have trouble, or open
%%       a ``New issue'' or create a new ``Wiki page'' on GitHub.  Otherwise,
%%       send an email to williamdemeo@gmail.com.
%%
%%    6. It will probably be hard to keep everyone's notation consistent.
%%       For the most basic symbols, we should have some conventions and use
%%       LaTeX macros to keep the conventions consistent and easy to remember.
%%       For example, to denote an algebra,
         \newcommand\alg[1]{\ensuremath{\mathbf{#1}}}
         \newcommand{\<}{\ensuremath{\langle}}
         \renewcommand{\>}{\ensuremath{\rangle}}
%%       So, an algebra in LaTeX is typed as $\alg{A} = \<A, F\>$.
%%       Similarly, for a field, let's use:
         \newcommand\fld[1]{\ensuremath{\mathbb{#1}}}
%%       So, a field in LaTeX is typed as $\fld{F}$.
%%       Example: We denote the integers by \fld{Z}. 
%%                This comes up often enough that it's useful to define
         \newcommand\Z{\fld{Z}}
%%
%%       For the Galois field, we use:
         \newcommand\GF{\ensuremath{\operatorname{GF}}}
%%
%%    7. Replace these names with yours!!!
         \metadata{Michael and Taylor}{Homework 4 -- 2014/04/14}
         \author{Michael Laughlin and Taylor Short}
%%
%%    8. Update the title and date as appropriate.
         \title{Homework 4}
         \date{due date: 2014/04/14}

\begin{document}

\maketitle

\noindent The label ``Problem'' is used for required problems. ``Exercise''
is for suggested exercises.

\medskip

%%%%%%%%%%%%%%%%%%%%%%%%%%%%%%%%%%%%%%%%%%%%%%%%%%%%%%%%%%%%%%%%%%%%%%%%%%%%
\begin{problem}[Golan 199]
Let $V$ be a vector space of finite dimension $n>0$ over $\R$ and, for each
positive integer $i$, let $U_i$ be a proper subspace of $V$.  
Show that $V \neq \bigcup_{i=1}^\infty U_i$.
\end{problem}
% \smallskip
% \begin{solution}
% \end{solution}

\probskip

%%%%%%%%%%%%%%%%%%%%%%%%%%%%%%%%%%%%%%%%%%%%%%%%%%%%%%%%%%%%%%%%%%%%%%%%%%%%
\begin{problem}[Golan 210]
Let $V$ be a vector space over a field $F$ and assume $V$ is not finitely
generated.  Show that there exists an infinite sequence $W_1, W_2, \dots$ of
proper subspaces of $V$ satisfying $\bigcup_{i=1}^\infty W_i = V$.
\end{problem}
% \smallskip
% \begin{solution}
% \end{solution}

\probskip


%%%%%%%%%%%%%%%%%%%%%%%%%%%%%%%%%%%%%%%%%%%%%%%%%%%%%%%%%%%%%%%%%%%%%%%%%%%%
\begin{ex}[Golan 239]
Let $V$ and $W$ be a vector space over $\fld{Q}$ and let 
$\alpha: V \rightarrow W$ be a function satisfying 
$\alpha(x+y) = \alpha(x) + \alpha(y)$ for all $x, y \in V$.
Is $\alpha$ necessarily a linear transformation?
\end{ex}
% \smallskip
% \begin{solution}
% \end{solution}

\probskip

%%%%%%%%%%%%%%%%%%%%%%%%%%%%%%%%%%%%%%%%%%%%%%%%%%%%%%%%%%%%%%%%%%%%%%%%%%%%
\begin{ex}[Golan 240]
Let $\alpha: \R \rightarrow \R$ be a continuous function satisfying
$\alpha (x + y) = \alpha(x) + \alpha(y)$ for all $a, b \in \R$.  Show that
$\alpha$ is a linear transformation.
\end{ex}
% \smallskip
% \begin{solution}
% \end{solution}

\probskip


%%%%%%%%%%%%%%%%%%%%%%%%%%%%%%%%%%%%%%%%%%%%%%%%%%%%%%%%%%%%%%%%%%%%%%%%%%%%
\begin{problem}[Golan 241]
Let $W_1$ and $W_2$ be subspaces of a vector space $V$ over a field $F$ and assume
we have linear transformations 
$\alpha_1: W_1 \rightarrow V$ and 
$\alpha_2: W_2 \rightarrow V$ satisfying the condition that $\alpha_1(v) = \alpha_2(v)$
for all $v \in W_1 \cap W_2$.  Find a linear transformation 
$\theta: W_1 + W_2 \rightarrow V$ such that the restriction of $\theta$ to $W_i$
equals $\alpha_i$ $(i=1, 2)$, or show why no such linear transformation exists.
\end{problem}
% \smallskip
% \begin{solution}
% \end{solution}

\probskip

%%%%%%%%%%%%%%%%%%%%%%%%%%%%%%%%%%%%%%%%%%%%%%%%%%%%%%%%%%%%%%%%%%%%%%%%%%%%
\begin{problem}[Golan 251]
Let $V$, $W$ and $Y$ be vector spaces finitely generated over a field $F$ and let 
$\alpha \in \Hom(V, W)$.  Let $\ann{\alpha}$ denote the set of those 
$\beta \in \Hom(W,Y)$ satisfying the condition that $\beta\alpha$ is the 0-transformation.
That is, 
\[
\ann{\alpha} = \{ \beta\in \Hom(W,Y) \mid \forall v \in V \; \beta\alpha(v) = 0_Y\}.
\]
Prove that $\ann{\alpha}$ is a subspace of  $\Hom(W,Y)$ and compute its
dimension.
\end{problem}
% \smallskip
% \begin{solution}
% \end{solution}

\probskip

%%%%%%%%%%%%%%%%%%%%%%%%%%%%%%%%%%%%%%%%%%%%%%%%%%%%%%%%%%%%%%%%%%%%%%%%%%%%
\begin{ex}[Golan 253]
Let $V$ and $W$ be vector spaces over a field $F$ and assume that there are
subspaces $V_1$ and $V_2$ of $V$, both of positive dimension, satisfying
$V = V_1 \bigoplus V_2$.  For  $i=1, 2$, let 
$U_i = \{\alpha \in \Hom(V,W) \mid V_i \subseteq \ker{\alpha}\}$.
Show that $\{U_1, U_2\}$ is an independent set of subspaces of $\Hom(V,W)$.
Is it necessarily true that $\Hom(V,W) = U_1 \bigoplus U_2$?
\end{ex}
% \smallskip
% \begin{solution}
% \end{solution}

\probskip

%%%%%%%%%%%%%%%%%%%%%%%%%%%%%%%%%%%%%%%%%%%%%%%%%%%%%%%%%%%%%%%%%%%%%%%%%%%%
\begin{problem}[Golan 256]
Let $V$ and $W$ be vector spaces over a field $F$.
Define a function $\phi : \Hom(V, W) \rightarrow \Hom(V\times W, V\times W)$
by setting
$\phi(\alpha): 
\begin{bmatrix} v\\ w \end{bmatrix} 
\mapsto 
\begin{bmatrix} 0_V\\ \alpha(v) \end{bmatrix}$.
Is $\phi$ a linear transformation of vector spaces over $F$? Is it a monomorphism?
\end{problem}
% \smallskip
% \begin{solution}
% \end{solution}

\probskip



%%%%%%%%%%%%%%%%%%%%%%%%%%%%%%%%%%%%%%%%%%%%%%%%%%%%%%%%%%%%%%%%%%%%%%%%%%%%
\begin{problem}[Golan 293 \& 294]
Let $V$, $W$ and $Y$ be vector spaces over a field $F$, and
let $\beta \in \Hom(V,Y)$. Prove the following:
\begin{enumerate}
\item  If $\alpha \in \Hom(W, Y)$ is an epimorphism, then 
there exists $\theta \in \Hom(V,W)$ such that $\beta = \alpha \theta$.

\begin{center}
\begin{tikzcd}
{} & V \arrow[dashed]{dl}[swap]{\exists \theta} \arrow{d}[swap]{\beta} & \\
W \arrow{r}{\alpha} & Y \arrow{r} & 0
\end{tikzcd}
\end{center}

\item  If $\alpha \in \Hom(V, W)$ is a monomorphism, then 
there exists $\theta \in \Hom(W,Y)$ such that $\beta = \theta \alpha$.

\begin{center}
\begin{tikzcd}
0 \arrow{r} & V \arrow{d}[swap]{\beta} \arrow{r}{\alpha} & W \\ 
 & Y\arrow[dashed]{ur}[swap]{\exists \theta} &
\end{tikzcd}
\end{center}

\end{enumerate}

\noindent \emph{Note to students:} Here is an alternative statement of the problem, with
naming conventions that agree with Golan. You may solve whichever version you prefer. 

\begin{enumerate}
\item  
If $\alpha \in \Hom(V,W)$ is an epimorphism, then for every 
$\beta \in \Hom(Y,W)$ there exists $\theta \in \Hom(Y,V)$ such that $\beta = \alpha \theta$.

\begin{center}
\begin{tikzcd}
{} & Y \arrow[dashed]{dl}[swap]{\exists \theta} \arrow{d}[swap]{\beta} & \\
V \arrow{r}{\alpha} & W \arrow{r} & 0
\end{tikzcd}
\end{center}

\item  If $\alpha \in \Hom(V, W)$ is a monomorphism, then 
 for every $\beta \in \Hom(V,Y)$ there exists $\theta \in \Hom(W,Y)$ such that $\beta = \theta \alpha$.

\begin{center}
\begin{tikzcd}
0 \arrow{r} & V \arrow{d}[swap]{\beta} \arrow{r}{\alpha} & W \\ 
 & Y\arrow[dashed]{ur}[swap]{\exists \theta} &
\end{tikzcd}
\end{center}

\end{enumerate}



%% \begin{center}
%% \begin{tikzpicture}[baseline= (a).base]
%% \node[scale=1] (a) at (0,0){
%% \begin{tikzcd}
%% 0 \arrow{r} & V \arrow{d}[swap]{\beta} \arrow{r}{\alpha} & W \\ 
%%  & Y\arrow[dashed]{ur}[swap]{\exists \theta} &
%% \end{tikzcd}
%% };
%% (a) at (0,0){
%% \begin{tikzcd}
%% 0 \arrow{r} & V \arrow{d}[swap]{\beta} \arrow{r}{\alpha} & W \\ 
%%  & Y\arrow[dashed]{ur}[swap]{\exists \theta} &
%% \end{tikzcd}
%% };
%% \end{tikzpicture}
%% \end{center}

\end{problem}
% \smallskip
% \begin{solution}
% \end{solution}


\probskip

%%%%%%%%%%%%%%%%%%%%%%%%%%%%%%%%%%%%%%%%%%%%%%%%%%%%%%%%%%%%%%%%%%%%%%%%%%%%
\begin{problem}[Golan 296]\hskip-2mm\protect\footnotemark
\label{prob:296}
\footnotetext{The claim in this problem seems incorrect to me. 
  If you agree, give a counter-example, then modify the claim so it is correct and prove it.
  If you disagree, and you believe the claim is correct, then prove it as given.}
Let $V$, $W$ be vector spaces over a field $F$, let 
$\alpha \in \Hom(V, W)$, and let $D$ be a nonempty linearly independent subset
of $\im{\alpha}$.  Show that there exists a basis $B$ of $V$ satisfying
$\{\alpha(v)\mid v \in B\} = D$.
\end{problem}
% \smallskip
% \begin{solution}
% \end{solution}

\probskip

%%%%%%%%%%%%%%%%%%%%%%%%%%%%%%%%%%%%%%%%%%%%%%%%%%%%%%%%%%%%%%%%%%%%%%%%%%%%
\begin{problem}[Golan 306]
Let $V$, $W$ and $Y$ be vector spaces over a field $F$.  Let 
$\{\alpha_1, \dots, \alpha_n\}$ be a finite subset of $\Hom(V, W)$
and let $\beta \in \Hom(V,Y)$ be a linear transformation satisfying
$\bigcap_{i=1}^n \ker{\alpha_i} \subseteq \ker{\beta}$.  Show that there exist
linear transformations $\gamma_1, \dots, \gamma_n$ in $\Hom(W,Y)$ satisfying
$\beta = \sum_{i=1}^n \gamma_i \alpha_i$.
\end{problem}
% \smallskip
% \begin{solution}
% \end{solution}

\probskip

%%%%%%%%%%%%%%%%%%%%%%%%%%%%%%%%%%%%%%%%%%%%%%%%%%%%%%%%%%%%%%%%%%%%%%%%%%%%
\begin{problem}[Golan 266]
Let $A$ and $B$ be nonempty sets.  Let $V$ be the collection of all subsets of
$A$ and let $W$ be the collection of all subsets of $B$, both of which are
vector spaces over $\GF(2)$.  Any function $f: A\rightarrow B$ defines a
function $\alpha_f : W \rightarrow V$ by setting 
$\alpha_f: D \mapsto \{a \in A : f(a) \in D\}$.  Show that each such function
$\alpha_f$ defines a linear transformation, and find its kernel.
\end{problem}


%\bibliographystyle{plainurl}
%\bibliography{Math700}

\end{document}
