\documentclass[11pt]{paper}
\usepackage[
letterpaper,
top    = 3cm,
bottom = 3cm,
left   = 3.00cm,
right  = 3.00cm]{geometry}


\usepackage{tikz-cd}
\usepackage{amsthm}


%%%%%%%%%%%%%%%%%%%%%%%%%%%%%%%%%%%%%%%%
% Basic packages
%%%%%%%%%%%%%%%%%%%%%%%%%%%%%%%%%%%%%%%%
\usepackage{amsmath,amsthm,amssymb}
\usepackage{mathtools}
\usepackage{etoolbox}
\usepackage{fancyhdr}
 \usepackage{xcolor}
\usepackage[colorlinks=true,urlcolor=blue,linkcolor=blue,citecolor=blue]{hyperref}
\usepackage{xspace}
\usepackage{comment}
\usepackage{url} % for url in bib entries
\usepackage{mathrsfs}


\theoremstyle{remark}
\newtheorem{theorem}{Theorem}
\newtheorem*{prop}{Proposition}
\newtheorem{problem}{Problem}
\newtheorem*{prob}{Problem}
\newtheorem*{solution}{{\bf Solution}}
\newtheorem*{hint}{{\it Hint}}
\newtheorem*{ex}{Exercise}


%%%%%%%%%%%%%%%%%%%%%%%%%%%%%%%%%%%%%%%%%%%%%%%%%%
%% Surround the problem and solution with 
%% \begin{ProbBox}  and   \end{ProbBox}
%% to prevent pagebreaks.
\newenvironment{ProbBox}{\noindent\begin{minipage}{\linewidth}}{\end{minipage}}

%%%%%%%%%%%%%%%%
% Acronyms     %
%%%%%%%%%%%%%%%%
\usepackage[acronym, shortcuts]{glossaries}

%% HERE IS HOW YOU DEFINE ACRONYMS:
\newacronym{FTA}{FTA}{Fundamental Theorem of Algebra}
\newacronym{CRT}{CRT}{Chinese Remainder Theorem}

% Make \ac robust.
\robustify{\ac}

%%%%%%%%%%%%%%%%%%%%%%%%
% Fancy page style     %
%%%%%%%%%%%%%%%%%%%%%%%%
\pagestyle{fancy}
\newcommand{\metadata}[2]{
  \lhead{}
  \chead{}
  \rhead{\bfseries Math 700: Linear Algebra}
  \lfoot{#1}
  \cfoot{#2}
  \rfoot{\thepage}
}
\renewcommand{\headrulewidth}{0.4pt}
\renewcommand{\footrulewidth}{0.4pt}


\newrobustcmd*{\vocab}[1]{\emph{#1}}
\newrobustcmd*{\latin}[1]{\textit{#1}}

%%%%%%%%%%%%%%%%%%%%%%%%%%%%%%%%%%
% Customize list enviroonments   %
%%%%%%%%%%%%%%%%%%%%%%%%%%%%%%%%%%
% package to customize three basic list environments: enumerate, itemize and description.
\usepackage{enumitem}
\setitemize{noitemsep, topsep=0pt, leftmargin=*}
\setenumerate{noitemsep, topsep=0pt, leftmargin=*}
\setdescription{noitemsep, topsep=0pt, leftmargin=*}

%%%%%%%%%%%%%%%%%%%%%%%%%%%%
%% Space between problems  %
%%%%%%%%%%%%%%%%%%%%%%%%%%%%
\newrobustcmd*{\probskip}{\vskip1cm}


%%%%%%%%%%%%%%%%%%%%%%%%%%
%%    Math shortcuts     %
%%%%%%%%%%%%%%%%%%%%%%%%%%
\newcommand\join{\ensuremath{\vee}}
\newcommand\meet{\ensuremath{\wedge}}
\newcommand\R{\fld{R}}
\newcommand\End{\ensuremath{\operatorname{End}}}
\newcommand\Aut{\ensuremath{\operatorname{Aut}}}
\newcommand\Hom{\ensuremath{\operatorname{Hom}}}
\newcommand{\Aff}{\ensuremath{\operatorname{Aff}}}
\newcommand{\ann}[1]{\ensuremath{\operatorname{ann}(#1)}}
\newcommand{\id}{\ensuremath{\operatorname{id}}}
\newcommand{\nulity}[1]{\ensuremath{\operatorname{null}(#1)}}
\renewcommand{\ker}[1]{\ensuremath{\operatorname{ker}(#1)}}
\renewcommand{\dim}[1]{\ensuremath{\operatorname{dim}(#1)}}
\newcommand\im[1]{\ensuremath{\operatorname{im}(#1)}}
\newcommand{\rank}[1]{\ensuremath{\operatorname{rank}(#1)}}
\newcommand{\trace}[1]{\ensuremath{\operatorname{trace}(#1)}}
\renewcommand{\phi}{\ensuremath{\varphi}}


%%test This is the Homework LaTeX template.  Use this file to fill in your solutions. 
%%
%% Notes: 
%%    1. Write your answers inside a \begin{solution}...\end{solution} environment.
%%
%%    2. If you will use references, insert bibtex reference entries in the file
%%       Math700.bib.  (Create that file if it doesn't yet exist.)
%%
%%    3. If you will use acronyms, please define them in the macros.tex file.
%%
%%    4. Please try to check that your file compiles:
%%
%%       Mac OS X users: you might try MacTeX. 
%%       Windows users: you might try proTeXt. 
%%       Linux users: most come with TeX; otherwise do a full install of TeXLive.
%%
%%       There is a Makefile in this directory, so on Linux you could just 
%%       enter `make` to compile all the Homework*.tex files at once.
%%
%%    5. Please don't hesitate to inform the prof if you have trouble, or open
%%       a ``New issue'' or create a new ``Wiki page'' on GitHub.  Otherwise,
%%       send an email to williamdemeo@gmail.com.
%%
%%    6. It will probably be hard to keep everyone's notation consistent.
%%       For the most basic symbols, we should have some conventions and use
%%       LaTeX macros to keep the conventions consistent and easy to remember.
%%       For example, to denote an algebra,
         \newcommand\alg[1]{\ensuremath{\mathbf{#1}}}
         \newcommand{\<}{\ensuremath{\langle}}
         \renewcommand{\>}{\ensuremath{\rangle}}
%%       So, an algebra in LaTeX is typed as $\alg{A} = \<A, F\>$.
%%       Similarly, for a field, let's use:
         \newcommand\fld[1]{\ensuremath{\mathbb{#1}}}
%%       So, a field in LaTeX is typed as $\fld{F}$.
%%       Example: We denote the integers by \fld{Z}. 
%%                This comes up often enough that it's useful to define
         \newcommand\Z{\fld{Z}}
%%
%%       For the Galois field, we use:
         \newcommand\GF{\ensuremath{\operatorname{GF}}}
%%
%%    7. Replace these names with yours!!!
         \metadata{Michael and Taylor}{Homework 4 -- 2014/04/09}
         \author{Michael Laughlin and Taylor Short}
%%
%%    8. Update the title and date as appropriate.
         \title{Homework 4}
         \date{due date: 2014/04/09}

\begin{document}

\maketitle

\noindent The label ``Problem'' is used for required problems. ``Exercise''
is for suggested exercises.

\medskip

%%%%%%%%%%%%%%%%%%%%%%%%%%%%%%%%%%%%%%%%%%%%%%%%%%%%%%%%%%%%%%%%%%%%%%%%%%%%
\begin{problem}[Golan 199]
Let $V$ be a vector space of finite dimension $n>0$ over $\R$ and, for each
positive integer $i$, let $U_i$ be a proper subspace of $V$.  
Show that $V \neq \bigcup_{i=1}^\infty U_i$.
\end{problem}
 \smallskip
 \begin{solution}
 We proceed by induction on $n$. For the base case $n=1$ the claim is obviously true, $V$ cannot be written as a union of proper subspaces as the only subspaces are $V$ and $\emptyset$. Assume the claim holds for vector spaces with dimension $n \geq 1$ and consider $V$ with $\dim{V} = n+1$. Now for $\bigcup_{i=1}^\infty U_i$ a union of proper subspaces of $V$, take $0 \neq u \in U_1$ and consider the span $\mathbb{R}\{u\}$. Since $\mathbb{R}$ is uncountable, the subspace $\mathbb{R}\{u\}$ is uncountable. Then by Proposition 5.15 in \cite{Golan:2012} there exists uncountably many, distinct complements $\{W_\alpha \}_{\alpha \in A}$ of $\mathbb{R}\{u\}$. Furthermore each $W_\alpha$ has dimension $n$, since by Grassman's Theorem
 $$
n+1 = \dim{V} = \dim{\mathbb{R}\{u\}} + \dim{W_\alpha} - \dim{\mathbb{R}\{u\}\cap W_\alpha} = 1 + \dim{W_\alpha} - 0.
 $$
 Now the collection $\{U_i\}_{i=1}^\infty$ of subspaces is countable and $\{W_\alpha \}_{\alpha \in A}$ is uncountable, so there exists some $W_{\alpha _0} \in \{W_\alpha \}_{\alpha \in A}$ that is distinct from all $U_i$. Moreover since $W_{\alpha _0}$ has dimension $n$, $W_{\alpha _0}$ is distinct from the $U_i$, and being proper the $U_i$ can have dimension at most $n$, then we must have $\dim{W_{\alpha _0} \cap U_i} < n$ for all $i$, that is, $W_{\alpha _0} \cap U_i$ is a proper subspace of $W_{\alpha _0}$. But now by the induction hypothesis, 
 $$
W_{\alpha _0} \neq \bigcup _{i=1}^\infty (W_{\alpha _0} \cap U_i).
 $$
 So there are elements of $W_{\alpha _0}$ a subspace of $V$ that do not belong to $\bigcup_{i=1}^\infty U_i$ and hence we must have $V \neq \bigcup_{i=1}^\infty U_i$, finishing the proof by induction.
 \end{solution}

\probskip

%%%%%%%%%%%%%%%%%%%%%%%%%%%%%%%%%%%%%%%%%%%%%%%%%%%%%%%%%%%%%%%%%%%%%%%%%%%%
\begin{problem}[Golan 210]
Let $V$ be a vector space over a field $F$ and assume $V$ is not finitely
generated.  Show that there exists an infinite sequence $W_1, W_2, \dots$ of
proper subspaces of $V$ satisfying $\bigcup_{i=1}^\infty W_i = V$.
\end{problem}
\smallskip
\begin{solution}
There exists a basis $B = \{b_1,b_2,b_3,\dots\}$ for $V$. Let $W_i = F\{b_1,b_2,\dots,b_i\}$.  Then 
each subspace $W_i$ is properly contained in $V$, and $\bigcup_{i=1}^\infty W_i = V$, since for every 
$v\in V$, $v$ is representable as a finite linear combination of elements in $B$, and if $b_n$ is the maximum 
indexed among these basis elements, $v\in W_n$.
\end{solution}

\probskip


%%%%%%%%%%%%%%%%%%%%%%%%%%%%%%%%%%%%%%%%%%%%%%%%%%%%%%%%%%%%%%%%%%%%%%%%%%%%
\begin{ex}[Golan 239]
Let $V$ and $W$ be a vector space over $\fld{Q}$ and let 
$\alpha: V \rightarrow W$ be a function satisfying 
$\alpha(x+y) = \alpha(x) + \alpha(y)$ for all $x, y \in V$.
Is $\alpha$ necessarily a linear transformation?
\end{ex}
\smallskip
\begin{solution}
Let $c\in\mathbb{Q}$ be arbitrary. We need only verify that $\alpha(cv) = c\alpha(v)$. Suppose $c = \frac{p}{q}$ for $p,q\in\mathbb{Z}$.  By supposition, for any $z\in\mathbb{Z}$, $\alpha(zv) = z\alpha(v)$.  Then we have $q\alpha(cv) = q\alpha(\frac{p}{q}v) = \alpha(q\frac{p}{q}v) = \alpha(pv) = p\alpha(v)$.  Hence, $\alpha(cv) = \frac{p}{q}\alpha(v) = c\alpha(v)$ and $\alpha$ is a linear transformation.
\end{solution}

\probskip

%%%%%%%%%%%%%%%%%%%%%%%%%%%%%%%%%%%%%%%%%%%%%%%%%%%%%%%%%%%%%%%%%%%%%%%%%%%%
\begin{ex}[Golan 240]
Let $\alpha: \R \rightarrow \R$ be a continuous function satisfying
$\alpha (x + y) = \alpha(x) + \alpha(y)$ for all $a, b \in \R$.  Show that
$\alpha$ is a linear transformation.
\end{ex}
\smallskip
\begin{solution}
Let $c\in\mathbb{R}$ be arbitrary. We need only verify that $\alpha(cx) = c\alpha(x)$.  Let $r\in\mathbb{Q}$.  Since $\alpha$ is continuous, we have by the results of the previous exercise $c\alpha(x) = \lim\limits_{r\to c}r\alpha(x) = \lim\limits_{r\to c}\alpha(rx) = \alpha(cx)$.  
\end{solution}

\probskip


%%%%%%%%%%%%%%%%%%%%%%%%%%%%%%%%%%%%%%%%%%%%%%%%%%%%%%%%%%%%%%%%%%%%%%%%%%%%
\begin{problem}[Golan 241]
Let $W_1$ and $W_2$ be subspaces of a vector space $V$ over a field $F$ and assume
we have linear transformations 
$\alpha_1: W_1 \rightarrow V$ and 
$\alpha_2: W_2 \rightarrow V$ satisfying the condition that $\alpha_1(v) = \alpha_2(v)$
for all $v \in W_1 \cap W_2$.  Find a linear transformation 
$\theta: W_1 + W_2 \rightarrow V$ such that the restriction of $\theta$ to $W_i$
equals $\alpha_i$ $(i=1, 2)$, or show why no such linear transformation exists.
\end{problem}
\smallskip
\begin{solution}
Let $\theta(v)$ be defined by $\theta(v) = \theta(w_1 + w_2) = \alpha_1(w_1) + \alpha_2(w_2)$ 
for all $v = w_1 + w_2 \in W_1 + W_2$.  First we check that $\theta$ is well-defined.  Suppose 
that $v = w_1 + w_2 = w_1^{\prime} + w_2^{\prime}$.  Then we have that $w_1^{\prime} - w_1 = 
w_2 - w_2^{\prime}$ and for some $u\in V$, $u + \alpha_1(w_1) + \alpha_2(w_2) = 
\alpha_1(w_1^{\prime}) + \alpha_2(w_2^{\prime})$.  But then

\begin{align*}
u &= \alpha_1(w_1^{\prime}) - \alpha_1(w_1) + \alpha_2(w_2^{\prime}) - \alpha_2(w_2) \\
  &= \alpha_1(w_1^{\prime}-w_1) + \alpha_2(w_2^{\prime}-w_2) \\
  &= \alpha_2(w_2 - w_2^{\prime}) + \alpha_2(w_2^{\prime}-w_2) \\
  &= \alpha_2(0) \\
  &= 0
\end{align*}

Since $\theta$ is well-defined, it suffices to check that $\theta$ is linear and satisfies the restriction property.  Let $w_1,w_1^{\prime}\in W_1$ and $w_2,w_2^{\prime}\in W_2$.  Then we have, for $v_1 = w_1+w_2$, $v_2 = w_1^{\prime}+w_2^{\prime}$, and $c\in F$ 

\begin{align*}
\theta(v_1 + v_2) &= \theta(w_1+w_2+w_1^{\prime}+w_2^{\prime}) \\
                  &= \alpha_1(w_1 + w_1^{\prime}) + \alpha_2(w_2 + w_2^{\prime}) \\
                  &= \alpha_1(w_1) + \alpha_2(w_2) + \alpha_1(w_1^{\prime}) + \alpha_2(w_2^{\prime})\\
                  &= \theta(w_1 + w_2) + \theta(w_1^{\prime} + w_2^{\prime})\\
                  &= \theta(v_1) + \theta(v_2) \\[0.3cm]
\theta(cv_1)      &= \theta(c(w_1+w_2))\\
                  &= \theta(cw_1 + cw_2)\\
                  &= \alpha_1(cw_1) + \alpha_2(cw_2)\\
                  &= c(\alpha_1(w_1) + \alpha_2(w_2))\\
                  &= c\theta(v_1)\\[0.3cm]
\theta(w_1)       &= \theta(w_1 + 0)\\
                  &= \alpha_1(w_1) + \alpha_2(0)\\
                  &= \alpha_1(w_1)\\[0.3cm]
\theta(w_2)       &= \theta(0+w_2)\\
                  &= \alpha_1(0) + \alpha_2(w_2)\\
                  &= \alpha_2(w_2)
\end{align*}
\end{solution}

\probskip

%%%%%%%%%%%%%%%%%%%%%%%%%%%%%%%%%%%%%%%%%%%%%%%%%%%%%%%%%%%%%%%%%%%%%%%%%%%%
\begin{problem}[Golan 251]
Let $V$, $W$ and $Y$ be vector spaces finitely generated over a field $F$ and let 
$\alpha \in \Hom(V, W)$.  Let $\ann{\alpha}$ denote the set of those 
$\beta \in \Hom(W,Y)$ satisfying the condition that $\beta\alpha$ is the 0-transformation.
That is, 
\[
\ann{\alpha} = \{ \beta\in \Hom(W,Y) \mid \forall v \in V \; \beta\alpha(v) = 0_Y\}.
\]
Prove that $\ann{\alpha}$ is a subspace of $\Hom(W,Y)$ and compute its
dimension.
\end{problem}
 \smallskip
 \begin{solution}
 First we show that $\ann{\alpha}$ is a subspace of $\Hom(W,Y)$ by showing it is closed under addition and scalar multiplication. Letting $\beta, \gamma \in \ann{\alpha}$ and $a\in F$, then
 $$
 (\beta + \gamma)\alpha = \beta \alpha + \gamma \alpha = 0+0
 $$
 so $\ann{\alpha}$ is closed under addition. Also
 $$
 (a\beta)\alpha = a(\beta\alpha) = a\cdot 0 = 0
 $$
 so $\ann{\alpha}$ is closed under scalar multiplication and hence, a subspace of $\Hom(W,Y)$. 

 To compute the dimension of $\ann{\alpha}$, first recall by Proposition 8.1 $\dim{\Hom(W,Y)} = \dim{W}\cdot \dim{Y}$. Also since any $\beta \in \ann{\alpha}$ maps everything in the image of $\alpha$ to $0_Y$, we must have $\dim{\ann{\alpha}} = \dim{\Hom(W/\im{\alpha}, Y}$. It follows that
\begin{align*}
\dim{\ann{\alpha}} &= \dim{\Hom(W/\im{\alpha}, Y} \\
&= \dim{\Hom(W/\im{\alpha}}\cdot \dim{Y} \\
&= (\dim{W} - \dim{\im{\alpha}})\cdot \dim{Y}.
\end{align*}
 
 \end{solution}

\probskip

%%%%%%%%%%%%%%%%%%%%%%%%%%%%%%%%%%%%%%%%%%%%%%%%%%%%%%%%%%%%%%%%%%%%%%%%%%%%
\begin{ex}[Golan 253]
Let $V$ and $W$ be vector spaces over a field $F$ and assume that there are
subspaces $V_1$ and $V_2$ of $V$, both of positive dimension, satisfying
$V = V_1 \bigoplus V_2$.  For  $i=1, 2$, let 
$U_i = \{\alpha \in \Hom(V,W) \mid V_i \subseteq \ker{\alpha}\}$.
Show that $\{U_1, U_2\}$ is an independent set of subspaces of $\Hom(V,W)$.
Is it necessarily true that $\Hom(V,W) = U_1 \bigoplus U_2$?
\end{ex}
 \smallskip
 \begin{solution}
 To show $\{U_1, U_2\}$ is an independent set of subspaces of $\Hom(V,W)$, we wish to show for any $\alpha _i \in U_i$ that $\alpha_1 + \alpha _2 = \sigma _0$ ($\sigma _0$ being the zero map) if, and only if, $\alpha _i = \sigma _0$. The direction $\alpha _i = \sigma _0$ implying $\alpha_1 + \alpha _2 = \sigma _0$ is obvious, so we only consider the remaining.

 So suppose that $\alpha_1 + \alpha _2 = \sigma _0$. Then letting $v \in V$ be arbitrary, since $V = V_1 \bigoplus V_2$ we can write $v = v_1 + v_2$ uniquely where $v_i\in V_i$. Now by assumption
 $$
0 = (\alpha_1 + \alpha _2)(v) = \alpha_1(v_1+v_2) + \alpha _2(v_1+v_2) = \alpha _1(v_1) + \alpha_1(v_2) + \alpha _2(v_1) + \alpha_2(v_2)
 $$
 where $\alpha _i(v_i) = 0$ since $V_i \subseteq \ker{\alpha_i}$. Thus we must have $0 = \alpha_1(v_2) + \alpha _2(v_1)$. As long as $\alpha_1(v_2) = -\alpha _2(v_1)$ is not possible (I don't see why this is), then we must have
 $0 = \alpha _1 (v_2) = \alpha _2 (v_1)$. Since $v$ was arbitrary, then $\alpha _1$ maps everything in $V_2$ to 0 and similarly for $\alpha _2$ on $V_1$. Hence $\alpha _i = \sigma _0$ as desired.
 \end{solution}

\probskip

%%%%%%%%%%%%%%%%%%%%%%%%%%%%%%%%%%%%%%%%%%%%%%%%%%%%%%%%%%%%%%%%%%%%%%%%%%%%
\begin{problem}[Golan 256]
Let $V$ and $W$ be vector spaces over a field $F$.
Define a function $\phi : \Hom(V, W) \rightarrow \Hom(V\times W, V\times W)$
by setting
$\phi(\alpha): 
\begin{bmatrix} v\\ w \end{bmatrix} 
\mapsto 
\begin{bmatrix} 0_V\\ \alpha(v) \end{bmatrix}$.
Is $\phi$ a linear transformation of vector spaces over $F$? Is it a monomorphism?
\end{problem}
\smallskip
\begin{solution}
We verify that $\phi$ is a monomorphism as follows: let $\alpha,\beta\in \Hom(V,W)$ and let $c\in F$.  Then we have

\begin{align*}
\phi(\alpha + \beta)\begin{bmatrix}v\\w\end{bmatrix} 
    &= \begin{bmatrix}0_V\\(\alpha + \beta)(v)\end{bmatrix}\\
    &= \begin{bmatrix}0_V\\\alpha(v)\end{bmatrix} + \begin{bmatrix}0_V\\\beta(v)\end{bmatrix}\\
    &= \phi(\alpha)\begin{bmatrix}v\\w\end{bmatrix} + \phi(\beta)\begin{bmatrix}v\\w\end{bmatrix}\\[0.3cm]
\phi(c\alpha)\begin{bmatrix}v\\w\end{bmatrix} 
    &= \begin{bmatrix}0_V\\(c\alpha)(v)\end{bmatrix}\\
    &= \begin{bmatrix}0_V\\c\alpha(v)\end{bmatrix}\\
    &= c\begin{bmatrix}0_V\\\alpha(v)\end{bmatrix}\\
    &= c\phi(\alpha)\begin{bmatrix}v\\w\end{bmatrix}
\end{align*}

Now suppose that $\phi(\alpha) = \phi(\beta)$.  Then we have $\begin{bmatrix}0_V\\\alpha(v)\end{bmatrix} = \phi(\alpha)\begin{bmatrix}v\\w\end{bmatrix} = \phi(\beta)\begin{bmatrix}v\\w\end{bmatrix} = \begin{bmatrix}0_V\\\beta(v)\end{bmatrix}$.  Hence, $\alpha(v) = \beta(v)$ for all $v\in V$, and $\alpha = \beta$.  Therefore, $\phi$ is a monomorphism.
\end{solution}

\probskip


%%%%%%%%%%%%%%%%%%%%%%%%%%%%%%%%%%%%%%%%%%%%%%%%%%%%%%%%%%%%%%%%%%%%%%%%%%%%
\begin{problem}[Golan 293 \& 294]
Let $V$, $W$ and $Y$ be vector spaces over a field $F$. Prove the following:

\begin{enumerate}
\item  
If $\alpha \in \Hom(V,W)$ is an epimorphism, then for every 
$\beta \in \Hom(Y,W)$ there exists $\theta \in \Hom(Y,V)$ such that $\beta = \alpha \theta$.
\item  If $\alpha \in \Hom(V, W)$ is a monomorphism, then 
 for every $\beta \in \Hom(V,Y)$ there exists $\theta \in \Hom(W,Y)$ such that $\beta = \theta \alpha$.
\end{enumerate}


\end{problem}
\smallskip
\begin{solution}
(1) For $v_1$ and $v_2$ in $V$, consider $v_1$ and $v_2$ equivalent ($v_1\sim v_2$) if $v_1-v_2\in\ker\alpha$.  As the name implies, this is an equivalence relation, which is easily verified.  For each equivalence class, pick exactly one representative, so that the equivalence class may be represented $[v]$ for some $v\in V$. 

Take $\theta:y\mapsto v$ such that $\alpha(v) = \beta(y)$, and $v$ is the representative of some equivalence class.  This map is well-defined, for if $v_1 = \theta(y) = v_2$, then $\alpha(v_1) = \alpha(v_2)$, so $\alpha(v_1-v_2) = 0$, $v_1-v_2\in\ker\alpha$, and $v_1$ and $v_2$ have the same equivalence class representative.  Further, $\alpha$ is still surjective over the equivalence class representatives, for if $v_1\sim v_2$, then $\alpha(v_1-v_2) = 0$, and $\alpha(v_1) = \alpha(v_2)$.

We verify that $\theta$ is a homomorphism.  Let $y_1,y_2\in Y$, suppose that $\theta(y_1) = v_1$, $\theta(y_2) = v_2$, and let $c\in F$.  Then we have $\beta(y_1 + y_2) = \beta(y_1) + \beta(y_2) = \alpha(v_1) + \alpha(v_2) = \alpha(v_1 + v_2)$.  Hence, $\theta(y_1) + \theta(y_2) = \theta(y_1 + y_2)$ (allowing for representing vectors by their representatives).  Also, $\alpha(cv_1) = c\alpha(v_1) = c\beta(y_1) = \beta(cy_1)$, so $c\theta(y_1) = cv_1 = \theta(cy_1)$.  Hence, $\theta$ is the desired homomorphism.\vspace{0.2cm}

(2) Take $\theta:\alpha(v)\mapsto\beta(v)$.  This map is well-defined since $\alpha$ is injective.  We verify that $\theta$ is a homomorphism.  Let $v_1,v_2\in V$ and let $c\in F$.  Then we have $\theta(\alpha(v_1) + \alpha(v_2)) = \theta(\alpha(v_1+v_2)) = \beta(v_1+v_2) = \beta(v_1) + \beta(v_2) = \theta(\alpha(v_1))+\theta(\alpha(v_2))$. Also, $\theta(c\alpha(v_1)) = \theta(\alpha(cv_1)) = \beta(cv_1) = c\beta(v_1) = c\theta(\alpha(v_1))$. Hence, $\theta$ is the desired homomorphism. 
\end{solution}


\probskip

%%%%%%%%%%%%%%%%%%%%%%%%%%%%%%%%%%%%%%%%%%%%%%%%%%%%%%%%%%%%%%%%%%%%%%%%%%%%
\begin{problem}[Golan 296]\hskip-2mm\protect\footnotemark
\label{prob:296}
\footnotetext{The claim in this problem seems incorrect to me. 
  If you agree, give a counter-example, then modify the claim so it is correct and prove it.
  If you disagree, and you believe the claim is correct, then prove it as given.}
Let $V$, $W$ be vector spaces over a field $F$, let 
$\alpha \in \Hom(V, W)$, and let $D$ be a nonempty linearly independent subset
of $\im{\alpha}$.  Show that there exists a basis $B$ of $V$ satisfying
$\{\alpha(v)\mid v \in B\} = D$.
\end{problem}
 \smallskip
 \begin{solution}
 The claim is incorrect, consider $V=W=\mathbb{R}^2$ over $\mathbb{R}$, the identity homomorphism $\alpha: \mathbb{R}^2 \rightarrow \mathbb{R}^2$, and take $D=\{(1,0)\}$. Then $D$ is linearly independent, but the image of any basis for $\mathbb{R}^2$ will have at least two elements and $D$ only has one, so we cannot have $\{\alpha(v)\mid v \in B\} = D$. We modify the claim as follows:\\

 \emph{Claim:} Let $V$, $W$ be vector spaces over a field $F$, let 
$\alpha \in \Hom(V, W)$, and let $D$ be a nonempty linearly independent subset
of $\im{\alpha}$.  Show that there exists a basis $B$ of $V$ satisfying
$\{\alpha(v)\mid v \in B\} \supseteq D$.\\

To prove this, suppose $D = \{w_1, w_2, \ldots, w_n\}$ is the linearly independent subset in the claim. Since $D$ is in the image of $\alpha$, for each $w_i$ there exists $v_i$ such that $\alpha (v_i) = w_i$. We claim the $v_i$ are linearly independent and to show this, suppose they are not. Then there exists $c_i \in F$ such that $\sum _{i=1}^nc_iv_i = 0$. But then 
$$
0 = \alpha\left( \sum _{i=1}^nc_iv_i \right) = \sum _{i=1}^nc_i \alpha(v_i) = \sum _{i=1}^nc_iw_i
$$
which is a contradiction to the fact that the $w_i$ are linearly independent. Hence the $v_i$ are linearly independent, so by Proposition 5.8 in \cite{Golan:2012} they are contained in some basis $B$ and hence $\alpha(B) \supseteq D$.

 \end{solution}

\probskip

%%%%%%%%%%%%%%%%%%%%%%%%%%%%%%%%%%%%%%%%%%%%%%%%%%%%%%%%%%%%%%%%%%%%%%%%%%%%
\begin{problem}[Golan 306]
Let $V$, $W$ and $Y$ be vector spaces over a field $F$.  Let 
$\{\alpha_1, \dots, \alpha_n\}$ be a finite subset of $\Hom(V, W)$
and let $\beta \in \Hom(V,Y)$ be a linear transformation satisfying
$\bigcap_{i=1}^n \ker{\alpha_i} \subseteq \ker{\beta}$.  Show that there exist
linear transformations $\gamma_1, \dots, \gamma_n$ in $\Hom(W,Y)$ satisfying
$\beta = \sum_{i=1}^n \gamma_i \alpha_i$.
\end{problem}
\smallskip
\begin{solution}
We shall use the facts that a homomorphism of spaces is uniquely determined by what it does to the basis elements and that a homomorphism of spaces always maps basis elements to basis elements (or to zero).  

First note that since $\bigcap_{i=1}^n \ker{\alpha_i} \subseteq \ker{\beta}$, not all of the $\alpha_k$s are zero on any basis element of $V$ for which $\beta$ is not also zero.  Let $B = \{b_i\}_i$ and consider the collection of basis elements for which $\alpha_1$ is nonzero.  For each of these basis elements $b$, let $\gamma_1\alpha_1(b) = \beta(b)$. Now consider for all the remaining basis elements ($B$ not including those basis elements for which $\gamma_1\alpha_1$ was nonzero) those on which $\alpha_2$ is nonzero.  For each of these basis elements $b$, let $\gamma_2\alpha_2(b) = \beta(b)$, and take $\gamma_2\alpha_2(b) = 0_Y$ for all other basis elements.  Consider for all the remaining basis elements those on which $\alpha_3$ is nonzero.  For each of these basis elements $b$, let $\gamma_3\alpha_3(b) = \beta(b)$, and take $\gamma_3\alpha_3(b) = 0_Y$ for all other basis elements.  Repeat this process to construct the collection of $\gamma_i$s.  

Each $\gamma_i$ is a homomorphism, as they are being constructed from the basis elements $\alpha_j(b)$ in $W$.  On the other hand, the sum of homomorphisms is a homomorphism, so $\sum_{i=1}^n \gamma_i \alpha_i$ is one, and it suffices to check that $\beta$ and $\sum_{i=1}^n \gamma_i \alpha_i$ agree on all of $B$.  This is trivial however, as for each $b\in B$, $b$ is mapped to a nonzero element in $Y$ for precisely one $\gamma_k\alpha_k$, and $\gamma_k\alpha_k(b) = \beta(b)$.  Hence, the theorem holds.
\end{solution}


\probskip

%%%%%%%%%%%%%%%%%%%%%%%%%%%%%%%%%%%%%%%%%%%%%%%%%%%%%%%%%%%%%%%%%%%%%%%%%%%%
\begin{problem}[Golan 266]
Let $A$ and $B$ be nonempty sets.  Let $V$ be the collection of all subsets of
$A$ and let $W$ be the collection of all subsets of $B$, both of which are
vector spaces over $\GF(2)$.  Any function $f: A\rightarrow B$ defines a
function $\alpha_f : W \rightarrow V$ by setting 
$\alpha_f: D \mapsto \{a \in A : f(a) \in D\}$.  Show that each such function
$\alpha_f$ defines a linear transformation, and find its kernel.
\end{problem}
 \smallskip
 \begin{solution}
 Recall $V$ and $W$ are vector spaces over $\GF(2)$ by taking vector addition to be symmetric difference and for a subset $A$ we have $0\cdot A = \emptyset$ and $1\cdot A = A$. Then for each $\alpha _f$ and any $D,E\in W$ we see that
 \begin{align*}
 \alpha_f (D+E) &= \{a \in A : f(a) \in D+E\}\\
 &= \{a \in A : f(a) \in (D\setminus E)\cup (E\setminus D) \}\\
&= \{a \in A : f(a) \in (D\setminus E)\} \cup \{a \in A : f(a) \in (E\setminus D) \}\\
&= \{a \in A : f(a) \in D\} \setminus \{a \in A : f(a) \in E\} \cup \{a \in A : f(a) \in E\} \setminus \{a \in A : f(a) \in D\}\\
&= \alpha_f (D) + \alpha_f(E).
 \end{align*}
Also we have
$$
\alpha_f (0 \cdot D) = \emptyset_V = 0 \cdot \alpha_f(D)
$$
and
$$
\alpha_f (1 \cdot D) = 1 \cdot \alpha_f(D)
$$
so indeed each $\alpha_f$ defines a linear transformation and we complete this problem by finding their kernel.

Now for some fixed $\alpha_f$, we see that
\begin{align*}
\ker{\alpha_f} &= \{D\in W : \alpha_f(D) = \emptyset _V\} \\
&= \{D\in W : f(a)\not \in D \text{ for all } a \in A \} \\
&= \{D\subseteq B : D\cap \im{f} = \emptyset \}.
\end{align*}
 \end{solution}


\bibliographystyle{plainurl}
\bibliography{Math700}

\end{document}
