\documentclass[11pt]{paper}
\usepackage[
letterpaper,
top    = 3cm,
bottom = 3cm,
left   = 3.00cm,
right  = 3.00cm]{geometry}


\usepackage{tikz-cd}
\usepackage{amsthm}


%%%%%%%%%%%%%%%%%%%%%%%%%%%%%%%%%%%%%%%%
% Basic packages
%%%%%%%%%%%%%%%%%%%%%%%%%%%%%%%%%%%%%%%%
\usepackage{amsmath,amsthm,amssymb}
\usepackage{mathtools}
\usepackage{etoolbox}
\usepackage{fancyhdr}
 \usepackage{xcolor}
\usepackage[colorlinks=true,urlcolor=blue,linkcolor=blue,citecolor=blue]{hyperref}
\usepackage{xspace}
\usepackage{comment}
\usepackage{url} % for url in bib entries
\usepackage{mathrsfs}


\theoremstyle{remark}
\newtheorem{theorem}{Theorem}
\newtheorem*{prop}{Proposition}
\newtheorem{problem}{Problem}
\newtheorem*{prob}{Problem}
\newtheorem*{solution}{{\bf Solution}}
\newtheorem*{hint}{{\it Hint}}
\newtheorem*{ex}{Exercise}


%%%%%%%%%%%%%%%%%%%%%%%%%%%%%%%%%%%%%%%%%%%%%%%%%%
%% Surround the problem and solution with 
%% \begin{ProbBox}  and   \end{ProbBox}
%% to prevent pagebreaks.
\newenvironment{ProbBox}{\noindent\begin{minipage}{\linewidth}}{\end{minipage}}

%%%%%%%%%%%%%%%%
% Acronyms     %
%%%%%%%%%%%%%%%%
\usepackage[acronym, shortcuts]{glossaries}

%% HERE IS HOW YOU DEFINE ACRONYMS:
\newacronym{FTA}{FTA}{Fundamental Theorem of Algebra}
\newacronym{CRT}{CRT}{Chinese Remainder Theorem}

% Make \ac robust.
\robustify{\ac}

%%%%%%%%%%%%%%%%%%%%%%%%
% Fancy page style     %
%%%%%%%%%%%%%%%%%%%%%%%%
\pagestyle{fancy}
\newcommand{\metadata}[2]{
  \lhead{}
  \chead{}
  \rhead{\bfseries Math 700: Linear Algebra}
  \lfoot{#1}
  \cfoot{#2}
  \rfoot{\thepage}
}
\renewcommand{\headrulewidth}{0.4pt}
\renewcommand{\footrulewidth}{0.4pt}


\newrobustcmd*{\vocab}[1]{\emph{#1}}
\newrobustcmd*{\latin}[1]{\textit{#1}}

%%%%%%%%%%%%%%%%%%%%%%%%%%%%%%%%%%
% Customize list enviroonments   %
%%%%%%%%%%%%%%%%%%%%%%%%%%%%%%%%%%
% package to customize three basic list environments: enumerate, itemize and description.
\usepackage{enumitem}
\setitemize{noitemsep, topsep=0pt, leftmargin=*}
\setenumerate{noitemsep, topsep=0pt, leftmargin=*}
\setdescription{noitemsep, topsep=0pt, leftmargin=*}

%%%%%%%%%%%%%%%%%%%%%%%%%%%%
%% Space between problems  %
%%%%%%%%%%%%%%%%%%%%%%%%%%%%
\newrobustcmd*{\probskip}{\vskip1cm}


%%%%%%%%%%%%%%%%%%%%%%%%%%
%%    Math shortcuts     %
%%%%%%%%%%%%%%%%%%%%%%%%%%
\newcommand\join{\ensuremath{\vee}}
\newcommand\meet{\ensuremath{\wedge}}
\newcommand\R{\fld{R}}
\newcommand\End{\ensuremath{\operatorname{End}}}
\newcommand\Aut{\ensuremath{\operatorname{Aut}}}
\newcommand\Hom{\ensuremath{\operatorname{Hom}}}
\newcommand{\Aff}{\ensuremath{\operatorname{Aff}}}
\newcommand{\ann}[1]{\ensuremath{\operatorname{ann}(#1)}}
\newcommand{\id}{\ensuremath{\operatorname{id}}}
\newcommand{\nulity}[1]{\ensuremath{\operatorname{null}(#1)}}
\renewcommand{\ker}[1]{\ensuremath{\operatorname{ker}(#1)}}
\renewcommand{\dim}[1]{\ensuremath{\operatorname{dim}(#1)}}
\newcommand\im[1]{\ensuremath{\operatorname{im}(#1)}}
\newcommand{\rank}[1]{\ensuremath{\operatorname{rank}(#1)}}
\newcommand{\trace}[1]{\ensuremath{\operatorname{trace}(#1)}}
\renewcommand{\phi}{\ensuremath{\varphi}}


%%test This is the Homework LaTeX template.  Use this file to fill in your solutions. 
%%
%% Notes: 
%%    1. Write your answers inside a \begin{solution}...\end{solution} environment.
%%
%%    2. If you will use references, insert bibtex reference entries in the file
%%       Math700.bib.  (Create that file if it doesn't yet exist.)
%%
%%    3. If you will use acronyms, please define them in the macros.tex file.
%%
%%    4. Please try to check that your file compiles:
%%
%%       Mac OS X users: you might try MacTeX. 
%%       Windows users: you might try proTeXt. 
%%       Linux users: most come with TeX; otherwise do a full install of TeXLive.
%%
%%       There is a Makefile in this directory, so on Linux you could just 
%%       enter `make` to compile all the Homework*.tex files at once.
%%
%%    5. Please don't hesitate to inform the prof if you have trouble, or open
%%       a ``New issue'' or create a new ``Wiki page'' on GitHub.  Otherwise,
%%       send an email to williamdemeo@gmail.com.
%%
%%    6. It will probably be hard to keep everyone's notation consistent.
%%       For the most basic symbols, we should have some conventions and use
%%       LaTeX macros to keep the conventions consistent and easy to remember.
%%       For example, to denote an algebra,
         \newcommand\alg[1]{\ensuremath{\mathbf{#1}}}
         \newcommand{\<}{\ensuremath{\langle}}
         \renewcommand{\>}{\ensuremath{\rangle}}
%%       So, an algebra in LaTeX is typed as $\alg{A} = \<A, F\>$.
%%       Similarly, for a field, let's use:
         \newcommand\fld[1]{\ensuremath{\mathbb{#1}}}
%%       So, a field in LaTeX is typed as $\fld{F}$.
%%       Example: We denote the integers by \fld{Z}. 
%%                This comes up often enough that it's useful to define
         \newcommand\Z{\fld{Z}}
%%
%%       For the Galois field, we use:
         \newcommand\GF{\ensuremath{\operatorname{GF}}}
%%
%%    7. Replace these names with yours!!!
         \metadata{Alexander}{Homework 5 -- 2014/04/21}
         \author{Alexander Brylev}
%%
%%    8. Update the title and date as appropriate.
         \title{Homework 5}
         \date{due date: 2014/04/30}

\begin{document}

\maketitle

\noindent The label ``Problem'' is used for required problems. ``Exercise''
is for suggested exercises.

%%%%%%%%%%%%%%%%%%%%%%%%%%%%%%%%%%%%%%%%%%%%%%%%%%%%%%%%%%%%%%%%%%%%%%%%%%%%
\begin{problem}[Golan 307]
Let $V$ be a vector space over a field $F$ and let $W$ be a subspace of $V$.  
For each $v \in V$, let $v + W = \{v+w \mid w \in W\}$.  Let 
$V/W = \{v+W \mid v \in V\}$ be the collection of all sets of the form $v+W$,
and define operations of addition and scalar multiplication on $V/W$ by 
setting $(v+W) + (v'+W) = (v+v')+W$ and $c(v+W) = (cv)+W$ for all $v, v' \in V$
and $c\in F$.  Show that
\begin{enumerate}
\item $v+W = v'+W$ if and only if $v-v'\in W$;
\item $V/W$, with the given operations, is a vector space over $F$;
\item The function $v\mapsto v+W$ is an epimorphism from $V$ to $V/W$,
the kernel of which equals $W$;
\item Every complement of $W$ in $V$ is isomorphic to $V/W$;
\item If $(v+W)\cap (v'+W) \neq \emptyset$, then $v+W = v'+W$.
\end{enumerate}
The space $V/W$ is called the \emph{factor space} of $V$ by $W$.
\end{problem}
\smallskip
\begin{solution}

\begin{enumerate}

\item Let $v+W = v'+W$ for some $v, v' \in V$. Then $v+0_W = v'+w$
for some $w \in W$, so that $v-v' = w-0_W \in W$.
\smallskip
Now let $v-v' \in W$. Then $v-v' = w'$ for some $w' \in W$, so that for
any $w \in W$ we have $v+w = (v'+w')+w = v'+(w'+w) \in v'+W$, which
shows $v+W \in v'+W$. Due to symmetry, we must also have $v'+W \in v+W$.
Hence $v+W = v'+W$.
\item First, let's check that the given operations are well-defined, i.e. if
$u+W = v+W$ for some $u,v \in W$ and $u'+W = v'+W$ for some $u',v' \in W$,
then $(u+u')+W = (v+v')+W$ and $cu+W = cv+W$ for all $c \in F$.
\smallskip
If $u+W = v+W$ and $u'+W = v'+W$ for some $u,v,u',v' \in W$, then, by part 1,
$u-v = w$ and $u'-v' = w'$ for some $w, w' \in W$. So $(u+u')-(v+v') = w+w' \in W$,
and, using part 1 again, we get $(u+u')+W = (v+v')+W$. Also, $cu-cv = c(u-v) = cw \in W$,
so that $cu+W = cv+W$, by part 1 as well. Thus, the given operations are indeed well-defined.
\smallskip
To show $V/W$, with the given operations, is a vector space, we first need to prove $V/W$
is an abelian group.
\smallskip
Let $u+W, u_1+W, u_2+W, u_3+W \in V/W$. Then
\smallskip
i) $((u_1+W)+(u_2+W))+(u_3+W) = ((u_1+u_2)+W)+(u_3+W) =
((u_1+u_2)+u_3)+W = (u_1+(u_2+u_3))+W = (u_1+W)+((u_2+u_3)+ W) =
(u_1+W)+((u_2+W)+(u_3+W))$;
\smallskip
ii) $(0_V+W)+(u+W) = (0_V+u)+W = u+W = (u+0_V)+W = (u+W)+(0_V+W)$;
\smallskip
iii) $(-u+W)+(u+W) = (-u+u)+W = 0_V+W$;
\smallskip
iv) $(u_1+W)+(u_2+W) = (u_1+u_2)+W = (u_2+u_1)+W = (u_2+W)+(u_1+W)$.
\smallskip
Hence $\underline{V/W}=<V/W,+,-,0_V+W>$ is an abelian group.
\smallskip
For each $r \in F$ consider $f_r: \underline{V/W} \rightarrow \underline{V/W}$ by $f_r(v+W)=rv+W$. 
\smallskip
Let $r, r_1, r_2 \in F$ and $v+W, v_1+W, v_2+W \in V/W$. Then
\smallskip
i) $f_r((v_1+W)+(v_2+W))=f_r((v_1+v_2)+W)=r(v_1+v_2)+W=(rv_1+rv_2)+W=(rv_1+W)+(rv_2+W)=
r(v_1+W)+r(v_2+W)=f_r(v_1+W)+f_r(v_2+W)$;
\smallskip
ii) $f_{r_1+r_2}(v+W)=(r_1+r_2)v+W=(r_1v+r_2v)+W=(r_1v+W)+(r_2v+W)=r_1(v+W)+r_2(v+W)=
f_{r_1}(v+W)+f_{r_2}(v+W)$;
\smallskip
iii) $f_{r_1}(f_{r_2}(v+W))=f_{r_1}(r_2v+W)=r_1r_2v+W=f_{r_1r_2}(v+W)$;
\smallskip
iv) $f_1(v+W)=1v+W=v+W$.
\smallskip
Therefore $V/W$ is a vector space over $F$.
\item The map $f:V \rightarrow V/W$ defined via $f(v)=v+W$ is obviously surjective and 
$f(v+u)=(v+u)+W=(v+W)+(u+W)=f(v)+f(u)$ for all $u,v \in V$, which shows it is a homomorphism. 
Using part 1, $f(v)=v+W=0+W$ if and only if $v=v-0 \in W$, which shows $Ker(f)=W$.
\item Using the first homomorphism theorem and part 3, $W^c=V/Ker(f) \cong Im(f)=V/W$.
\item Let $(v+W)\cap (v'+W) \neq \emptyset$. This means $v+w=v'+w'$ for some $w,w' \in W$, which yields 
$v-v'=w'-w \in W$. By part 1, this implies $v+W=v'+W$.

\end{enumerate}

\end{solution}

\probskip

%%%%%%%%%%%%%%%%%%%%%%%%%%%%%%%%%%%%%%%%%%%%%%%%%%%%%%%%%%%%%%%%%%%%%%%%%%%%
\begin{problem}[Golan 325]
Let $\alpha \in \Aut(\R^2)$ be defined by 
$\alpha: 
\begin{bmatrix} a\\ b \end{bmatrix} 
\mapsto 
\begin{bmatrix} -b\\ a \end{bmatrix}$.
Show that $\R\{\alpha, \sigma_1\}$ is a unital subalgebra of $\End(\R^2)$.  Show
that it is proper by giving an example of an endomorphism of $\R^2$ not in this
subalgebra. 
\end{problem}
% \smallskip
% \begin{solution}
% \end{solution}

\probskip

%%%%%%%%%%%%%%%%%%%%%%%%%%%%%%%%%%%%%%%%%%%%%%%%%%%%%%%%%%%%%%%%%%%%%%%%%%%%
\begin{problem}[Golan 326]
Let $V$ be the space of all real-valued functions on the interval $[-1, 1]$
which are infinitely differentiable, and let $\delta$ be the endomorphism of $V$
which assigns to each function $f$ its derivative.  Find the kernel and image of
$\delta$. 
\end{problem}
\smallskip
\begin{solution}

We have $Ker\delta=\{v \in V : \delta(v)=0 \} = \{f \in C^{\infty}[-1,1] : f'=0 \} = 
\{f \in C^{\infty}[-1,1] : f(x) = c$ for some c $\in R\}$, i.e., a set of all constant 
functions defined on $[-1,1]$.
\smallskip
We have $Im\delta=\{v \in V : \delta(u)=v$ for some $u \in V \} = 
\{f \in C^{\infty}[-1,1] : g'=f$ for some $g \in C^{\infty}[-1,1] \} = 
\{f \in C^{\infty}[-1,1] : $ f is integrable on $[-1,1] \}$, i.e., a set of all 
integrable functions from $C^{\infty}[-1,1]$.

\end{solution}

\probskip

%%%%%%%%%%%%%%%%%%%%%%%%%%%%%%%%%%%%%%%%%%%%%%%%%%%%%%%%%%%%%%%%%%%%%%%%%%%%
\begin{problem}[Golan 338]
Let $V$ be a vector space over a field $F$ which is not finitely generated, and
let $\sigma_0 \neq \alpha \in \End(V)$.  Set 
$A = \{\beta \in \End(V) \mid \alpha \beta = \sigma_1\}$.  Show that if $A$ has
more than one element then it is infinite.
\end{problem}
\smallskip
\begin{solution}

Suppose $A$ has two elements, $\beta_1$ and $\beta_2$. Then there exists a basis 
vector $v$ of $V$ such that $\beta_1(v) \ne \beta_2(v)$. For $n \ge 3$, define 
$\beta_n \in \End(V)$ via $\beta_n(v)=(n-1)\beta_1(v)-(n-2)\beta_2(v)$ and 
$\beta_n(u)=\beta_1(u)$, where $u$ is a basis vector of $V$ such that $u \ne v$. 
Then $\alpha\beta_n(v)=(n-1)\alpha\beta_1(v)-(n-2)\alpha\beta_2(v)=(n-1)v-(n-2)v=v$ 
and $\alpha\beta_n(u)=\alpha\beta_1(u)=u$ for a basis vector $u$ of $V$ such that 
$u \ne v$. Thus, $\beta_n \in A$ for all $n$. 
For $n \ne k$, $\beta_n(v)-\beta_k(v)=(n-k)(\beta_1(v)-\beta_2(v)) \ne 0$, which 
shows $\beta_n \ne \beta_k$ for $n \ne k$. Hence $A$ contains infinitely many elements.

\end{solution}

\probskip

%%%%%%%%%%%%%%%%%%%%%%%%%%%%%%%%%%%%%%%%%%%%%%%%%%%%%%%%%%%%%%%%%%%%%%%%%%%%
\begin{problem}[Golan 340]
Let $V$ be a vector space over a field $F$ satisfying the condition that
$\alpha\beta = \beta\alpha$ for all $\alpha, \beta \in \End(V)$. Show that
$\dim{V} = 1$.
\end{problem}
\smallskip
\begin{solution}

Suppose $\dim{V}>1$. Then there exist two linearly independent vectors $e_1$ and 
$e_2$ in $V$. Define $\alpha \in \End(V)$ via $\alpha(e_1)=e_2$ and $\alpha(v)=0$ 
if $v \notin span(e_1)$. Define $\beta \in \End(V)$ via $\beta(e_2)=e_1$ and 
$\beta(v)=0$ if $v \notin span(e_2)$. Then $\alpha\beta(e_1)=\alpha(0)=0$, which 
does not equal $\beta\alpha(e_1)=\beta(e_2)=e_1$, a contradiction.

\end{solution}

\probskip

%%%%%%%%%%%%%%%%%%%%%%%%%%%%%%%%%%%%%%%%%%%%%%%%%%%%%%%%%%%%%%%%%%%%%%%%%%%%
\begin{problem}[Golan 354]
Let $V$ be a vector space over a field $F$ and let $\alpha \in \Aut(V)$.  Let
$W_1, \dots, W_k$ be subspaces of $V$ satisfying $V = \bigoplus_{i=1}^k W_i$.
For each $1\leq i \leq k$, let $Y_i = \{\alpha(w) \mid w \in W_i\}$.
Is $V = \bigoplus_{i=1}^k Y_i$?
\end{problem}
\smallskip
\begin{solution}

Let $y \in V$. Since $\alpha \in Aut(V)$, then $\alpha$ is surjective and hence 
there exists $x \in V$ such that $\alpha(x)=y$. Since $x \in V$ and 
$V = \bigoplus_{i=1}^k W_i$, then $x=w_1+w_2+...w_k$ for some $w_i \in W_i, 1 \le i \le k$. 
Then $y=\alpha(x)=\alpha(w_1+w_2+...+w_k)=\alpha(w_1)+\alpha(w_2)+...+\alpha(w_k) \in Y_1 + Y_2 + ... + Y_k$.

\smallskip

Let $v \in Y_i \cap Y_j$ for some $i \ne j$. Then $v=\alpha(w_i)=\alpha(w_j)$ 
for some $w_i \in W_i$ and $w_j \in W_j$. Since $\alpha \in Aut(V)$, then $\alpha$ 
is injective, and so $w_i=w_j \in W_i \cap W_j$. From $V = \bigoplus_{i=1}^k W_i$, 
it follows that $W_i \cap W_j = \{0\}$, and hence $w_i=w_j=0$. So $v=\alpha(w_i)=\alpha(0)=0$.

\end{solution}

%%%%%%%%%%%%%%%%%%%%%%%%%%%%%%%%%%%%%%%%%%%%%%%%%%%%%%%%%%%%%%%%%%%%%%%%%%%%
\begin{ex}[Golan 415]
Let $V$ be the subspace of $\R[X]$ consisting of all polynomials of degree less
than 3 and choose the basis $B = \{1, X, X^2\}$ for $V$. Let $\alpha \in \End(V)$ satisfy
\[
\Phi_{BB}(\alpha) = 
\begin{bmatrix} 
1 & 1 & 1\\ 
0 & 2 & 2\\
0 & 0 & 3
\end{bmatrix}.
\]
Let $D$ be the basis $\{1, X+1, 2X^2 + 4X + 3\}$ for $V$.  What is $\Phi_{DD}(\alpha)$?
\end{ex}

\probskip

%%%%%%%%%%%%%%%%%%%%%%%%%%%%%%%%%%%%%%%%%%%%%%%%%%%%%%%%%%%%%%%%%%%%%%%%%%%%
\begin{ex}[Golan 467]
Let $n$ be a positive integer and let $F$ be a field.  Let 
$A, B\in \mathcal{M}_{n\times n}(F)$ satisfy $A + B = I$.  Show that $AB = \mathbf{0}$
if and only if $A$ and $B$ are idempotent.
\end{ex}

\probskip

%%%%%%%%%%%%%%%%%%%%%%%%%%%%%%%%%%%%%%%%%%%%%%%%%%%%%%%%%%%%%%%%%%%%%%%%%%%%
\begin{ex}[Golan 530]
Let $n$ be a positive integer and let $F$ be a field.  If
$A \in \mathcal{M}_{n\times n}(F)$ is nonsingular, is the same necessarily true
of $A + A^T$?
\end{ex}

%\bibliographystyle{plainurl}
%\bibliography{Math700}

\end{document}
