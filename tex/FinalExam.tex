\documentclass[11pt]{paper}

%%%%%%%%%%%%%%%%%%%%%%%%%%%%%%%%%%%%%%%%
% Basic packages
%%%%%%%%%%%%%%%%%%%%%%%%%%%%%%%%%%%%%%%%
\usepackage{amsmath,amsthm,amssymb}
\usepackage[mathscr]{euscript}
\usepackage{mathtools}
\usepackage{tikz-cd}
\usepackage{etoolbox}
\usepackage{fancyhdr}
\usepackage{xcolor}
\usepackage{hyperref}
\usepackage{xspace}
\usepackage{comment}
\usepackage{url} % for url in bib entries
%\usepackage{mathrsfs}


\theoremstyle{remark}
\newtheorem{theorem}{Theorem}
\newtheorem*{prop}{Proposition}
\newtheorem{problem}{Problem}
\newtheorem*{prob}{Problem}
\newtheorem*{solution}{{\bf Solution}}
\newtheorem*{hint}{{\it Hint}}

%%%%%%%%%%%%%%%%%%%%%%%%%%%%%%%%%%%%%%%%%%%%%%%%%%
%% Surround the problem and solution with 
%% \begin{ProbBox}  and   \end{ProbBox}
%% to prevent pagebreaks.
\newenvironment{ProbBox}{\noindent\begin{minipage}{\linewidth}}{\end{minipage}}

%%%%%%%%%%%%%%%%%%%%%%%%%%%%%%%%%%%%%%%%
% Acronyms
%%%%%%%%%%%%%%%%%%%%%%%%%%%%%%%%%%%%%%%%
\usepackage[acronym, shortcuts]{glossaries}

%% HERE IS HOW YOU DEFINE ACRONYMS:
\newacronym{FTA}{FTA}{Fundamental Theorem of Algebra}
\newacronym{CRT}{CRT}{Chinese Remainder Theorem}

% Make \ac robust.
\robustify{\ac}

\usepackage{enumerate}

\usepackage[
top    = 3cm,
bottom = 3cm,
left   = 3.00cm,
right  = 3.00cm]{geometry}

%%%%%%%%%%%%%%%%%%%%%%%%%%%%%%%%%%%%%%%%
% Fancy page style
%%%%%%%%%%%%%%%%%%%%%%%%%%%%%%%%%%%%%%%%
\pagestyle{fancy}
\newcommand{\metadata}[2]{
  \rhead{Final Exam}
  \chead{}
  \lhead{Math 700: Linear Algebra}
%  \lfoot{#1}\cfoot{#2}
  % \lfoot{}\cfoot{}
  % \rfoot{\thepage}
}
\renewcommand{\headrulewidth}{0.4pt}
\renewcommand{\footrulewidth}{0.4pt}

\newrobustcmd*{\vocab}[1]{\emph{#1}}
\newrobustcmd*{\latin}[1]{\textit{#1}}

%%%%%%%%%%%%%%%%%%%%%%%%%%%%%%%%%%%%%%%%
% Customize list enviroonments
%%%%%%%%%%%%%%%%%%%%%%%%%%%%%%%%%%%%%%%%
% package to customize three basic list environments: enumerate, itemize and description.
% \usepackage{enumitem}
% \setitemize{noitemsep, topsep=0pt, leftmargin=*}
% \setenumerate{noitemsep, topsep=0pt, leftmargin=*}
% \setdescription{noitemsep, topsep=0pt, leftmargin=*}

%%%%%%%%%%%%%%%%%%%%%%%%%%%%%%%%%%%%%%%%
%% Space between problems
\newrobustcmd*{\probskip}{\vskip5mm}

\usepackage{tikz}
\usetikzlibrary{matrix,arrows}
%% 
         \newcommand\alg[1]{\ensuremath{\mathbf{#1}}}
         \newcommand{\<}{\ensuremath{\langle}}
         \renewcommand{\>}{\ensuremath{\rangle}}
         \newcommand\fld[1]{\ensuremath{\mathbb{#1}}}
         \newcommand\N{\ensuremath{\fld{N}}}
         \newcommand\R{\ensuremath{\fld{R}}}
         \newcommand\C{\ensuremath{\fld{C}}}
         \newcommand\spec{\ensuremath{\operatorname{spec}}}
         \newcommand\Ann{\ensuremath{\operatorname{Ann}}}
         \renewcommand\char{\ensuremath{\operatorname{char}}}
         \newcommand\GF{\ensuremath{\operatorname{GF}}}
         \newcommand\End{\ensuremath{\operatorname{End}}}
         \newcommand\Hom{\ensuremath{\operatorname{Hom}}}
         \newcommand{\Aff}{\ensuremath{\operatorname{Aff}}}
         \newcommand{\ann}[1]{\ensuremath{\operatorname{ann}(#1)}}
         \newcommand{\id}{\ensuremath{\operatorname{id}}}
         \newcommand{\nulity}[1]{\ensuremath{\operatorname{null}(#1)}}
         \renewcommand{\ker}[1]{\ensuremath{\operatorname{ker}(#1)}}
         \renewcommand{\dim}[1]{\ensuremath{\operatorname{dim}(#1)}}
         \newcommand\im[1]{\ensuremath{\operatorname{im}(#1)}}
         \newcommand{\rank}[1]{\ensuremath{\operatorname{rank}(#1)}}
         \newcommand{\trace}[1]{\ensuremath{\operatorname{trace}(#1)}}
         \renewcommand{\phi}{\ensuremath{\varphi}}
         \newcommand{\Sub}{\ensuremath{\operatorname{Sub}}}
         \renewcommand{\leq}{\ensuremath{\leqslant}}
         \renewcommand{\nleq}{\ensuremath{\nleqslant}}
         \renewcommand{\geq}{\ensuremath{\geqslant}}
         \renewcommand{\lneq}{\ensuremath{\lneqslant}}
         \renewcommand{\gneq}{\ensuremath{\gneqslant}}
         \renewcommand{\ngeq}{\ensuremath{\ngeqslant}}
         \newcommand{\sB}{\ensuremath{\mathscr{B}}}
         \newcommand{\sD}{\ensuremath{\mathscr{D}}}
         \newcommand{\sS}{\ensuremath{\mathscr{S}}}


         \metadata{Math 700}{Final Exam -- 2014/05/05}
         \author{}
%%
%%    8. Update the title and date as appropriate.
         \title{Final Exam}
         \subtitle{Math 700: Spring 2014}
         \date{5 May 2014}



\begin{document}

\maketitle
\vskip-1cm

\noindent Deadline: {\bf 5pm Monday, May 5}

\vskip5mm

\noindent {\bf INSTRUCTIONS:} 
\begin{itemize}
\item 
Solve the problems below. Write up your solutions,
giving complete justifications for all arguments.  When you have finished, 
\begin{quote}
\emph{turn in a hard copy of your solutions to my office by} {\bf 5pm Monday, May 5}. 
\end{quote}
If I am not in my office when you are ready to turn in your exam, 
please slide it under my office door.

\medskip

\item The questions are meant to test your understanding of elementary concepts, and
you should write down definitions of any technical terms you use, even if these
terms are mentioned in the statement of the problem. Of course, you must use
your best judgment about which definitions to state.  (You probably don't want
to define the integers or real numbers, for example.)

\medskip

\item It will help me (and probably your grade) if you do the following:
\begin{enumerate}
\item 
State what you are trying to prove.
\item 
Mention informally how you plan to prove it before giving the details.
\item If you believe your proof is complete, use an end-of-proof symbol (like
  QED or \qedsymbol); on the other hand, if you believe your proof is
  incomplete, say so.
\end{enumerate}
\end{itemize}

\noindent {\bf HONOR CODE:} You are expected to solve the exam problems on your own
  with no outside help.  You may consult the lecture notes and textbook for this
  course only.  No other books or internet usage is allowed.
  If you get stuck, please ask \emph{me} for help, and I may post hints on our wiki page. 


\medskip

\noindent When you finish the exam, please sign the following pledge:\\
\\
``On my honor as a student I,
\underline{\phantom{XXXXXXXXXXXXXXXX}}, have neither
given nor received unauthorized aid on this exam.''
\hbox{} \hskip .5in {\small (Print Name)}\\[4pt]
\begin{flushright} Signature: \underline{\phantom{XXXXXXXXXXXXXXXXXXXX}}
  Date: \underline{\phantom{XXXXXXXXXX}}
\end{flushright}

\newpage

\noindent {\bf NOTATION:}
For the most part, we follow the notation used in the textbook.
Recall that if $V$ is a vector space over the field $F$, then 
$W\leq V$ denotes that $W$ is a subspace of $V$, 
whereas $W\subseteq V$ means that $W$ is a subset of $V$ 
(which may or may not be a subspace).  If $\phi : V \rightarrow W$, then
$\im{\phi} := \phi(V)$, $\ker{\phi} := \{v \in V : \phi(v) = 0_W\}$.
Finally, if $\alpha$ belongs to $\End(V)$ or $\mathcal{M}_{n\times n}(F)$, then
$\spec(\alpha)$ denotes the set of eigenvalues of $\alpha$.


\probskip
%%%%%%%%%%%%%%%%%%%%%%%%%%%%%%%%%%%%%%%%%%%%%%%%%%%%%%%%%%%%%%%%%%%%%%%%%%%%
\begin{problem}
Let $F = \GF(q)$ be the finite field of order $q$, and let $V = F^n$.
\begin{enumerate}[(a)]
\item How many nonzero vectors are there in $V$?
\item Given a nonzero vector $v_1\in V$, how many vectors $v_2\in V$ are 
  such that $\{v_1, v_2\}$ is a linearly independent set?
\item Let $S = \{v_1, v_2, \dots, v_k\}\subset V$ be a subset of $k$ 
  linearly independent vectors.  Explain why there are exactly
  $(q^n-1)(q^n-q)(q^n-q^2)\cdots (q^n-q^k)$ such subsets of $V$.
\item Fix a subspace $W \leq V$ with $\dim{W} = k$.  How many distinct
  bases does $W$ have?
\item How many $k$-dimensional subspaces of $V$ are there?
\end{enumerate}
\end{problem}

\probskip
%%%%%%%%%%%%%%%%%%%%%%%%%%%%%%%%%%%%%%%%%%%%%%%%%%%%%%%%%%%%%%%%%%%%%%%%%%%%
\begin{problem}
Let $F$ be  a field and let $(K, \bullet)$ be an associative unital
$F$-algebra.  If $\alpha \in K$ and if $p(X) = \sum_{i=0}^k c_i X^i \in  F[X]$,
then $p(\alpha) = \sum_{i=0}^k c_i \alpha^i \in  F[X]$.  Recall that 
\[
\Ann(\alpha) := \{p(X) \in F[X] \mid p(\alpha) = 0_K\}.
\]
\begin{enumerate}[(a)]
\item State the definition of the \vocab{minimal polynomial}, $m_\alpha(X)$, of
  $\alpha$, and explain why $m_\alpha(X)$ is unique, if it exists.
\item Give a condition on $(K, \bullet)$ that guarantees existence of 
 $m_\alpha(X)$. A detailed proof is not required, but you should give some
  justification for your claim; e.g., by citing a result you learned in
  lecture or from the text. 
\item Suppose $A \in K = \mathcal{M}_{n\times n}(F)$. Is $m_A(X)$ guaranteed to
  exist in this case?
\end{enumerate}
\end{problem}

\probskip
%%%%%%%%%%%%%%%%%%%%%%%%%%%%%%%%%%%%%%%%%%%%%%%%%%%%%%%%%%%%%%%%%%%%%%%%%%%%
\begin{problem}
Let $\alpha\in \End(\R^3)$ be given by 
\[
\alpha 
\begin{pmatrix}
x \\  y \\ z
\end{pmatrix}
  = 
\begin{pmatrix}
x +z \\  y+z \\ x+y
\end{pmatrix}.
\]
%\begin{enumerate}[(a)]
\begin{enumerate}[(a)]
\item  Find the matrix $\Phi_{BB}(\alpha)$ that represents $\alpha$ with
  respect to the basis  
\[
B = \left\{ 
\begin{pmatrix}1 \\  0 \\ 0\end{pmatrix},
\begin{pmatrix}1 \\  1 \\ 0\end{pmatrix},
\begin{pmatrix}1 \\  1 \\ 1\end{pmatrix}\right\}.
\]
\item Find the characteristic and minimal polynomials of $\alpha$? 
\item Compute the eigenvalues of $\alpha$?
\end{enumerate}
\end{problem}


\probskip



%%%%%%%%%%%%%%%%%%%%%%%%%%%%%%%%%%%%%%%%%%%%%%%%%%%%%%%%%%%%%%%%%%%%%%%%%%%%
\begin{problem}
  Let $F$ be a field. Recall that a linear transformation is \vocab{diagonalizable}
  means there exists a basis of eigenvectors for the transformation. A pair of
  linear transformations is \vocab{simultaneously diagonalizable} means that there
  exists a basis consisting of vectors that are eigenvectors of both
  transformations. Said another way, there exists a change of basis matrix that
  diagonalizes both transformations. 
  \begin{enumerate}[(a)]
  \item Let $A \in \mathcal{M}_{n\times n}(F)$, let $W \subseteq F^n$ be a
    nontrivial $A$-invariant subspace, and let $A{\mid_W}$ denote the restriction of
    $A$ to $W$. Show that if $A$ is diagonalizable, then so is $A{\mid_W}$.\\
    (Hint: consider the minimum polynomials of $A$ and $A{\mid_W}$.)
  \item Let $\lambda \in F$ and let 
    $W_\lambda = \{x \in F^n\mid Ax = \lambda x\}$
    be the eigenspace of $A$ associated with $\lambda$. 
    Let $A$ be diagonalizable and let $\lambda_1, \dots, \lambda_k$ 
    be its distinct eigenvalues. Show that 
    $F^n = W_{\lambda_1} \oplus \cdots \oplus W_{\lambda_k}$.
  \item	Let $B \in \mathcal{M}_{n\times n} (F)$. Show that if $B$ commutes with $A$ 
    (i.e., $A B = BA$), then $B W_\lambda \subseteq W_\lambda$.
  \item Let $A, B \in \mathcal{M}_{n\times n}(F)$ and assume that both matrices
    are individually diagonalizable. Use the previous parts to show that if $A B = B A$,
    then $A$ and $B$ are simultaneously diagonalizable.
    \item Prove the converse: If $A$ and $B$ are simultaneously 
      diagonalizable, then $A B = B A$.%\\ (Hint: verify the condition for a basis.)

  \end{enumerate}

\end{problem}



\probskip
%%%%%%%%%%%%%%%%%%%%%%%%%%%%%%%%%%%%%%%%%%%%%%%%%%%%%%%%%%%%%%%%%%%%%%%%%%%%
\begin{problem}
Let $V = \C_3[X]$ be the vector space of polynomials with complex
coefficients and degree at most three. Define
$\alpha \in \End(V)$ by $\alpha p(X) = p(X + 1)$. 
Find the eigenvalues of $\alpha$ and, for each eigenvalue, 
describe the corresponding eigenspace. \\
(Hint: begin by finding $\Phi_{BB}(\alpha)$ for the basis $B = \{1, X, X^2,X^3\}$.)\\
%{\bf Extra credit:} Find the rational canonical decomposition for $\alpha$.

\end{problem}

\probskip


%%%%%%%%%%%%%%%%%%%%%%%%%%%%%%%%%%%%%%%%%%%%%%%%%%%%%%%%%%%%%%%%%%%%%%%%%%%%
\begin{problem}
  Let $V$ be a vector space over a field $F$. The \emph{dual space} of
  $V$ is defined by $V^* := \Hom(V, F)$.
  For a subspace $U\leq V$, the \emph{annihilator} of $U$ is
  \[
  \ann{U} := \{\theta \in V^* \mid \theta(u) = 0 \text{ for all } u \in U\}
  = \{\theta \in V^* \mid U \subseteq \ker{\theta}\}.
  \]
  As we have seen, $\ann{U} \leq V^*\leq F^V$, a chain of subspaces.

  \begin{enumerate}[(a)]
  \item For this part, assume that $V$ is finitely generated over $F$. 
    Show that if $\theta_1, \dots, \theta_r$ is a basis for $\ann{U}$,
    then $U = \bigcap\limits_{i=1}^r \ker{\theta_i}$.

\item If $W$ is another vector space over $F$, and if 
  $\phi \in \Hom(V, W)$, then we define the \emph{dual} of $\phi$ to be the
  linear transformation  $\phi^* \in \Hom(W^*, V^*)$ that takes each 
  $\beta \in W^*$ to the composition $\beta\circ \phi$.  
  That is, $\phi^*(\beta)  = \beta \circ \phi$. 
  \begin{center}
    \begin{tikzcd}
      V \arrow{dr}[swap]{\beta \circ \phi} \arrow{rr}{\phi} && W \arrow{dl}{\beta} \\ 
      &  F &
    \end{tikzcd}
  \end{center}
  Prove that $\ker{\phi^*} = \ann{\im{\phi}}$ and that $\im{\phi^*} =
  \ann{\ker{\phi}}$.
  \end{enumerate}

\end{problem}


\end{document}












%%%%%%%%%%%%%%%%%%%%%%%%%%%%%%%%%%%%%%%%%%%%%%%%%%%%%%%%%%%%%%%%%%%%%%%%
%%%%%%%%%%%%%%%%%%%%%%%%%%%%%%%%%%%%%%%%%%%%%%%%%%%%%%%%%%%%%%%%%%%%%%%%















\probskip
%%%%%%%%%%%%%%%%%%%%%%%%%%%%%%%%%%%%%%%%%%%%%%%%%%%%%%%%%%%%%%%%%%%%%%%%%%%%
\begin{problem}
  Let $\alpha, \beta \in \End(V)$.
  \begin{enumerate}[(a)]
%  \item  Define what it means for $\lambda$ to be an eigenvalue of $\alpha$.
  \item  Suppose $\lambda \in \spec(\alpha \beta)$ and $\lambda \neq 0$.
    Show that $\lambda \in \spec(\beta \alpha)$. \\
    (Hint: This can, and should, be done using the definitions of eigenvalue and
    eigenvector; that is, without using determinants of ``$\lambda$-matrices.'')
  \item Show that if $V$ is finitely generated and $0\in \spec(\alpha \beta)$,
    then $0 \in \spec(\beta \alpha)$.
  \item Give an example showing the previous part can be false if $V$ is
    not finitely generated.
  \end{enumerate}
\end{problem}

\probskip
%%%%%%%%%%%%%%%%%%%%%%%%%%%%%%%%%%%%%%%%%%%%%%%%%%%%%%%%%%%%%%%%%%%%%%%%%%%%
\begin{problem}
Let $J \in \mathcal{M}_{n\times n}(F)$ and suppose every entry of $J$ is 1 (the
multiplicative identity of $F$).
\begin{enumerate}[(a)]
\item Find the kernel of $J$.
\item Find the minimal polynomial of $J$. (Hint: Evaluate $J^2$.)
\item Find the eigenvalues of $J$.
%\item Find the Jordan normal form of $J$ both in the case $\char(F) \mid n$
%and $\char(F) \nmid n$.
\end{enumerate}
\end{problem}




%%%%%%%%%%%%%%%%%%%%%%%%%%%%%%%%%%%%%%%%%%%%%%%%%%%%%%%%%%%%%%%%%%%%%%%%%%%%
\begin{problem}
 Let 
\[
A = 
\begin{pmatrix}
  1 & 1 & -1 \\
  -1 & 3 & -1 \\
  -1 & 2 & 0 
\end{pmatrix}
\]
\begin{enumerate}[(a)]
\item Find the characteristic polynomial and the minimal polynomial of $A$.
\item What is the Jordan canonical form of $A$?
\end{enumerate}
\end{problem}

\probskip








%%%%%%%%%%%%%%%%%%%%%%%%%%%%%%%%%%%%%%%%%%%%%%%%%%%%%%%%%%%%%%%%%%%%%%%%%%%%
\begin{problem}
For each of the following statements, prove that it is true or give an example to
show that it is false. Throughout, $A \in \mathcal{M}_{n \times n}(\C)$ and 
$\spec(A)$ denotes the set of eigenvalues of $A$.
\begin{enumerate}[(a)]
\item 
If $\lambda \in \spec(A)$ and $\mu \in \C$, then 
$\lambda - \mu \in \spec(A - \mu I)$.
\item If every entry of $A$ is real and $\lambda \in \spec(A)$, then $-\lambda \in \spec(A)$.
\item If $\lambda \in \spec(A)$ and $A$ is nonsingular, then $\lambda^{-1}$ is
  an eigenvalue of $A^{-1}$.
\end{enumerate}
\end{problem}

\probskip


\probskip
%%%%%%%%%%%%%%%%%%%%%%%%%%%%%%%%%%%%%%%%%%%%%%%%%%%%%%%%%%%%%%%%%%%%%%%%%%%%
\begin{problem}
Let $\alpha \in \End(V)$ and let $\lambda_1, \lambda_2, \dots, \lambda_k$ be
distinct eigenvalues with corresponding eigenvectors $v_1, v_2, \dots, v_k$.
Show that $v_1, v_2, \dots, v_k$ are linearly independent. (Hint: Use induction.)
\end{problem}



\probskip
%%%%%%%%%%%%%%%%%%%%%%%%%%%%%%%%%%%%%%%%%%%%%%%%%%%%%%%%%%%%%%%%%%%%%%%%%%%%
\begin{problem}
Let $V$ be a vector space finitely generated over $F$ with basis 
$B = \{b_1, \dots, b_n\}$. Let $\alpha \in \End(V)$, and let 
$\sB := \Phi_{BB}(\alpha)$ be the matrix representation of $\alpha$ with respect to the
basis $B$.
\begin{enumerate}[(a)]
\item Suppose $D = \{d_1, \dots, d_n\}$ is another basis for $V$, 
and let $\sD:= \Phi_{DD}(\alpha)$.
Prove that $m_{\sD}(X) = m_\alpha(X) = m_{\sB}(X)$.
\item Define the \vocab{characteristic polynomial} of the matrix $\Phi_{BB}(\alpha)$.
\item State the relationship between the characteristic polynomial of
$\Phi_{BB}(\alpha)$ and the minimal polynomial $m_\alpha(X)$.  No proof is
  required, but you should name the theorem that allows you to relate
  these two polynomials. 
\item Prove that the roots of $m_\sB(X)$ are the eigenvalues of $\sB$.
\end{enumerate}
\end{problem}

%%%%%%%%%%%%%%%%%%%%%%%%%%%%%%%%%%%%%%%%%%%%%%%%%%%%%%%%%%%%%%%%%%%%%%%%%%%%
\begin{problem}
  Let $V$ be a finitely generated vector space over $\C$, and suppose 
  $\alpha \in \End(V)$ satisfies $\alpha^3 = \alpha$. (We call such $\alpha$ \vocab{tripotent}.) 
  Show that $\alpha$ is diagonalizable. (Hint: $X^3 - X \in \Ann(\alpha)$.)
  %% Let $A \in \mathcal{M}_{n\times n}(\C)$. Suppose $A^3 = A$ (i.e., $A$ is \vocab{tripotent}). 
  %% Show that $A$ is diagonalizable. (Hint: $A$ is annihilated by
  %% $p(X) =X^3 - X \in \C[X]$.)
\end{problem}

\probskip
