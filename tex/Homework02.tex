\documentclass[11pt]{paper}
\usepackage[letterpaper]{geometry}

<<<<<<< HEAD
\usepackage{tikz-cd}
\usepackage{amsthm}

=======
>>>>>>> 358e325480ac6f6e427bd26065219bf4a98e5290
%%%%%%%%%%%%%%%%%%%%%%%%%%%%%%%%%%%%%%%%
% Basic packages
%%%%%%%%%%%%%%%%%%%%%%%%%%%%%%%%%%%%%%%%
\usepackage{amsmath,amsthm,amssymb}
\usepackage{mathtools}
\usepackage{etoolbox}
\usepackage{fancyhdr}
\usepackage{xcolor}
\usepackage{hyperref}
\usepackage{xspace}
\usepackage{comment}
\usepackage{url} % for url in bib entries
\usepackage{mathrsfs}


\theoremstyle{remark}
\newtheorem{theorem}{Theorem}
\newtheorem*{prop}{Proposition}
\newtheorem{problem}{Problem}
\newtheorem*{prob}{Problem}
\newtheorem*{solution}{{\bf Solution}}
\newtheorem*{hint}{{\it Hint}}

%%%%%%%%%%%%%%%%%%%%%%%%%%%%%%%%%%%%%%%%%%%%%%%%%%
%% Surround the problem and solution with 
%% \begin{ProbBox}  and   \end{ProbBox}
%% to prevent pagebreaks.
\newenvironment{ProbBox}{\noindent\begin{minipage}{\linewidth}}{\end{minipage}}

%%%%%%%%%%%%%%%%%%%%%%%%%%%%%%%%%%%%%%%%
% Acronyms
%%%%%%%%%%%%%%%%%%%%%%%%%%%%%%%%%%%%%%%%
\usepackage[acronym, shortcuts]{glossaries}

%% HERE IS HOW YOU DEFINE ACRONYMS:
\newacronym{FTA}{FTA}{Fundamental Theorem of Algebra}
\newacronym{CRT}{CRT}{Chinese Remainder Theorem}

% Make \ac robust.
\robustify{\ac}

%%%%%%%%%%%%%%%%%%%%%%%%%%%%%%%%%%%%%%%%
% Fancy page style
%%%%%%%%%%%%%%%%%%%%%%%%%%%%%%%%%%%%%%%%
\pagestyle{fancy}
\newcommand{\metadata}[2]{
  \lhead{}
  \chead{}
  \rhead{\bfseries Math 700: Linear Algebra}
  \lfoot{#1}
  \cfoot{#2}
  \rfoot{\thepage}
}
\renewcommand{\headrulewidth}{0.4pt}
\renewcommand{\footrulewidth}{0.4pt}


\newrobustcmd*{\vocab}[1]{\emph{#1}}
\newrobustcmd*{\latin}[1]{\textit{#1}}

%%%%%%%%%%%%%%%%%%%%%%%%%%%%%%%%%%%%%%%%
% Customize list enviroonments
%%%%%%%%%%%%%%%%%%%%%%%%%%%%%%%%%%%%%%%%
% package to customize three basic list environments: enumerate, itemize and description.
\usepackage{enumitem}
\setitemize{noitemsep, topsep=0pt, leftmargin=*}
\setenumerate{noitemsep, topsep=0pt, leftmargin=*}
\setdescription{noitemsep, topsep=0pt, leftmargin=*}

%%%%%%%%%%%%%%%%%%%%%%%%%%%%%%%%%%%%%%%%
%% Space between problems
\newrobustcmd*{\probskip}{\vskip1cm}



%%test This is the Homework LaTeX template.  Use this file to fill in your solutions. 
%%
%% Notes: 
%%    1. Write your answers inside a \begin{solution}...\end{solution} environment.
%%
%%    2. If you will use references, insert bibtex reference entries in the file
%%       Math700.bib.  (Create that file if it doesn't yet exist.)
%%
%%    3. If you will use acronyms, please define them in the macros.tex file.
%%
%%    4. Please try to check that your file compiles:
%%
%%       Mac OS X users: you might try MacTeX. 
%%       Windows users: you might try proTeXt. 
%%       Linux users: most come with TeX; otherwise do a full install of TeXLive.
%%
%%       There is a Makefile in this directory, so on Linux you could just 
%%       enter `make` to compile all the Homework*.tex files at once.
%%
%%    5. Please don't hesitate to inform the prof if you have trouble, or open
%%       a ``New issue'' or create a new ``Wiki page'' on GitHub.  Otherwise,
%%       send an email to williamdemeo@gmail.com.
%%
%%    6. It will probably be hard to keep everyone's notation consistent.
%%       For the most basic symbols, we should have some conventions and use
%%       LaTeX macros to keep the conventions consistent and easy to remember.
%%       For example, to denote an algebra,
         \newcommand\alg[1]{\ensuremath{\mathbf{#1}}}
         \newcommand{\<}{\ensuremath{\langle}}
         \renewcommand{\>}{\ensuremath{\rangle}}
%%       So, an algebra in LaTeX is typed as $\alg{A} = \<A, F\>$.
%%       Similarly, for a field, let's use:
         \newcommand\fld[1]{\ensuremath{\mathbb{#1}}}
%%       So, a field in LaTeX is typed as $\fld{F}$.
%%       Example: We denote the integers by \fld{Z}. 
%%                This comes up often enough that it's useful to define
         \newcommand\Z{\fld{Z}}
%%
%%       For the Galois field, we use:
         \newcommand\GF{\ensuremath{\operatorname{GF}}}
%%
%%    7. Replace these names with yours!!!
         \metadata{Matt and Melissa}{Homework 2 -- 2014/02/19}
         \author{Matthew Corley and Melissa Murphy}
%%
%%    8. Update the title and date as appropriate.
         \title{Homework 2}
         \date{due date: tbd}

\begin{document}

\maketitle


\newcommand\join{\ensuremath{\vee}}
\newcommand\meet{\ensuremath{\wedge}}

%%%%%%%%%%%%%%%%%%%%%%%%%%%%%%%%%%%%%%%%%%%%%%%%%%%%%%%%%%%%%%%%%%%%%%%%%%%%
\begin{problem}[Golan 56]

Is it possible to define on $\Z/(4)$ the structure of a vector space over
$\GF(2)$ in such a way that the vector addition is the usual addition in
$\Z/(4)$?\\
\\
~[{\it Hints}: Recall that $(n)$ denotes the set $\{\dots, -2n, -n, 0, n, 2n,
  3n, \dots\}$, which we denoted in lecture by $n\Z$.  This is the ideal
  generated by $n$ in the ring $\Z$, but don't worry about that for now. Just
  take $\Z/(n)$ to be the abelian group of integers $\{0, 1, 2, \dots, n-1\}$
  with addition modulo $n$.  In lecture, we used $\Z/n\Z$ to denote
  $\Z/(n)$. Use whichever notation you prefer.]

\end{problem}
\smallskip
\begin{solution}
Assume toward a contradiction that $\Z/(4)$ is a vector space over GF(2), with vector addition defined as the usual modular addition. Then for any $v \in \Z/(4)$,
\begin{eqnarray*}
0 & = & (0)v \\ & = & (1+1)v \\ & = & 1 \cdot v + 1 \cdot v \\ & = & v + v
\end{eqnarray*}
Now, let $v = 3$ to see that $3 + 3 = 2\mod{4} \neq 0$. Thus we have a contradiction. The answer is no, it is not possible with the usual modular vector addition.
\end{solution}

\probskip

\newcommand\R{\fld{R}}

%%%%%%%%%%%%%%%%%%%%%%%%%%%%%%%%%%%%%%%%%%%%%%%%%%%%%%%%%%%%%%%%%%%%%%%%%%%%
\begin{problem}[Golan 60]\hskip-2mm\protect\footnotemark
\label{prob:2}
\footnotetext{In the original problem, the notation
 $f \boxplus g$ was used.  We use $f \join g$ instead, since this 
 is fairly standard notation for the function $\max\{f, g\}$.}
\renewcommand\boxplus{\ensuremath{\join}}
Let $V = C(0,1)$. Define the relation $\boxplus$ on $V$ by setting 
$(f \boxplus g)(x)= \max \{f(x), g(x)\}$.  
If we think of $\boxplus$ as a ``vector addition,'' does this, together with the
usual scalar multiplication, make $V$ into a vector space over $\R$?

\end{problem}
\smallskip
\begin{solution}
It is true that $C(0,1)$ is closed uder $\join$ (see the
\hyperref[appendix]{Appendix \ref*{appendix}} for a
proof), and we can easily verify that $\join$ is a commutative associative
binary operation on the set $C(0,1)$, so $\<C(0,1), \join\>$ is a 
commutative semigroup.  In fact, letting  
$f \meet g = \min\{f, g\}$, we can check that $\<C(0,1), \join,
\meet\>$  is a lattice.\footnote{See~\cite[Sec.~30]{Aliprantis:1998} for a
  discussion of vector lattices, such as $\< C(0,1), \join, \meet\>$.}

However, recall that a vector space is built up from an additive abelian group.
Is it possible for $\join$ to serve as vector addition?  If so, what would be
the additive identity?  We need a function $e \in C(0,1)$ such that for all $f
\in C(0,1)$ we have $f \join e = f$.  It is clear that no such $e$ exists.

\end{solution}
\probskip



%%%%%%%%%%%%%%%%%%%%%%%%%%%%%%%%%%%%%%%%%%%%%%%%%%%%%%%%%%%%%%%%%%%%%%%%%%%%
\begin{problem}[Golan 63]
\label{prob:3}
Let $V = \{ i \in \Z \mid 0 \leq i < 2^n \}$ for some fixed positive integer
$n$.  Define operations of vector addition and scalar multiplication on $V$ in
such a way as to turn it into a vector space over $\GF(2)$.\\
\\
~[{\it Hints}: Recall that $\GF(2)$ denotes the Galois field with two elements,
  $\{0, 1\}$, with addition mod 2 and the usual multiplication.
  Other than this field, the only restriction given in the
  problem is that $V$ must have $2^n$ elements. Do you know of any sets of this size?]
\end{problem}
\smallskip
\begin{solution}
Let $W$ be the set of binary strings of length $n$, that is, length $n$
sequences of 0's and 1's.  We can also view these as maps from the set 
$n := \{0, 1, \dots, n-1\}$ to the set $2 := \{0,1\}$.  So, in this sense, $W$
\emph{is} the set $2^n$ of maps from $n$ to $2$.  So, it is not really an abuse
of notation to write $W = 2^n$.

Since $|V| = 2^n$ (here $2^n$ is a number!), there is a bijection between $V$
and $W$, and we will identify each $i\in V$ with its string representation in $W$ using the notation
$i = (i_0, i_1, \dots, i_{n-1})$, where $i_k \in \{0,1\}$.
Define vector addition in $V$ by adding strings ``bitwise'' modulo 2.  That is
\begin{align*}
i+ j &= (i_0, i_1, \dots, i_{n-1}) + (j_1, \dots, j_{n-1})\\
&= (i_0, i_1, \dots, i_{n-1}) + (j_1, \dots, j_{n-1})\\
&= (i_0+j_0, i_1+j_1, \dots, i_{n-1}+j_{n-1})
\end{align*}
where for each $0\leq k < n$, the $k$-th element of $i+j$ is
\[
i_k + j_k =
\begin{cases}
  0, &i_k = j_k\\
  1, &i_k \neq j_k.
\end{cases}
\]
Clearly the latter addition is commutative, and therefore, the vector addition is
commutative: $i+j = j+i$.  The zero vector $\mathbf{0} = (0, \dots, 0)$ 
is the additive identity, and each vector is its own additive inverse, that is, 
$-v = v$. Thus, we have an abelian group $\<2^n, +, -, \mathbf{0}\>$.   
To make this into a vector space over $\GF(2)$, take 
the set of scalars $\{0, 1\}$ and define scalar multiplication
as follows: $0 i = \mathrm{0}$ and $1 i = i$.  It is easily verified that 
$\<2^n, +, \mathbf{0}, \{0, 1\}\>$ has the remaining ($\GF(2)$-module)
properties that make it a vector space over $\GF(2)$.
\end{solution}
\probskip

%%%% NOTE: delete the problem box environment if you want to allow problems to be
%%%% split across multiple pages.
\begin{ProbBox}  
%%%%%%%%%%%%%%%%%%%%%%%%%%%%%%%%%%%%%%%%%%%%%%%%%%%%%%%%%%%%%%%%%%%%%%%%%%%%
\newcommand{\F}{\fld{F}}
\begin{problem}[Golan 70]
Show that $\Z$ is not a vector space over any field.
\end{problem}
\smallskip
\begin{solution}
Let $\fld{F} = \GF(2)$. Assume $\Z$ is a vector space over $\F$. Then
\begin{eqnarray*}
0 & = & 1_Z(1_F + 1_F) \\
  & = & 1_Z(1_{F})+_Z 1_Z(1_F) \\
  & = & 2 
\end{eqnarray*}
But $0 \neq 2$, so $\Z$ is not a vector space over $\GF(2)$. Now let $\F$ be a field with characteristic greater than 2. Assume $\Z$ a vector space over $\F$. Then we know
$$ 1_F + 1_F = 2_F \implies 1_F = \frac{1}{2_F} + \frac{1}{2_F}$$
Therefore
\begin{eqnarray*}
1_Z & = & 1_F(1_Z) \\
	& = & (\frac{1}{2_F} + \frac{1}{2_F})1_Z \\
	& = & \frac{1}{2_F}(1_Z) +_Z \frac{1}{2_F}(1_Z) \\
\end{eqnarray*}
Now let $\frac{1}{2_F}(1_Z) = n \in \Z$.  But there is no element in $n \in \Z$ that satisfies $n + n = 1$. So $\Z$ is not a vector space over any field with characteristic greater than 2. Thus $\Z$ is not a vector space over any field. 
\end{solution}
\end{ProbBox}

\probskip

%%%%%%%%%%%%%%%%%%%%%%%%%%%%%%%%%%%%%%%%%%%%%%%%%%%%%%%%%%%%%%%%%%%%%%%%%%%%
\begin{problem}[Golan 76]

Let $V = \R^\R$ and let $W$ be the subset of $V$ containing the constant
function $x\mapsto 0$ and all of those functions $f \in V$ satisfying the
following condition: $f(a) = 0$ for at most finitely many real numbers $a$.  Is
$W$ a subspace of $V$.\\
\\~
[{\it Hint:} It's easy.]
\end{problem}
\smallskip
\begin{solution}
Let us assume toward a contradiction that $W$ is a subspace of $V$. Let $p(x), g(x), h(x)$ be distinct functions in $W$, such that $p(x) = g(x) + h(x)$ and $g(x) = 0$ for $x = b_0,\dots ,b_l \in \R$ and $h(x) = 0$ for $x = c_0, \dots , c_m \in \R$.  Then $p(x) = 0$ when $g(x) = h(x) = 0$ or when $g(x) = -h(x)$.  The former happens when $b_i = c_j$ for $1 \leq i \leq l, 1 \leq j \leq m$.  We can see this is a finite set of points.  The latter, however, could happen for an infinite number of points (e.g. define $g(x) = -h(x)$ for $x > \max(b_l,c_m) \in \R$). In that case, $p(a) = 0$ for infinitely many real numbers $a$, but it is not the constant function $x \mapsto 0$. So vector addition is not closed and therefore $W$ is not a subspace in $V$.
\end{solution}
\probskip



%%%%%%%%%%%%%%%%%%%%%%%%%%%%%%%%%%%%%%%%%%%%%%%%%%%%%%%%%%%%%%%%%%%%%%%%%%%%
\begin{ProbBox}  
\begin{problem}[Golan 79]
A function $f \in \R^\R$ is \emph{piecewise constant} if and only if it is a
constant function $x \mapsto c$ or there exist 
$a_1 < a_2 < \cdots < a_n$ and 
$c_0, c_1, \cdots, c_n$ in $\R$ such that 
\[
f : x \mapsto 
\begin{cases}
  c_0 & \text{ if $x < a_1$,}\\
  c_i & \text{ if $a_i\leq x < a_i$ for $1\leq i < n$,}\\
  c_n & \text{ if $a_n\leq x$.}
\end{cases}
\]
Does the set of all piecewise constant functions form a subspace of the vector
space $\R^\R$ over $\R$?

\end{problem}
\end{ProbBox}  

\smallskip
\begin{solution}

Let $W$ denote the set of piecewise constant functions. Then $W$ is clearly a
subset of $\R^{\R}$. Let $r \in \R$ and $k_i = rc_i$. Then
\[
rf : x \mapsto 
\begin{cases}
  k_0 & \text{ if $x < a_1$,}\\
  k_i & \text{ if $a_i\leq x < a_i$ for $1\leq i < n$,}\\
  k_n & \text{ if $a_n\leq x$.}
\end{cases}
\]
So $W$ is closed under scalar multiplication.

Let us describe $f, g \in W$ using the characteristic function $\chi$:
$$f(x) = \sum_{i=1}^n f(a_i)\chi_{[a_i,a_{i+1})}(x),$$
$$g(x) = \sum_{i=1}^m g(b_i)\chi_{[b_i,b_{i+1})}(x),$$
where $\chi_{[c_i,c_{i+1})}(x)$ is 1 if $c_i\leq x < c_{i+1}$ and 0 otherwise.

Now let the set $\{z_0,z_1,\dots, z_N\}$ be the union 
$\{a_0,a_1,\dots ,a_n\}\cup \{b_0,b_1,\dots ,b_m\}$
such that $z_0 < z_1 < \cdots < z_N$.\footnote{Note to students:
we use the word ``union'' explicitly to emphasize that the resulting
 $\{z_0,z_1,\dots, z_N\}$ will be a set---i.e., there will be no repetitions.}
Then we can describe the sum $f + g$ as follows:
$$f+g =  \sum_{i=1}^N (f(z_i)+g(z_i))\chi_{[z_i,z_{i+1})}.$$
Thus $f + g$ is a piecewise constant function and so $W$ is closed under vector
addition.  Therefore $W$ is a subspace of $\R^{\R}$. 
\end{solution}

\probskip

%%%%%%%%%%%%%%%%%%%%%%%%%%%%%%%%%%%%%%%%%%%%%%%%%%%%%%%%%%%%%%%%%%%%%%%%%%%%
\begin{problem}[Golan 81]

Let $W$ be the subset of $V = \GF(2)^5$ consisting of all vectors 
$(a_1, \dots, a_5)$ satisfying $\sum_{i=1}^5 a_i = 0$.  Is $W$ a subspace of $V$?

\end{problem}
\smallskip
\begin{solution}
We know $W$ is a subset of $V$ and $O_V \in V$. Let $x \in V, x = (a_1,a_2,a_3,a_4,a_5)$ and $\sum_{i=1}^{5}a_i = 0$. Let $b \in F$. Then 
$$bx = (b(a_1),b(a_2),b(a_3),b(a_4),b(a_5))$$
And
$$\sum_{i=1}^{5}(ba_i) = b\left(\sum_{i=1}^{5}a_i\right) = b(0) = 0$$
So $W$ is closed under scalar multiplication.

Let $y \in V$, $y = (c_1,c_2,c_3,c_4,c_5)$ such that $\sum_{i=1}^{5}c_i = 0$.  Then
$$x+y = (a_1+c_1,a_2+c_2,a_3+c_3,a_4+c_4,a_5+c_5)$$
Now let $d_i = a_i + c_i$.
\begin{eqnarray*}
\sum_{i=1}^{5}d_i & = & \sum_{i=1}^{5}(a_i + c_i) \\
				  & = & \sum_{i=1}^{5}a_i + \sum_{i=1}^{5}c_i \\
				  & = & 0 + 0 = 0
\end{eqnarray*}
Therefore $W$ is closed under vector addition and scalar multiplication so it is a subspace.
\end{solution}
\probskip


%%%%%%%%%%%%%%%%%%%%%%%%%%%%%%%%%%%%%%%%%%%%%%%%%%%%%%%%%%%%%%%%%%%%%%%%%%%%
\begin{problem}[Golan 85]
Let $V = \R^\R$ and let $W$ be the subset of $V$ consisting of all
functions $f$ satisfying the following condition: there exists $c\in \R$
(that depends on $f$) such that $|f(a)|\leq c|a|$ for all $a\in \R$.
Is $W$ a subspace of $V$?

\end{problem}
\smallskip
\begin{solution}

We know $W$ is a subset of $V$. Let $g \in W$. There exists a $c_1 \in \R$ such that $\forall x \in \R$  $|g(x)| \leq c_1|x|$.
Let $b \in \R$. Let $h = b g$. Then

\begin{align*}
|h(x)| & = |b(g(x))| \\
       & = |b| |g(x)| \\
       & \leq  |b|(c_1|x|) \\
       & =  c|x|,
\end{align*}
where $c = |b|c_1$. So $\exists c \in \R$ such that $\forall x \in \R$ we have $|h(x)| \leq c|x|$.
Therefore $W$ is closed under scalar multiplication.  

Let $p, g \in W$.  There exist $c_1,c_2 \in \R$ such that $\forall x \in \R$
$|p(x)| \leq c_1|x|$ and $|q(x)| \leq c_2|x|$.
If $t(x) = p(x) + g(x)$, then
\begin{align*}
t(x) & =  |p(x) + g(x)| \\
     & \leq  |p(x)| + |g(x)| \\
     & \leq c_1|x| + c_2|x| \\
     & =  (c_1+c_2)|x| \\
     & = c_3|x|,
\end{align*}
where $c_3 = c_1 + c_2$. Therefore $\exists c \in \R$ such that $\forall x \in \R$ we have
$t(x) \leq c|x|$.
Thus $W$ is closed under addition and scalar multiplication so $W$ is a subspace of $V$.
\end{solution}
\probskip



%%%%%%%%%%%%%%%%%%%%%%%%%%%%%%%%%%%%%%%%%%%%%%%%%%%%%%%%%%%%%%%%%%%%%%%%%%%%
\begin{problem}[Golan 93]
\label{prob:9}
Let $V$ be a vector space over a field $F$ and let $P$ be the collection of all
subsets of $V$, which we know is a vector space over $\GF(2)$.  Is the
collection of all subspaces of $V$ a subspace of $P$?

\end{problem}
\smallskip
\begin{solution}
Let $U$ be the collection of all subspaces of $V$. As all subspaces of $V$ must
be subsets of $V$, we know that $U \subseteq P$. We will prove that 
$U\nleq P$---that is, $U$ is not a \emph{subspace} of $P$---by showing that it
is not closed under the scalar multiplication of $P$.  

Indeed, fix a subspace $X \in U$, and recall that 
scalar multiplication $0 \in \GF(2)$ in $P$ always  
results in the empty set. (See the example in our textbook~\cite[p.~24]{Golan:2012}.)
That is, $0X = \emptyset$, which is not a subspace.
Since $0X\notin U$, we see that $U$ is not closed under the scalar
multiplication in $P$, so it is not a subspace of $P$.

\bigskip

\noindent {\it Followup question:} Could you fix the foregoing, perhaps by
letting $P$ be all subsets containing $0_V$ and defining $0X = \{0_V\}$ for
all $X \in P$?  It might not work if we take addition to be symmetric
difference, but what if we use the binary string interpretation, where the
string of all zeros corresponds to $\{0_V\}$?  Note that $P$ would then correspond to
the set of all binary strings of length $|V|-1$.
\end{solution}
\probskip



%%%%%%%%%%%%%%%%%%%%%%%%%%%%%%%%%%%%%%%%%%%%%%%%%%%%%%%%%%%%%%%%%%%%%%%%%%%%
\begin{problem}[Golan 105]

Let $V$ be a vector space over a field $F$ and let $0_V \neq w \in V$.  Given a
vector $v \in V \setminus Fw$, find the set $G$ of all scalars $a \in F$
satisfying $F\{v, w\} = F\{v, aw\}$.

\end{problem}
\smallskip
\begin{solution}

$$F\{v,w\} = \{bv + cw | b,c \in \fld{F} \} $$
$$F\{v,aw\} = \{dv + e(aw) | d,e \in \fld{F}\}$$

$F$ is a field, so it is closed under multiplication. $v$ is not a scalar multiple of $w$. We know $1 \in G$. Thus $\fld{F}\{v,aw\} \leq \fld{F}\{v,w\} \forall a \in \fld{F}$. Let $x \in \fld{F}\{v,w\}.$. Then for $s,t \in \fld{F}$

\begin{eqnarray*}
x & = & sv + tw  \\
  & = & sv + \frac{t}{a}(aw) \\
\end{eqnarray*}

So $\frac{t}{a} = ta^-1$, and $t \in \fld{F}, a^-1 \in \fld{F}$ so $ta^-1 \in \fld{F}$ as long as $a \neq 0$. So $G = \fld{F} \setminus 0_V$.
\end{solution}
\probskip


\section*{Appendix}
\label{appendix}

\noindent
{\it Notes on \hyperref[prob:2]{Problem \ref*{prob:2}}.}
Here we show how one could prove that $V$ is closed under $\boxplus$, that is, 
for all $f, g \in C(0,1)$, we have $f \boxplus g \in C(0,1)$. (Though, as we
noted in the solution to
\hyperref[prob:2]{Problem \ref*{prob:2}}, closure only proves that 
$\boxplus$ is a binary operation on $C(0,1)$; it does not prove that
$\boxplus$ can serve as vector addition.)

Let $\epsilon > 0$. Assume $f$,$g$ are continuous on $(0,1)$. Then, there exist $\delta_f > 0$ and $\delta_g > 0$ such that

$$ |x - x_0| < \delta_f \implies |f(x)-f(x_0)| < \frac{\epsilon}{2}$$ $$|x - x_0| < \delta_g \implies |g(x)-g(x_0)| < \frac{\epsilon}{2}$$
Let $h(x) = (f\join g)(x) = \max\{f(x),g(x)\}$. Let $\delta_h = \min\{\delta_f,\delta_g\}$. Then
\begin{eqnarray*}
|x - x_0| < \delta_h & \implies & |h(x) - h(x_0)| \\ & = & \left|\frac{f(x)+g(x)+|f(x)-g(x)|-f(x_0)-g(x_0)-|f(x_0)-g(x_0)|}{2}\right| \\
& \leq & \left|\frac{f(x)-f(x_0)}{2}\right| + \left|\frac{g(x)-g(x_0)}{2}\right|
+ \left|\frac{f(x)-f(x_0)-(g(x)-g(x_0))}{2}\right| \\
& < & \frac{\epsilon}{2} + \left|\frac{f(x)-f(x_0)-(g(x)-g(x_0))}{2}\right| \\
& < & \frac{\epsilon}{2} + \frac{\epsilon}{2} = \epsilon
\end{eqnarray*}
Therefore $f \boxplus g$ is continuous for $x_0 \in (0,1)$, so vector addition is closed in $V$. With the usual scalar multiplication, $V$ is a vector space over $\R$.

\bigskip

\noindent {\it Remarks.} The proof above is correct.  Alternatively, you could simply note that
the sum (and difference) of two continuous functions is continuous, and the
functions  $x \mapsto |x|$ and $x \mapsto x/2$ are continuous.  Therefore, since
function composition preserves continuity, it is clear that
$f \join g = \frac{1}{2}(f+g+|f-g|)$ is continuous. 

Yet another alternative is to use the fact that
$h \in C(0,1)$ if and only if for all $-\infty\leq a < b \leq \infty$ the set
$h^{-1}(a, b) = \{x \in (0,1) \mid a< h(x) < b\}$ is open in $(0,1)$.
Note that
\[
(f\join g)^{-1}(a, b) =
\left(\{x : a< f(x)\}\cup \{x : a< g(x)\}\right) \cap
\{x : f(x)< b\} \cap \{x : g(x)<b\}.
\]
If $f$ and $g$ are continuous, all of the sets on the right are open, so
$(f\join g)^{-1}(a, b)$ is open.

\probskip

\noindent {\it Notes on on \hyperref[prob:3]{Problem \ref*{prob:3}}.}
Consider the proposed solution:
\begin{quote}
Let us define vector addition in a ``bitwise xor'' fashion such that $v + v = 0$
and $v + w = 1$ for all $v,w \in V, w \neq v$. Furthermore, let us define scalar
multiplication in the natural way such that $1\cdot v = v$ and $0\cdot v = 0$.
Then we can see that vector addition is closed as $0,1 \in V$, as well as being
associative and commutative. And every $v$ has an additive inverse, namely
$v$. Scalar multiplication is also closed in $V$, as the product is always $0
\in V$ or $v \in V$. So $V$ is a vector space over $\GF(2)$.  
\end{quote}

\bigskip

\noindent {\it Remarks.} Having an addition that works as ``xor'' is the right idea.
However, this proof is incorrect.  In particular, you need an
additive identity, that is, an $e\in V$ such that $v + e = v$ for
all $v \in V$.  In the proposed solution, $0$ cannot serve as the additive identity
because $v+w = 1$ for all $w\neq v$; in particular, $v + 0 = 1$
whenever $v \neq 0$. (See the correct solution given above.)

\probskip

\noindent {\it Notes on on \hyperref[prob:9]{Problem \ref*{prob:9}}.}
The originally proposed solution began as follows:
\begin{quote}
$P$ is the collection of all subset of $V$. Let $U$ be the collection of all subspaces of $V$. As all subspaces of $V$ must be subsets of $V$, we know that $U \subseteq P$. We want to show that $U$ is a subspace of $P$.

Let $X \in U$. Then $X \in P$. Let $a \in F$. Let $y,z \in X$.  Then $ay, az \in
aX$. $ay + az = a(x+y)$.  As $X$ is a subspace, $(x+y) \in X$ so $a(x+y) \in
aX$. Thus $aX$ is closed under addition.
\end{quote}

Here we have fallen into the trap of considering the wrong field.  It is true
that we should fix some $X \in U$, and then try to show that for each field
element $a$, we have $aX \in U$.  However, we must take $a$ from $\GF(2)$, since
that is the field over which $P$ is defined, and we are trying to prove $U\leq P$.

Now, you might then argue that now it is even easier because we only have to
consider $0X$ and $1X$. The latter is simply $X$, but what is $0X$?  It
would be nice if it were $\{0_V\}$, because then we would have $0X= \{0_V\} \in
U$.  Unfortunately, in $P$, scalar multiplication by 0 gives the empty set (or
the length-$|V|$ string of zeros, if you prefer to think of the
elements of $P$ as binary strings).  The empty set is not a subspace, so $U$ is
not closed under the scalar multiplication of $P$.



%% If you will use references, add your refs to the Math700.bib file.
%% and then uncomment the following lines.
\bibliographystyle{plainurl}
\bibliography{Math700}

\end{document}
