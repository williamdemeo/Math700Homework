\documentclass[11pt]{paper}
\usepackage[letterpaper]{geometry}

<<<<<<< HEAD
\usepackage{tikz-cd}
\usepackage{amsthm}

=======
>>>>>>> 358e325480ac6f6e427bd26065219bf4a98e5290
%%%%%%%%%%%%%%%%%%%%%%%%%%%%%%%%%%%%%%%%
% Basic packages
%%%%%%%%%%%%%%%%%%%%%%%%%%%%%%%%%%%%%%%%
\usepackage{amsmath,amsthm,amssymb}
\usepackage{mathtools}
\usepackage{etoolbox}
\usepackage{fancyhdr}
\usepackage{xcolor}
\usepackage{hyperref}
\usepackage{xspace}
\usepackage{comment}
\usepackage{url} % for url in bib entries
\usepackage{mathrsfs}


\theoremstyle{remark}
\newtheorem{theorem}{Theorem}
\newtheorem*{prop}{Proposition}
\newtheorem{problem}{Problem}
\newtheorem*{prob}{Problem}
\newtheorem*{solution}{{\bf Solution}}
\newtheorem*{hint}{{\it Hint}}

%%%%%%%%%%%%%%%%%%%%%%%%%%%%%%%%%%%%%%%%%%%%%%%%%%
%% Surround the problem and solution with 
%% \begin{ProbBox}  and   \end{ProbBox}
%% to prevent pagebreaks.
\newenvironment{ProbBox}{\noindent\begin{minipage}{\linewidth}}{\end{minipage}}

%%%%%%%%%%%%%%%%%%%%%%%%%%%%%%%%%%%%%%%%
% Acronyms
%%%%%%%%%%%%%%%%%%%%%%%%%%%%%%%%%%%%%%%%
\usepackage[acronym, shortcuts]{glossaries}

%% HERE IS HOW YOU DEFINE ACRONYMS:
\newacronym{FTA}{FTA}{Fundamental Theorem of Algebra}
\newacronym{CRT}{CRT}{Chinese Remainder Theorem}

% Make \ac robust.
\robustify{\ac}

%%%%%%%%%%%%%%%%%%%%%%%%%%%%%%%%%%%%%%%%
% Fancy page style
%%%%%%%%%%%%%%%%%%%%%%%%%%%%%%%%%%%%%%%%
\pagestyle{fancy}
\newcommand{\metadata}[2]{
  \lhead{}
  \chead{}
  \rhead{\bfseries Math 700: Linear Algebra}
  \lfoot{#1}
  \cfoot{#2}
  \rfoot{\thepage}
}
\renewcommand{\headrulewidth}{0.4pt}
\renewcommand{\footrulewidth}{0.4pt}


\newrobustcmd*{\vocab}[1]{\emph{#1}}
\newrobustcmd*{\latin}[1]{\textit{#1}}

%%%%%%%%%%%%%%%%%%%%%%%%%%%%%%%%%%%%%%%%
% Customize list enviroonments
%%%%%%%%%%%%%%%%%%%%%%%%%%%%%%%%%%%%%%%%
% package to customize three basic list environments: enumerate, itemize and description.
\usepackage{enumitem}
\setitemize{noitemsep, topsep=0pt, leftmargin=*}
\setenumerate{noitemsep, topsep=0pt, leftmargin=*}
\setdescription{noitemsep, topsep=0pt, leftmargin=*}

%%%%%%%%%%%%%%%%%%%%%%%%%%%%%%%%%%%%%%%%
%% Space between problems
\newrobustcmd*{\probskip}{\vskip1cm}



%%test This is the Homework LaTeX template.  Use this file to fill in your solutions. 
%%
%% Notes: 
%%    1. Write your answers inside a \begin{solution}...\end{solution} environment.
%%
%%    2. If you will use references, insert bibtex reference entries in the file
%%       Math700.bib.  (Create that file if it doesn't yet exist.)
%%
%%    3. If you will use acronyms, please define them in the macros.tex file.
%%
%%    4. Please try to check that your file compiles:
%%
%%       Mac OS X users: you might try MacTeX. 
%%       Windows users: you might try proTeXt. 
%%       Linux users: most come with TeX; otherwise do a full install of TeXLive.
%%
%%       There is a Makefile in this directory, so on Linux you could just 
%%       enter `make` to compile all the Homework*.tex files at once.
%%
%%    5. Please don't hesitate to inform the prof if you have trouble, or open
%%       a ``New issue'' or create a new ``Wiki page'' on GitHub.  Otherwise,
%%       send an email to williamdemeo@gmail.com.
%%
%%    6. It will probably be hard to keep everyone's notation consistent.
%%       For the most basic symbols, we should have some conventions and use
%%       LaTeX macros to keep the conventions consistent and easy to remember.
%%       For example, to denote an algebra,
         \newcommand\alg[1]{\ensuremath{\mathbf{#1}}}
         \newcommand{\<}{\ensuremath{\langle}}
         \renewcommand{\>}{\ensuremath{\rangle}}
%%       So, an algebra in LaTeX is typed as $\alg{A} = \<A, F\>$.
%%       Similarly, for a field, let's use:
         \newcommand\fld[1]{\ensuremath{\mathbb{#1}}}
%%       So, a field in LaTeX is typed as $\fld{F}$.
%%       Example: We denote the integers by \fld{Z}. 
%%                This comes up often enough that it's useful to define
         \newcommand\Z{\fld{Z}}
%%
%%       For the Galois field, we use:
         \newcommand\GF{\ensuremath{\operatorname{GF}}}
%%
%%    7. Replace these names with yours!!!
         \metadata{Matt and Melissa}{Homework 2 -- 2014/02/??}
         \author{Matthew Corley and Melissa Murphy}
%%
%%    8. Update the title and date as appropriate.
         \title{Homework 2}
         \date{due date: tbd}

\begin{document}

\maketitle


%%%%%%%%%%%%%%%%%%%%%%%%%%%%%%%%%%%%%%%%%%%%%%%%%%%%%%%%%%%%%%%%%%%%%%%%%%%%
\begin{problem}[Golan 56]

Is it possible to define on $\Z/(4)$ the structure of a vector space over
$\GF(2)$ in such a way that the vector addition is the usual addition in
$\Z/(4)$?\\
\\
~[{\it Hints}: Recall that $(n)$ denotes the set $\{\dots, -2n, -n, 0, n, 2n,
  3n, \dots\}$, which we denoted in lecture by $n\Z$.  This is the ideal
  generated by $n$ in the ring $\Z$, but don't worry about that for now. Just
  take $\Z/(n)$ to be the abelian group of integers $\{0, 1, 2, \dots, n-1\}$
  with addition modulo $n$.  In lecture, we used $\Z/n\Z$ to denote
  $\Z/(n)$. Use whichever notation you prefer.]

\end{problem}
\smallskip
\begin{solution}
(type your solution here)
\end{solution}
\probskip



\newcommand\R{\fld{R}}

%%%%%%%%%%%%%%%%%%%%%%%%%%%%%%%%%%%%%%%%%%%%%%%%%%%%%%%%%%%%%%%%%%%%%%%%%%%%
\begin{problem}[Golan 60]

Let $V = C(0,1)$. Define an operation $\boxplus$ on $V$ by setting 
$f \boxplus g : x \mapsto \max \{f(x), g(x)\}$.  
Does this operation of vector addition, together with the usual scalar
multiplication make $V$ into a vector space over $\R$?

\end{problem}
\smallskip
\begin{solution}
(type your solution here)
\end{solution}
\probskip



%%%%%%%%%%%%%%%%%%%%%%%%%%%%%%%%%%%%%%%%%%%%%%%%%%%%%%%%%%%%%%%%%%%%%%%%%%%%
\begin{problem}[Golan 63]

Let $V = \{ i \in \Z \mid 0 \leq i < 2^n \}$ for some fixed positive integer
$n$.  Define operations of vector addition and scalar multiplication on $V$ in
such a way as to turn it into a vector space over $\GF(2)$.\\
\\
~[{\it Hints}: Recall that $\GF(2)$ denotes the Galois field with two elements,
  $\{0, 1\}$, with addition mod 2 and the usual multiplication.
  Other than this field, the only restriction given in the
  problem is that $V$ must have $2^n$ elements. Do you know of any sets of this size?]
\end{problem}
\smallskip
\begin{solution}
(type your solution here)
\end{solution}
\probskip

%%%% NOTE: delete the problem box environment if you want to allow problems to be
%%%% split across multiple pages.
\begin{ProbBox}  
%%%%%%%%%%%%%%%%%%%%%%%%%%%%%%%%%%%%%%%%%%%%%%%%%%%%%%%%%%%%%%%%%%%%%%%%%%%%
\begin{problem}[Golan 70]
Show that $\Z$ is not a vector space over any field.
\end{problem}
\smallskip
\begin{solution}
(type your solution here)
\end{solution}
\probskip
\end{ProbBox}


%%%%%%%%%%%%%%%%%%%%%%%%%%%%%%%%%%%%%%%%%%%%%%%%%%%%%%%%%%%%%%%%%%%%%%%%%%%%
\begin{problem}[Golan 76]

Let $V = \R^\R$ and let $W$ be the subset of $V$ containing the constant
function $x\mapsto 0$ and all of those functions $f \in V$ satisfying the
following condition: $f(a) = 0$ for at most finitely many real numbers $a$.  Is
$W$ a subspace of $V$.\\
\\~
[{\it Hint:} It's easy.]
\end{problem}
\smallskip
\begin{solution}

(type your solution here)

\end{solution}
\probskip



%%%%%%%%%%%%%%%%%%%%%%%%%%%%%%%%%%%%%%%%%%%%%%%%%%%%%%%%%%%%%%%%%%%%%%%%%%%%
\begin{problem}[Golan 79]

A function $f \in \R^\R$ is \emph{piecewise constant} if and only if it is a
constant function $x \mapsto c$ or there exist 
$a_1 < a_2 < \cdots < a_n$ and 
$c_0 < c_1 < \cdots < c_n$ in $\R$ such that 
\[
f : x \mapsto 
\begin{cases}
  c_0 & \text{ if $x < a_1$,}\\
  c_i & \text{ if $a_i\leq x < a_i$ for $1\leq i < n$,}\\
  c_n & \text{ if $a_n\leq x$.}
\end{cases}
\]
Does the set of all piecewise constant functions form a subspace of the vector
space $\R^\R$ over $\R$?

\end{problem}
\smallskip
\begin{solution}
(type your solution here)
\end{solution}
\probskip


%%%%%%%%%%%%%%%%%%%%%%%%%%%%%%%%%%%%%%%%%%%%%%%%%%%%%%%%%%%%%%%%%%%%%%%%%%%%
\begin{problem}[Golan 81]

Let $W$ be the subspace of $V = \GF(2)^5$ consisting of all vectors 
$(a_1, \dots, a_5)$ satisfying $\sum_{i=1}^5 a_i = 0$.  Is $W$ a subspace of $V$?

\end{problem}
\smallskip
\begin{solution}

(type your solution here)

\end{solution}
\probskip


%%%%%%%%%%%%%%%%%%%%%%%%%%%%%%%%%%%%%%%%%%%%%%%%%%%%%%%%%%%%%%%%%%%%%%%%%%%%
\begin{problem}[Golan 85]
Let $V = \R^\R$ and let $W$ be the subset of $V$ consisting of all
functions $f$ satisfying the following condition: there exists $c\in \R$
(that depends on $f$) such that $|f(a)|\leq c|a|$ for all $a\in \R$.
Is $W$ a subspace of $V$?

\end{problem}
\smallskip
\begin{solution}

(type your solution here)

\end{solution}
\probskip



%%%%%%%%%%%%%%%%%%%%%%%%%%%%%%%%%%%%%%%%%%%%%%%%%%%%%%%%%%%%%%%%%%%%%%%%%%%%
\begin{problem}[Golan 93]

Let $V$ be a vector space over a field $F$ and let $P$ be the collection of all
subsets of $V$, which we know is a vector space over $\GF(2)$.  Is the
collection of all subpaces of $V$ a subspace of $P$?

\end{problem}
\smallskip
\begin{solution}

(type your solution here)

\end{solution}
\probskip



%%%%%%%%%%%%%%%%%%%%%%%%%%%%%%%%%%%%%%%%%%%%%%%%%%%%%%%%%%%%%%%%%%%%%%%%%%%%
\begin{problem}[Golan 105]

Let $V$ be a vector space over a field $F$ and let $0_V \neq w \in V$.  Given a
vector $v \in V \setminus Fw$, find the set $G$ of all scalars $a \in F$
satisfying $F\{v, w\} = F\{v, aw\}$.

\end{problem}
\smallskip
\begin{solution}

(type your solution here)

\end{solution}
\probskip





%% If you will use references, add your refs to the Math700.bib file.
%% and then uncomment the following lines.
%% \bibliographystyle{plain}
%% \bibliography{Math700}

\end{document}
