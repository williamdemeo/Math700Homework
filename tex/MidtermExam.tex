\documentclass[11pt]{paper}

%%%%%%%%%%%%%%%%%%%%%%%%%%%%%%%%%%%%%%%%
% Basic packages
%%%%%%%%%%%%%%%%%%%%%%%%%%%%%%%%%%%%%%%%
\usepackage{amsmath,amsthm,amssymb}
\usepackage[mathscr]{euscript}
\usepackage{mathtools}
\usepackage{etoolbox}
\usepackage{fancyhdr}
\usepackage{xcolor}
\usepackage{hyperref}
\usepackage{xspace}
\usepackage{comment}
\usepackage{url} % for url in bib entries
%\usepackage{mathrsfs}


\theoremstyle{remark}
\newtheorem{theorem}{Theorem}
\newtheorem*{prop}{Proposition}
\newtheorem{problem}{Problem}
\newtheorem*{prob}{Problem}
\newtheorem*{solution}{{\bf Solution}}
\newtheorem*{hint}{{\it Hint}}

%%%%%%%%%%%%%%%%%%%%%%%%%%%%%%%%%%%%%%%%%%%%%%%%%%
%% Surround the problem and solution with 
%% \begin{ProbBox}  and   \end{ProbBox}
%% to prevent pagebreaks.
\newenvironment{ProbBox}{\noindent\begin{minipage}{\linewidth}}{\end{minipage}}

%%%%%%%%%%%%%%%%%%%%%%%%%%%%%%%%%%%%%%%%
% Acronyms
%%%%%%%%%%%%%%%%%%%%%%%%%%%%%%%%%%%%%%%%
\usepackage[acronym, shortcuts]{glossaries}

%% HERE IS HOW YOU DEFINE ACRONYMS:
\newacronym{FTA}{FTA}{Fundamental Theorem of Algebra}
\newacronym{CRT}{CRT}{Chinese Remainder Theorem}

% Make \ac robust.
\robustify{\ac}

\usepackage{enumerate}

\usepackage[
top    = 3cm,
bottom = 3cm,
left   = 3.00cm,
right  = 3.00cm]{geometry}

%%%%%%%%%%%%%%%%%%%%%%%%%%%%%%%%%%%%%%%%
% Fancy page style
%%%%%%%%%%%%%%%%%%%%%%%%%%%%%%%%%%%%%%%%
\pagestyle{fancy}
\newcommand{\metadata}[2]{
  \rhead{Midterm Exam}
  \chead{}
  \lhead{Math 700: Linear Algebra}
%  \lfoot{#1}\cfoot{#2}
  % \lfoot{}\cfoot{}
  % \rfoot{\thepage}
}
\renewcommand{\headrulewidth}{0.4pt}
\renewcommand{\footrulewidth}{0.4pt}

\newrobustcmd*{\vocab}[1]{\emph{#1}}
\newrobustcmd*{\latin}[1]{\textit{#1}}

%%%%%%%%%%%%%%%%%%%%%%%%%%%%%%%%%%%%%%%%
% Customize list enviroonments
%%%%%%%%%%%%%%%%%%%%%%%%%%%%%%%%%%%%%%%%
% package to customize three basic list environments: enumerate, itemize and description.
% \usepackage{enumitem}
% \setitemize{noitemsep, topsep=0pt, leftmargin=*}
% \setenumerate{noitemsep, topsep=0pt, leftmargin=*}
% \setdescription{noitemsep, topsep=0pt, leftmargin=*}

%%%%%%%%%%%%%%%%%%%%%%%%%%%%%%%%%%%%%%%%
%% Space between problems
\newrobustcmd*{\probskip}{\vskip5mm}

\usepackage{tikz}
\usetikzlibrary{matrix,arrows}
%% 
         \newcommand\alg[1]{\ensuremath{\mathbf{#1}}}
         \newcommand{\<}{\ensuremath{\langle}}
         \renewcommand{\>}{\ensuremath{\rangle}}
         \newcommand\fld[1]{\ensuremath{\mathbb{#1}}}
         \newcommand\N{\ensuremath{\fld{N}}}
         \newcommand\GF{\ensuremath{\operatorname{GF}}}
         \newcommand\End{\ensuremath{\operatorname{End}}}
         \newcommand\Hom{\ensuremath{\operatorname{Hom}}}
         \newcommand{\Aff}{\ensuremath{\operatorname{Aff}}}
         \newcommand{\ann}[1]{\ensuremath{\operatorname{ann}(#1)}}
         \newcommand{\id}{\ensuremath{\operatorname{id}}}
         \newcommand{\nulity}[1]{\ensuremath{\operatorname{null}(#1)}}
         \renewcommand{\ker}[1]{\ensuremath{\operatorname{ker}(#1)}}
         \renewcommand{\dim}[1]{\ensuremath{\operatorname{dim}(#1)}}
         \newcommand\im[1]{\ensuremath{\operatorname{im}(#1)}}
         \newcommand{\rank}[1]{\ensuremath{\operatorname{rank}(#1)}}
         \newcommand{\trace}[1]{\ensuremath{\operatorname{trace}(#1)}}
         \renewcommand{\phi}{\ensuremath{\varphi}}
         \newcommand{\Sub}{\ensuremath{\operatorname{Sub}}}
         \renewcommand{\leq}{\ensuremath{\leqslant}}
         \renewcommand{\nleq}{\ensuremath{\nleqslant}}
         \renewcommand{\geq}{\ensuremath{\geqslant}}
         \renewcommand{\lneq}{\ensuremath{\lneqslant}}
         \renewcommand{\gneq}{\ensuremath{\gneqslant}}
         \renewcommand{\ngeq}{\ensuremath{\ngeqslant}}
         \newcommand{\sD}{\ensuremath{\mathscr{D}}}
         \newcommand{\sS}{\ensuremath{\mathscr{S}}}


         \metadata{Math 700}{Midterm Exam -- 2014/03/07}
         \author{}
%%
%%    8. Update the title and date as appropriate.
         \title{Midterm Exam}
         \subtitle{Math 700: Spring 2014}
         \date{6 March 2014}



\begin{document}

\maketitle
\vskip-1cm

\noindent {\bf INSTRUCTIONS:} 
\begin{itemize}
\item 
Solve the problems below. Write up your solutions (neatly!),
giving complete justifications for all arguments, and turn in a hard copy of your
solutions in class on the\\[4pt]
Due Date: {\bf Wednesday, March 19}

\medskip

\item The questions are meant to test your understanding of elementary concepts, and
you should write down definitions of any technical terms you use, even if these
terms are mentioned in the statement of the problem. Of course, you must use
your best judgment about which definitions to state.  (You probably don't want
to define the integers or real numbers, for example.)

\medskip

\item It will help me (and probably your grade) if you do the following:
\begin{enumerate}
\item 
State what you are trying to prove.
\item 
Mention informally how you plan to prove it before giving the details.
\item If you believe your proof is complete, use an end-of-proof symbol (like
  QED or \qedsymbol); on the other hand, if you believe your proof is
  incomplete, say so.
\end{enumerate}
\end{itemize}

\noindent {\bf HONOR CODE:} You are expected to solve the exam problems on your own
  with no outside help.  You may consult the lecture notes and textbook for this
  course only.  No other books or internet usage is allowed.
  If you get stuck, please ask \emph{me} for help, and I may post hints on our wiki page. 


\medskip

\noindent When you finish the exam, please sign the following pledge:\\
\\
``On my honor as a student I,
\underline{\phantom{XXXXXXXXXXXXXXXX}}, have neither
given nor received unauthorized aid on this exam.''
\hbox{} \hskip .5in {\small (Print Name)}\\[4pt]
\begin{flushright} Signature: \underline{\phantom{XXXXXXXXXXXXXXXXXXXX}}
  Date: \underline{\phantom{XXXXXXXXXX}}
\end{flushright}

\medskip

\noindent {\bf NOTATION:}
For the most part, we follow the notation used in the textbook.
Recall that if $V$ is a vector space over the field $F$ and if
$c\in F$, then $\sigma_c v = cv$ for all $v \in V$.
In particular, $\sigma_0$ and $\sigma_1$ denote the zero and identity maps,
respectively. However, when $V$ and $W$ are vector spaces over the same field,
it is clearer to denote their identity maps by $\id_V$ and $\id_W$, resp.
We use $W\leq V$ to denote that $W$ is a subspace of $V$, 
whereas $W\subseteq V$, means that $W$ is a subset of $V$ 
(which may or may not be a subspace).  By an ``$F$-vector space'' we mean a
vector space over the field $F$. If $\phi : V \rightarrow W$, then
$\im{\phi} := \phi(V)$, $\ker{\phi} := \{v \in V : \phi(v) = 0_W\}$,
$\rank{\alpha} := \dim{\im{\alpha}}$, and $\nulity{\alpha} := \dim{\ker{\alpha}}$.
\probskip

%%%%%%%%%%%%%%%%%%%%%%%%%%%%%%%%%%%%%%%%%%%%%%%%%%%%%%%%%%%%%%%%%%%%%%%%%%%%
\begin{problem}
Let $V$ and $W$ be finite dimensional vector spaces over the field $F$, and
suppose $\alpha \in \Hom(V,W)$.
Circle true or false, where true means ``always true'' (no proof required):
\begin{enumerate}[(a)]
\item If $\alpha(v) = 0_W$ only when $v=0_V$, then $\dim{V} = \dim{W}$.
 \hfill true \hskip1cm false
\item If $\im{\alpha} = \{0_W\}$, then $\alpha = \sigma_0$.
 \hfill true \hskip1cm false
\item If $\dim{V} = \rank{\alpha}$, then $\ker{\alpha} = \{0_V\}$.
 \hfill true \hskip1cm false
\item $\ker{\alpha}\leq \ker{\alpha^2}$
 \hfill true \hskip1cm false
\item $\im{\alpha}\geq \im{\alpha^2}$
 \hfill true \hskip1cm false
\item $\nulity{\alpha} \leq \rank{\alpha}$
 \hfill true \hskip1cm false
\item $\nulity{\alpha} \leq \dim{V}$
 \hfill true \hskip1cm false
\item $\alpha$ is a one-to-one if and only if $\ker{\alpha} = \{0_V\}$.
 \hfill true \hskip1cm false
\item $\alpha$ is a one-to-one if and only if $\dim{V} \leq \dim{W}$.
 \hfill true \hskip1cm false
\item $\alpha$ is a one-to-one if and only if $\nulity{\alpha} = 0$.
 \hfill true \hskip1cm false
\item $\alpha$ is a onto if and only if $\dim{V} \geq \dim{W}$.
 \hfill true \hskip1cm false
\item $\alpha$ is a onto if and only if $\rank{\alpha} =\dim{W}$.
 \hfill true \hskip1cm false
\end{enumerate}
\end{problem}

\probskip

%%%%%%%%%%%%%%%%%%%%%%%%%%%%%%%%%%%%%%%%%%%%%%%%%%%%%%%%%%%%%%%%%%%%%%%%%%%%
\begin{problem}
Prove that the lattice of subspaces of a vector space
is modular but not necessarily distributive, as follows:
Let $U$, $Y$, and $W$ be subspaces of a vector space $V$. 
\begin{enumerate}[(a)]
\item Show that $U\cap (Y + (U \cap W)) = (U \cap Y) + (U \cap W)$.
\item Show that $U \cap (Y + W) = (U \cap Y) + (U \cap W)$ is not always valid.
\end{enumerate}
\end{problem}

\probskip

%%%%%%%%%%%%%%%%%%%%%%%%%%%%%%%%%%%%%%%%%%%%%%%%%%%%%%%%%%%%%%%%%%%%%%%%%%%%
\begin{problem}
\label{prob:zorn}
Prove that every nontrivial vector space $V$ has a basis.\\[4pt]
[{\it Hint:} First prove that
every linearly independent subset of $V$ is contained in a basis.
As we did in class, let $S$ be a linearly independent subset and
  let $\sS$ be the set of all linearly independent subsets that contain $S$. 
  Partially order $\sS$ by inclusion $\subseteq$ and apply
  Zorn's Lemma.\footnote{{\bf Zorn's Lemma:} 
    If a partially ordered set $(\sS, \subseteq)$ has the property that every
    chain $S_1 \subseteq S_2 \subseteq \cdots$ has an upper bound in $\sS$, 
    then $\sS$ contains a maximal element.}
  Finally, say why if follows from this that every vector space has a basis.]
\end{problem}

\probskip

%%%%%%%%%%%%%%%%%%%%%%%%%%%%%%%%%%%%%%%%%%%%%%%%%%%%%%%%%%%%%%%%%%%%%%%%%%%%
\begin{problem}
Let $V$ be a vector space over the field $F$, and suppose the subset 
$S \subseteq V$ satisfies $FS = V$. Prove that $S$ contains a basis.
\end{problem}

\probskip

%%%%%%%%%%%%%%%%%%%%%%%%%%%%%%%%%%%%%%%%%%%%%%%%%%%%%%%%%%%%%%%%%%%%%%%%%%%%
\begin{problem}
Let $V$ be a vector space over the field $F$ and let $\Omega$ be a (possibly
uncountable) set.  Describe $V^\Omega$.  Can you make 
$V^\Omega$ into an  $F$-vector space?  An $F$-algebra? Explain.
\end{problem}

\probskip

%%%%%%%%%%%%%%%%%%%%%%%%%%%%%%%%%%%%%%%%%%%%%%%%%%%%%%%%%%%%%%%%%%%%%%%%%%%%
\begin{problem}
Let $V$ and $W$ be vector spaces over the field $F$, and let 
$\phi : V \rightarrow W$ be a linear transformation. Give details and proof of
the following: $\phi$ is injective (respectively, surjective) if and only if 
there is a linear transformation $\psi: W \rightarrow V$ such that the
composition of $\phi$ with $\psi$ is the identity map.\\[4pt]
[{\it Hint:} There are two claims to prove, (a) ``$\phi$
  is injective iff...'', and (b) ``$\phi$ is surjective
  iff...'' There are two ways to form the composition, 
$\phi \psi = \id_W$ and $\psi \phi = \id_V$. Figure out which composition
you need to prove each claim.]
\end{problem}

\probskip

%%%%%%%%%%%%%%%%%%%%%%%%%%%%%%%%%%%%%%%%%%%%%%%%%%%%%%%%%%%%%%%%%%%%%%%%%%%%
\begin{problem} Let $V$ and $W$ be vector spaces over the field $F$.
Recall that $W^V$ denotes the set of maps $\{f : V \rightarrow W\}$.
For fixed $\alpha \in \Hom(V,W)$ and $w \in W$, define the 
\emph{affine transformation} $\zeta_{\alpha, w}: V \rightarrow W$ to be the map 
$v \mapsto \alpha(v) + w$.
Denote by $\Aff (V, W)$ the set of all such affine transformations.  That is,
$\Aff (V, W) := \{\zeta_{\alpha, w} : \alpha \in \Hom(V,W) \text{ and } w \in W\}$.
\begin{enumerate}[(a)]
\item Can you make $W^V$ into an $F$-vector space? An $F$-algebra? Explain.
\item Prove or disprove: $\Hom(V,W) \leq \Aff (V, W) \leq W^{V}$. 
(Interpret $\leq$ here as you see fit.)
\item Note that $\zeta$ (without subscripts) may be viewed as a map from 
$\Hom(V,W)$ to $\Aff(V,W)^W$. Is $\zeta$ a vector space
  homomorphism?  An $F$-algebra homomorphism? Prove.
\end{enumerate}

\end{problem}

\probskip

%%%%%%%%%%%%%%%%%%%%%%%%%%%%%%%%%%%%%%%%%%%%%%%%%%%%%%%%%%%%%%%%%%%%%%%%%%%%
\begin{problem}
\item Let $V$ be a finite dimensional vector space over the field $F$ and 
suppose $T \in \End(V)$.\\
Prove the following:
  \begin{enumerate}[(a)]
  \item
$\{0_V\} \leq \ker{T} \leq \ker{T^2} \leq \cdots$
\item
$V \geq \im{T} \geq \im{T^2} \geq \cdots$
\item $\dim{V} = \rank{T^k} + \nulity{T^k}$, for each $k= 0, 1, \dots$
\item The sets
    $V_1 := \bigcap\limits_{k=1}^\infty \im{T^k}$ and
    $V_2 := \bigcup\limits_{k=1}^\infty \ker{T^k}$ are $T$-invariant subspaces.
  \item $V = V_1 \oplus V_2$.
  \item If $T_i$ is the restriction of $T$ to $V_i$, then 
    $T_1$ is an isomorphism and $T_2$ is
    \emph{nilpotent}.\footnote{$\alpha\in \End(V)$ is called \emph{nilpotent} if
      there is a positive integer $k$ such that $\alpha^k = \sigma_0$.} 
  \end{enumerate}
\end{problem}

\probskip



\end{document}
